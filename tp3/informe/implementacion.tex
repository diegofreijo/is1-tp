\subsection{Servidor}
\subsubsection{Modificaciones respecto al TP2}
Las modificaciones fueron m'inimas, la mayor'ia m'as que nada fueron temas sint'acticos. El 'unico cambio importante fue que un ResultadoCraps posea el valor de punto pose'ido por la mesa al momento de ser ejecutada.

Algunos algoritmos fueron modificados desde un punto de vista sint'actico en la implementaci'on s'olo por simpleza. La mayor'ia se deben al uso de Linq en el manejo de XML, listas y diccionarios.

\subsubsection{Forma de uso}
Para ejecutar el servidor se debe ejecutar \verb|Servidor.exe|, encontrado dentro del directorio \verb|bin|. Los XML de entrada se alojar'an en el directorio \verb|buffer_entrada| y los de salida ser'an depositados en \verb|buffer_salida|. Los archivos de configuraci'on ser'an tomados de la carpeta \verb|config|. Para finalizarlo simplemente cerrar la consola.

\subsection{Clientes}
Al igual que en el caso del servidor, las modificaciones realizadas a los clientes fueron pocas en lo que refiere a dise'no. Sin embargo s'i hay varios cambios en lo que respecta a nombres de m'etodos y una fusi'on entre las clases de los controladores de ventanas y las de ventanas provistas por las herramientas y bibliotecas de desarrollo utilizadas (Visual C$\sharp$ Express usando Window Forms).

\subsubsection{Cliente Administrador}
Para el cliente administrador, con el fin de poder pedir el estado del casino para poder listar las mesas disponibles en caso de necesitarlas para la configuraci'on de la jugada feliz del modo dirigido, fue necesario agregar la clase singleton ``TerminalInfo'' conteniendo informaci'on acerca del id de la terminal virtual, igual a la que ya ten'ia el cliente para jugadores. Para pedir el estado del casino tambi'en es necesario enviar un nombre de usuario v'alido que est'e logueado, lo cual no ten'iamos disponible. Las alternativas que se nos ocurrieron frente a esta situaci'on fueron modificar el protocolo, modificar el servidor para que ni bien arranque autologuee un usuario reservado que pueda usar un cliente administrador y por 'ultima la opci'on que terminamos implementando: agregamos todo lo necesario para que el cliente administrador pueda loguearse moment'aneamente al casino como observador, generando nombres distintos de alguna forma hasta conseguir entrar, pedimos el estado del casino e inmediatamente nos deslogueamos. Elegimos la 'ultima opci'on ya que nos resultaba la menos invasiva para el c'odigo del servidor y no requer'ia modificar el protocolo, aunque no es una forma muy elegante de resolver el problema (fue divertido!).

\subsubsection{Forma de uso}
Debe ejecutarse la aplicaci'on pas'andole un 'unico par'ametro que sea un n'umero en el rango 0-9999 (ser'a el n'umero de terminal virtual utilizado para esa instancia de la aplicaci'n). Creemos que las interfases de usuario logradas son de f'acil uso.

En el cliente jugador hemos utilizado una modalidad para realizar apuestas tal vez poco intuitiva, la que explicaremos a continuaci'on.
El jugador tiene posee en todo momento ``fichas en la mano''. No nos interes'o verificar que las fichas que tiene en mano no excedan el monto disponible que tiene el jugador para apostar. El jugador puede tomar fichas disponibles en esa jornada del casino del monto que desee (seleccionar el valor de la ficha en el combo-box y luego presionar el bot'on ``Pick''). Tambi'en puede dejar todas las fichas que ten'ia en la mano en caso de arrepentirse (usar el bot'on ``DropAll''). Pueden verse las fichas en mano en todo momento en el recuadro de texto correspondiente de la interfaz de usuario. Una vez que el jugador tiene en su mano todas las fichas que desea utilizar para realizar una apuesta debe presionar el bot'on ``Bet'' de la apuesta que desea realizar para efectivizarla. En caso de que la apuesta sea aceptada por el servidor las fichas que el jugador ten'ia en la mano ser'an colocadas en la mesa de juego, quedando su mano vac'ia de fichas.
