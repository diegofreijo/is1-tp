\subsection{ Objetivo del documento	}

En el presente documento ser'a una herramienta en el proceso 
de desarrollo de la informatizaci'on del ``Casino On line'', donde 
esperamos especificar el funcionamiento, operatoria, resposabilidades y alcance
del mismo.


\subsection{ Convenciones de notaci'on	}
\begin{itemize}
    \item En las FSM el estado inicial lleva la etiqueta \textbf{inicial}
\end{itemize}


\subsection{ Destinatarios del documento	}
Los destinatarios del documentos son los SOCIOS. En particular:

\begin{itemize}
    \item Armando Paredes (Jefe)
    \item Lic. Galinardi (Marketing)
    \item Claudio Gallo (Contador)
    \item Dr. Foronga (Contador y mano derecha del Jefe)
\end{itemize}


\subsection{ Descripci'on del problema }
El problema consiste en especificar un sistema que describa el funcionamiento
de un casino online (ver minutas de las reuniones),
el cual constar'a inicialmente de 2 tipos de juegos  ``Craps'' y ``M'aquina Tragamonedas''.

Existe la figura del invitado, el cual podrá mirar cualquier juego.
este podr'a darse de alta, debera tramitarlo con la secretaria.
Así como en un casino convencional los jugadores podrán cambiar 
sus fichas o comprar más para lo cual tambi'en gestionanran esto por fuera del aplicativo con 
la secretaria.

Los jugadores podr'an jugar a ambos juegos. Podr'a haber cualquier cantidad de mesas de ambos
juegos.
Hay clientes V.I.P. quienes podr'an apostar todo lo que quieran (su saldo podr'a ser negativo).

Habr'a tipos de jugadas y pozos especiales. Tambi'en podr'an ``visitar'' el casino personas que no posean una cuenta de usuario, a las que denominaremos ``invitados''.


\subsection{ Documentos relacionados}
El presente informe se basa en la minuta de la segunda reuni'on con clientes del casino (versi'on 1.6) y los varios e-mails enviados a la lista \textit{isoft1-alu@googlegroups.com} hasta el d'ia de la fecha.


\subsection{ Organizaci'on del informe	}
Se espera que el documento pueda ser le'ido por cualquier individuo con conocimientos t'ecnicos m'inimos con excepci'on de algunos diagramas correspondientes a la secci'on de requerimientos espec'ificos. Aqu'i, un mayor conocimiento del significado de los mismos ser'a necesario para su entendimiento.



