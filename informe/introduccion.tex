\subsection{ Objetivo del documento	}
En el presente documento esperamos especificar el funcionamiento del casino online 
que el grupo ``timbalistas'' nos solicit'o y reflejar la operatoria
del mismo. Tambi'en nos gustaria que este documento nos ayude a entender mejor el problema, 
lo que ayudar'a a hacer un mejor sistema y por consiguiente a que nuestros contratistas ganen m'as dinero
y as'i puedan viajar a ``Las Vegas: el para'iso Timbero''.


\subsection{ Convenciones de notaci'on	}
\begin{itemize}
    \item En las FSM el estado inicial lleva la etiqueta \textbf{inicial}
\end{itemize}


\subsection{ Destinatarios del documento	}
Los destinatarios del documentos son los SOCIOS. En particular:

\begin{itemize}
    \item Armando Paredes (Jefe)
    \item Lic. Galinardi (Marketing)
    \item Claudio Gallo (Contador)
    \item Dr. Foronga (Contador y mano derecha del Jefe)
\end{itemize}


\subsection{ Descripci'on del problema }

El problema consiste en especificar el funcionamiento de un casino online, el cual constar'a inicialmente de 2 tipos
de juegos: ``Craps'' y ``M'aquina Tragamonedas''. Los jugadores podr'an jugar a ambos juegos. Podr'a haber
cualquier cantidad de mesas de ambos juegos. Los jugadores tienen una cuenta con dinero, cuenta que se gestiona con la secretaria.
Hay clientes V.I.P. quienes podr'an apostar todo lo que quieran (su saldo podr'a ser negativo).
Habr'a tipos de jugadas y pozos especiales. Tambi'en podr'an ``visitar'' el casino personas que no posean una cuenta de usuario, a las que denominaremos ``invitados''.
El sistema ser'a on line.


\subsection{ Documentos relacionados}
El presente informe se basa en la minuta de la segunda reuni'on con clientes del casino (versi'on 1.6) y los varios e-mails enviados a la lista \textit{isoft1-alu@googlegroups.com} hasta el d'ia de la fecha.


\subsection{ Organizaci'on del informe	}
Se espera que el documento pueda ser le'ido por cualquier individuo con conocimientos t'ecnicos m'inimos con excepci'on de algunos diagramas correspondientes a la secci'on de requerimientos espec'ificos. Aqu'i, un mayor conocimiento del significado de los mismos ser'a necesario para su entendimiento.
