Debido a la necesidad de una interfaz gr'afica de usuario para la utilizaci'on del producto, decidimos adoptar como estrategia para el modelado una aproximaci'on a un sistema de eventos y ventanas tal como los masivamente usados en casi todas las plataformas. En este tipo de sistemas el desarrollador de aplicaciones consta con bibliotecas y APIs que le permiten interactuar con el sistema de ventanas y eventos. Hemos realizado la simplificaci'on y abstracci'on que creemos conveniente para que nuestros diagramas sean simples y al mismo tiempo lo suficientemente expresivos para no dejar de lado cuestiones que hacen al mundo de la programaci'on orientada a eventos.
Presentamos a continuaci'on la descripci'on de los m'odulos componentes.

\subsubsection{GUI}
Es el conjunto de clases que modelan la interfaz gr'afica presentada al usuario.

\subsubsubsection{Clases relevantes}

\begin{description}
\item[\italica{Ventana}] Modela la clase base a toda representaci'on subyacente de un gr'afico de ventana. Generalmente esta clase es provista por una biblioteca. Es de esta clase que, mediante herencia, podemos definir las ventanas personalizadas que utilizan nuestros clientes.
\end{description}

\subsubsection{Controladores GUI}
Consisten en el conjunto de clases que controlan los eventos desencadenados por acciones del usuario en su interacci'on con el sistema de ventanas o generados exclusivamente por el sistema de ventanas. Existe un controlador por cada ventana a excepci'on de ventanas particularmente sencillas de las cuales no resulta imprescindible manejar de una forma interesante los eventos que podr'ia generar un usuario al interaccionar con las mismas.

Nuestro diagrama puede dar una noci'on parcial referente a la apariencia visual de las ventanas gr'aficas definidas si inferimos y caracterizamos de alguna forma intuitiva los nombres de los m'etodos de las clases de ventanas y controladores. Cre'imos apropiado incluir una descripci'on aproximada pero mucho m'as precisa de c'omo lucir'an nuestras interfases gr'aficas en el producto final. Adem'as, es en estas interfases en las que pensamos al momento de dise'nar el cliente, con lo cual hay una correspondencia 1 a 1 (salvando algunos casos) con las clases presentadas en el diagrama.

\subsubsubsection{LogIn}
\imagen{PrototiposPantalla/PlayerClient_SignIn.png}{Prototipo de pantalla de ingreso al casino. Clase \italica{VentanaLogin}}{1}
\clearpage

\subsubsubsection{Lobby}
\imagen{PrototiposPantalla/PlayerClient_Lobby.png}{Prototipo de pantalla del lobby del casino. Clase \italica{VentanaLobby}}{1}
\clearpage

\subsubsubsection{Selecci'on de ficha}
\imagen{PrototiposPantalla/PlayerClient_SelectCoinValue.png}{Prototipo de pantalla de selecci'on de ficha. Clase \italica{VentanaSeleccionarValorFicha}}{1}
\clearpage

\subsubsubsection{Selecci'on de mesa}
\imagen{PrototiposPantalla/PlayerClient_SelectTable.png}{Prototipo de pantalla de selecci'on de mesa. Clase \italica{VentanaSeleccionarMesa}}{1}
\clearpage

\subsubsubsection{Tragamonedas}
\imagen{PrototiposPantalla/PlayerClient_Tragamonedas.png}{Prototipo de pantalla del juego Tragamonedas. Clase \italica{VentanaTragamonedas}}{0.8}
\clearpage

\subsubsubsection{Craps}
\imagenvertical{PrototiposPantalla/PlayerClient_Craps.png}{Prototipo de pantalla del juego Craps. Clase \italica{VentanaCraps}}{0.65}
\clearpage

\subsubsection{Comunicaci'on}
Esquema semejante al que utilizamos para el dise'no del servidor aunque con algunos cambios que vale la pena mencionar.
