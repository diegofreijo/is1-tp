Planteamos 2 enfoques para la construcci'on de los escenarios y diagramas de secuencia. La primera 
fue basada en el echo de que contabamos con protocolos que inducian interaci'on de
nuestras clases. El otro fue basado en los casos de uso.

\textbf{Notaci'on en los diagramas}

Para tratar con los XML en los DS asumiremos que contamos con cirta funcionalidad, provista por los objetos, por ejemplo:
\begin{itemize}
\item  Para obtener el atributo \textbf{idMesa} de un XML ejecutamos \textit{XML.idMesa}, donde idmesa es el tag que se encuentra en el xml.
\item As'i tambi'en para la lista de jugadores, XML.jugadoresEnMesa devuelve una \textbf{lista$<$jugadores$>$}.

\end{itemize}




\escenario{ \textbf{Startup del Casino}
\begin{itemize}
\item se inicia el servidor
\item se leen los achivos de configuraci'on
\end{itemize}

}

\escenario{ \textbf{Startup Cliente}
\begin{itemize}
\item se incia el cliente
\item 
\end{itemize}

}





\subsection{Funcionalidades generales entre los jugadores y el casino}

\escenario{
\begin{itemize}
  \item el jugador entra al casino.
  \item se loguea
  \item el casino autentica sus datos.
  \item da la respuesta
\end{itemize}
}



\escenario{
\begin{itemize}
  \item el jugador le pide salir al casino
  \item el casino verifica si puede hacerlo
  \item da la respuesta
\end{itemize}

}

\escenario{

    \begin{itemize}
      \item elige jugar al tragamonedas, especifica el valor de ficha
      \item el casino valida el valor ingresado, resulta valido
      \item le da el n'umero de mesa
      \item o le niega la entrada
    \end{itemize}

}




\subsection{Funcionalidades referentes a Tragamonedas}

 \escenario{Tirar Tragamonedas}
 { El usuario puede estar en modo jugador o no. Una vez validado el usuario, ahora jugador, puede ingresar al Tragamonedas o no. Puede o no haber hecho una apuesta. Es una jugada normal. Gira los rodillos. La apuesta se resuelve y se le acreditan creditos si corresponde.}
%imagen
\imagenvertical{DS_Tragamonedas/TirarTragamonedas/Tirar Tragamonedas.png}{Tirar Tragamonedas}{0.5}


 \escenario{Entrar Tragamonedas}{
El jugador puede estar en el casino en modo jugador o no.
Podr'ia estar en alguna mesa. En este escenario mostramos en particular como se hace para enviar los mensajes donde se Acepta o Deniega, en este caso la Entrada al juego.
Se muesta la iteracci'on con los mesajeros. 
}

%imagen


\subsection{Funcionalidades referentes a Craps}
\escenario{ Entrar Craps}{
Este escenario es bastante gen'erico. Se muestra como se valida cada cosa, como actua el sistema en cada caso
y que mensaje de error da.

El usuario puede o no estar en el casino en modo jugador.(incluye modo observador o no haber ingresado)
Puede estar en otra mesa o puede desear crearla.
}

% imagen


\escenario{Resolverse Apuesta de Sitio a Ganar}{
La ronda esta en ``Est'an Saliendo'' sali'o un 4. La apuesta se resuelve, pasa a estar cerrada
}

% imagen



\escenario{Resolverse Apuesta Venir}{
Se estableci'o el punto. Se le paga. La apuesta se cierra
}

% imagen


\escenario{ \textbf{Jugador de Craps haciendo una apuesta}
  \begin{itemize}
   \item El usuario esta en una mesa de craps 
  \item elige un valor de ficha de \$20
  \item elige un valor de ficha de \$15
  \item elige el tipo de apuesta
  \item por cada eleccion se ve un mensaje en el log
  \item si elige una apuesta antes de una ficha  da un error
  \end{itemize}
}





\subsection{Funcionalidades generales de los administradores}

En el DS: puede verse la secuencia de un seteo de el modo dirigido de dados.


En el DS: puede verse el pedido de el reporte PedidoReporteDetalleMovimientosPorJugador()




