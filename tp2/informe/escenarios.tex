Planteamos 2 enfoques para la construcci'on de los escenarios y diagramas de secuencia. La primera 
fue basada en el echo de que contabamos con protocolos que inducian interaci'on de
nuestras clases. El otro fue basado en los casos de uso.

\textbf{Notaci'on en los diagramas}

Para tratar con los XML en los DS asumiremos que contamos con cirta funcionalidad, provista por los objetos, por ejemplo:
\begin{itemize}
\item  Para obtener el atributo \textbf{idMesa} de un XML ejecutamos \textit{XML.idMesa}, donde idmesa es el tag que se encuentra en el xml.
\item As'i tambi'en para la lista de jugadores, XML.jugadoresEnMesa devuelve una \textbf{lista$<$jugadores$>$}.

\end{itemize}




\escenario{ \textbf{Startup del Casino}
\begin{itemize}
\item se inicia el servidor
\item se leen los achivos de configuraci'on
\end{itemize}

}

\escenario{ \textbf{Startup Cliente}
\begin{itemize}
\item se incia el cliente
\item 
\end{itemize}

}





\subsection{Funcionalidades generales entre los jugadores y el casino}

\escenario{
\begin{itemize}
  \item el jugador entra al casino.
  \item se loguea
  \item el casino autentica sus datos.
  \item da la respuesta
\end{itemize}
}

\escenario{
\begin{itemize}
  \item el jugador le pide salir al casino
  \item el casino verifica si puede hacerlo
  \item da la respuesta
\end{itemize}

}

\escenario{

    \begin{itemize}
      \item elige jugar al tragamonedas, especifica el valor de ficha
      \item el casino valida el valor ingresado, resulta valido
      \item le da el n'umero de mesa
      \item o le niega la entrada
    \end{itemize}

}




\subsection{Funcionalidades referentes a Tragamonedas}
\escenario
{
\begin{itemize}
 \item Un jugador con una mesa elegida para jugar
\item  inserta una cantidad v'alida de fichas en una m'aquina tragamonedas
\item  gira los rodillos. 
\item 'Esta, luego de debitar el monto establecido con aterioridad correspondiente al pozo progresivo, consulta el tipo de jugada al casino 
\item le responde que es una jugada normal
\item la m'aquina genera el resultado de la jugada en base a las probabilidades que le establecieron. 
\item 'Esta termina arrojando un resultado ganador 
\item el jugador observa el resultado y tipo de jugada 
\item recibe el cobro.

\end{itemize}

}

\escenario
{
\begin{itemize}
  \item Un jugador con una mesa elegida para jugar inserta una cantidad v'alida de fichas en una m'aquina tragamonedas 
  \item gira los rodillos. 
  \item debita el monto establecido con aterioridad correspondiente al pozo progresivo
  \item consulta el tipo de jugada al casino
  \item le responde que es una jugada normal
  \item la m'aquina genera el resultado de la jugada en base a las probabilidades que le establecieron
  \item termina arrojando un resultado perdedor
  \item el jugador observa el resultado y tipo de jugada.
\end{itemize}
}

\escenario
{
Un jugador con una mesa elegida para jugar inserta una cantidad v'alida de fichas en una m'aquina tragamonedas y gira los rodillos. 'Esta, luego de debitar el monto establecido con aterioridad correspondiente al pozo progresivo, consulta el tipo de jugada al casino y le responde que es una jugada todos ponen. Luego la m'aquina genera el resultado de la jugada en base a las probabilidades que le establecieron. 'Esta termina arrojando un resultado ganador para el jugador qui'en luego observa el resultado y tipo de jugada, recibe el cobro y la m'aquina debita en el pozo feliz el importe correspondiente.
}

\escenario
{
Un jugador con una mesa elegida para jugar inserta una cantidad v'alida de fichas en una m'aquina tragamonedas y gira los rodillos. 'Esta, luego de debitar el monto establecido con aterioridad correspondiente al pozo progresivo, consulta el tipo de jugada al casino y le responde que es una jugada feliz. Luego la m'aquina genera el resultado de la jugada en base a las probabilidades que le establecieron. 'Esta termina arrojando un resultado ganador para el jugador. 'Este observa el resultado, el tipo de jugada y luego cobra la suma entre el pago por la apuesta ganada y el pozo feliz hasta ese momento. Luego el pozo feliz es reiniciado a su valor inicial. 
}

\escenario
{
Un jugador con una mesa elegida para jugar inserta la m'axima cantidad de fichas en una m'aquina tragamonedas luego de jugar en ella el mismo importe tantas veces como las necesarias para que pueda ganar el pozo progresivo y gira los rodillos. 'Esta, luego de debitar el monto establecido con aterioridad correspondiente al pozo progresivo, consulta el tipo de jugada al casino y le responde que es una jugada de pozo progresivo. Luego la m'aquina genera el resultado de la jugada en base a las probabilidades que le establecieron. 'Esta termina arrojando el resultado con pago m'aximo. El jugador observa el resultado y tipo de jugada. Luego cobra la suma entre el pago por la apuesta ganada y el monto total del pozo progresivo. Luego el pozo progresivo es reiniciado a su valor inicial.
}



\subsection{Funcionalidades referentes a Craps}
\escenario{ Entrar Craps}{
Este escenario es bastante gen'erico. Se muestra como se valida cada cosa, como actua el sistema en cada caso
y que mensaje de error da.

El usuario puede o no estar en el casino en modo jugador.(incluye modo observador o no haber ingresado)
Puede estar en otra mesa o puede desear crearla.
}
\tam

\clearpage
\escenario{Resolverse Apuesta de Sitio a Ganar}{
La ronda esta en ``Est'an Saliendo'' sali'o un 4. La apuesta se resuelve, pasa a estar cerrada
}
% imagen
\imagenvertical{DS_Craps/CrapsResolverse_ApuestaDeSitioaGanar.png}{Resolverse Apuesta de Sitio a Ganar}{0.5}

\clearpage
\escenario{Resolverse Apuesta Venir}{
Se estableci'o el punto. Se le paga. La apuesta se cierra
}
% imagen

\imagenvertical{DS_Craps/CrapsResolverse_AuestaVenir.png}{Resolverse Apuesta Venir}{0.5}

\escenario{ \textbf{Jugador de Craps haciendo una apuesta} }{
     El usuario esta en una mesa de craps. Elige un valor de ficha de \$20
  Elige un valor de ficha de \$15. Elige el tipo de apuestatem por cada eleccion se ve un mensaje en el log. Si elige una apuesta antes de una ficha  da un error.

}
% imagen




\subsection{Funcionalidades generales de los administradores}
\escenario{Modo Dirigido Craps} { 
Puede verse la secuencia de un seteo de el modo dirigido de craps, donde se setea alg'un valor de cada dado
y que es una jugada \textit{todosPonen}.}
\tam
% \imagen{DS_Admin/ConfigurarModoDirigidoCraps.png}{Modo Dirigido Craps}{0.4}


\escenario{ Modo Dirigido setear JugadaFeliz }{
Puede verse la secuencia de un seteo de el modo dirigido para la jugada Feliz.
}
\tam


\escenario{Pedir Reporte Ranking de Jugadores}{
Un administrador pide el reporte de ranking de jugadores el cual le informa cu'ales son los jugadores m'as ganadores y perdedores en lo que va del d'ia.

Para poder recibirlo debe introducir la password correcta.
}

\imagen{DS_Admin/PedirReporte/DS_PedidoReporteRankingDeJugadores.png}{Validaci'on de la password}{0.5}

\imagen{DS_Admin/Pedir Reporte/DS_RespuestaReporteRankingDeJugadores.png}{Armado del reporte}{0.5}


\escenario{Pedir Reporte de Movimientos}{
Un administrador pide el reporte de todos los movimientos por jugador (apuestas,
premios ganados, monto ganado) desde que ingresaron al casino.

Para poder recibirlo debe introducir la password correcta.
}

\imagen{DS_Admin/PedirReporte/DS_RespuestaReporteMovimientos.png}{Armado del reporte}{0.5}


\escenario{Pedir Estado Actual}{
Un administrador pide el reporte de movimientos el cual le informa cu'ales es el saldo de los jugadores y el saldo del casino.

Para poder recibirlo debe introducir la password correcta.
}

\imagen{DS_Admin/PedirReporte/DS_RespuestaReporteEstadoActual.png}{Armado del reporte}{0.5}



