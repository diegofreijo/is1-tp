Para la construcci'on de los escenarios y diagramas de secuencia fue basada en el echo de que contabamos con los mensajes de protocolo los cuales inducen interaci'on y modificaci'on de nuestro modelo.

Tambi'en decidimos hacer que nuestros escenarios sean lo m'as gen'ericos posible siempre y cuando esto no complique la lectura. 

En cuanto a la profundidad para algunos DS mostramos la interacci'on de punta 
a punta estos se encuentran en la secci'on del mismo nombre.
Factorizandolos en 2 o 3 DS's cada uno. El resto de los diagramas, salvo excepciones, comienzan con la llamada al m'etodo de la fachada del modelo.

Dado que la secci'on de recepci'on de pedidos y el despacho es muy parecida en estos diagramas 
decidimos obviarlos (salvo, claro, en los de punta a punta).


Por simpleza no usamos los \textit{ObtenerInstancia} de los Singletons. Estos estar'an implicitos.


Para tratar con los XML asumiremos que contamos con cierta funcionalidad, provista por los objetos, por ejemplo:
\begin{itemize}
\item  Para obtener el atributo \textbf{idMesa} de un XML usamos \textit{XML.idMesa}, donde idMesa es el tag que se encuentra en el XML, y  \textit{XML.idMesa}, nos devoveria un entero almacenado en ese tag.

\item As'i tambi'en para la lista de jugadores, \textit{XML.jugadoresEnMesa} devuelve una \textbf{lista$<$jugadores$>$}.
\end{itemize}

% 
% \escenario{ \textbf{Startup del Casino}
% \begin{itemize}
% \item se inicia el servidor
% \item se leen los achivos de configuraci'on
% \end{itemize}
% 
% }
% 
% \escenario{ \textbf{Startup Cliente}
% \begin{itemize}
% \item se incia el cliente
% \item ???????????????????????????????????
% \end{itemize}
% 
% }

\subsection{Diagramas de punta a punta}

En esta secci'on mostraremos DS desde que se leen los arhivos XML hasta que el mensajero de salida despacha los archivos XML de respuesta.

\subsubsection{Craps}




\textbf{Tirar Dados}


Este escenario es bastante gen'erico,

\escenario{
      \begin{itemize}
      \item El usuario puede o no estar en la mesa.
      \item El usuario puede o no ser el tirador.
      \item En caso de que sea el tirador y est'e en la mesa.
      \item Se tiran los dados.
      \item Si sali'o alg'un valor propio de punto este se setea.
      \item Hay alguna cantidad de apuestas de alg'un tipo que se resuelven o no con su l'ogica particular, dependiendo del contexto %$^1$
      \item Estamos en un ``est'an saliendo''% $^3$
      \item Es una jugada feliz %$^2$
    \end{itemize}
}

% 
% $^1$ Resoluci'on de distintas apuestas se ve en otros DS's (en sitio a perder y a venir)
% $^2$ En otro DS se ve la situacion con el caso de que el punto esté establecido
% $^3$ En otro DS se ve la jugada Todos ponen

El pozo feliz se reparte si o si el servidor de Jugadas no devuelve una feliz mientras el pozo no lleque al m'inimo.
La notificaci'on a de que una mesa cambió se pude ver en apostar, sin bien est'a instanciado en una mesa, aplica si la mesa fuera gen'erica.


Este DS se dividió en 3 secciones:
\begin{enumerate}
 \item Recepcion de pedido: es la recepcion de pedido,  y la respuesta hacia el modulo de comunciación, no se muestra lo que sucede en la llamada TirarCraps (usuario, unXML)
\item  TirarCraps: Hace todo lo concerniente a la validaci'on, no se hace zoom en TirarDados.
\item TirarDados: aqui puede verse lo que pasa cuando se hace un tirar dados de una mesa
\end{enumerate}

% \subsubsection{Tragamonedas}
% 


% 
% \subsection{Casino}
% %  \escenario{Entrar Casino}{
Un usuario desea entrar al casino, puede hacerlo en modo jugador o en modo observador.
Si ya ha ingresado en modo jugador no se lo dejar'a entrar nuevamente. Si est'a en modo observador y desea ingresar en el mismo modo tampoco podr'a hacerlo.

En cambio si quiere entrar como jugador (independientemente de si ingres'o como observador o si no ingres'o) se deber'a validar que sea un usuario autorizado por marketing:

\begin{itemize}
 \item En caso afirmativo quedar'a ingresado en modo jugador.
 \item En caso negativo quedar'a en modo observador o fuera del casino seg'un cual fuese su estado anterior.
\end{itemize}
 }
\imagen{DS_Casino/EntarCasino/DS_EntrarCasinoFueraDelModelo.png}{Entrar casino fuera del Modelo}{0.6}
%aca iria una imagen------------------------------------------------------------
Entrar casino dentro del Modelo \tam

\clearpage



%aca iria una imagen------------------------------------------------------------

\clearpage

\textbf{Pedir Estado Casino}

Un usuario desea informarse sobre el estado del casino. Se le informar'a s'olo si ha ingresado en el casino, m'as haya si es en modo jugador o en modo observador.

El estado del casino est'a formado por:

\begin{itemize}
 \item La lista de jugadores y observadores ingresados en el casino;
 \item El valor del pozo feliz y del pozo progresivo;
 \item El estado de las mesas del juegos de craps:
	\begin{itemize}
	 \item Los jugadores;
	 \item El 'ultimo tirador y el pr'oximo;
	 \item Si el siguiente es tiro de salida o ya est'a el punto establecido;
	 \item El valor de los dados en el 'ultimo tiro;
	\end{itemize}
 \item El estado de las mesas del juego tragamonedas:
	\begin{itemize}
 	 \item Los jugadores;
 	 \item El valor de los rodillos en el 'ultimo tiro;
 	 \item El 'ultimo tirador y el pr'oximo;
	\end{itemize}
\end{itemize}
\tam

%aca iria una imagen------------------------------------------------------------

\clearpage

\textbf{Salir Casino}

Usuario desea salir del casino.

Si ha ingresado como observador no tendr'a ning'un tipo de validaci'on, por consiguiente tampoco problemas.

Si ha ingresado como jugador, para poder salir deber'a estar fuera de toda mesa. Es decir, no puede pretender salir del casino si es que est'a dentro de una mesa jugando.

\imagen{DS_Casino/SalirCasino/DS_SalirCasino.png}{Salir del Casino}{0.6}

%aca iria una imagen------------------------------------------------------------




% 
% \subsection{Diagr'amas de que no son de punta a punta}
% Dado que la secci'on de recepci'on de pedidos y el despacho es muy parecida en estos diagramas decidimos obviarlo.
% 
% \subsubsection{Funcionalidades de inicializaci'on}
% Los escenarios aqui presentados son muy gen'ericos.


\escenario{ la Configuraci'on general del Casino}
{
Se setea el valor de las fichas, el saldo del casino y la pasword del aministrador
}
% imagen

\imagen{DS_InicioServidor/DS_InicializarConfiguracion.png}{Inicializar Configuraci'on}{0.5}

\escenario{Jugadores Registrados}{
En este DS se ve como se setea la lista de jugadores registrados del casino
}
\imagenvertical{DS_InicioServidor/DS_InicializarJugadoresRegistrados.png}{Inicializar Jugadores Registrados}{0.4}

% imagen


\escenario{Inicializar Mesas}{
En este DS puede verse como se inicalizan las mesas abiertas, se asigna el observador de cambios.
}
\imagen{DS_InicioServidor/DS_InicializarMesas.png}{Inicializar Mesas}{0.5}
% imagen


\escenario{Inicio del Servidor}{
En este DS se puede ver como se ``enciende'' el servidor,
como se crea el el Obtenerdor de pedidos, el receptor de pedidos de archivos,  

}
\imagen{DS_InicioServidor/DS_InicioServidor.png}{Inicio Servidor}{0.5}

% 
% \subsubsection{Funcionalidades generales de los administradores}
% \escenario{Modo Dirigido Craps} { 
Puede verse la secuencia de un seteo de el modo dirigido de craps, donde se setea alg'un valor de cada dado
y que es una jugada \textit{todosPonen}.}
\tam
% \imagen{DS_Admin/ConfigurarModoDirigidoCraps.png}{Modo Dirigido Craps}{0.4}


\escenario{ Modo Dirigido setear JugadaFeliz }{
Puede verse la secuencia de un seteo de el modo dirigido para la jugada Feliz.
}
\tam


\escenario{Pedir Reporte Ranking de Jugadores}{
Un administrador pide el reporte de ranking de jugadores el cual le informa cu'ales son los jugadores m'as ganadores y perdedores en lo que va del d'ia.

Para poder recibirlo debe introducir la password correcta.
}

\imagen{DS_Admin/PedirReporte/DS_PedidoReporteRankingDeJugadores.png}{Validaci'on de la password}{0.5}

\imagen{DS_Admin/Pedir Reporte/DS_RespuestaReporteRankingDeJugadores.png}{Armado del reporte}{0.5}


\escenario{Pedir Reporte de Movimientos}{
Un administrador pide el reporte de todos los movimientos por jugador (apuestas,
premios ganados, monto ganado) desde que ingresaron al casino.

Para poder recibirlo debe introducir la password correcta.
}

\imagen{DS_Admin/PedirReporte/DS_RespuestaReporteMovimientos.png}{Armado del reporte}{0.5}


\escenario{Pedir Estado Actual}{
Un administrador pide el reporte de movimientos el cual le informa cu'ales es el saldo de los jugadores y el saldo del casino.

Para poder recibirlo debe introducir la password correcta.
}

\imagen{DS_Admin/PedirReporte/DS_RespuestaReporteEstadoActual.png}{Armado del reporte}{0.5}



% 
% \subsubsection{Funcionalidades generales del casino}
%  \escenario{Entrar Casino}{
Un usuario desea entrar al casino, puede hacerlo en modo jugador o en modo observador.
Si ya ha ingresado en modo jugador no se lo dejar'a entrar nuevamente. Si est'a en modo observador y desea ingresar en el mismo modo tampoco podr'a hacerlo.

En cambio si quiere entrar como jugador (independientemente de si ingres'o como observador o si no ingres'o) se deber'a validar que sea un usuario autorizado por marketing:

\begin{itemize}
 \item En caso afirmativo quedar'a ingresado en modo jugador.
 \item En caso negativo quedar'a en modo observador o fuera del casino seg'un cual fuese su estado anterior.
\end{itemize}
 }
\imagen{DS_Casino/EntarCasino/DS_EntrarCasinoFueraDelModelo.png}{Entrar casino fuera del Modelo}{0.6}
%aca iria una imagen------------------------------------------------------------
Entrar casino dentro del Modelo \tam

\clearpage



%aca iria una imagen------------------------------------------------------------

\clearpage

\textbf{Pedir Estado Casino}

Un usuario desea informarse sobre el estado del casino. Se le informar'a s'olo si ha ingresado en el casino, m'as haya si es en modo jugador o en modo observador.

El estado del casino est'a formado por:

\begin{itemize}
 \item La lista de jugadores y observadores ingresados en el casino;
 \item El valor del pozo feliz y del pozo progresivo;
 \item El estado de las mesas del juegos de craps:
	\begin{itemize}
	 \item Los jugadores;
	 \item El 'ultimo tirador y el pr'oximo;
	 \item Si el siguiente es tiro de salida o ya est'a el punto establecido;
	 \item El valor de los dados en el 'ultimo tiro;
	\end{itemize}
 \item El estado de las mesas del juego tragamonedas:
	\begin{itemize}
 	 \item Los jugadores;
 	 \item El valor de los rodillos en el 'ultimo tiro;
 	 \item El 'ultimo tirador y el pr'oximo;
	\end{itemize}
\end{itemize}
\tam

%aca iria una imagen------------------------------------------------------------

\clearpage

\textbf{Salir Casino}

Usuario desea salir del casino.

Si ha ingresado como observador no tendr'a ning'un tipo de validaci'on, por consiguiente tampoco problemas.

Si ha ingresado como jugador, para poder salir deber'a estar fuera de toda mesa. Es decir, no puede pretender salir del casino si es que est'a dentro de una mesa jugando.

\imagen{DS_Casino/SalirCasino/DS_SalirCasino.png}{Salir del Casino}{0.6}

%aca iria una imagen------------------------------------------------------------




% 
% \subsubsection{Tragamonedas}
% \escenario
{
\begin{itemize}
 \item Un jugador con una mesa elegida para jugar
\item  inserta una cantidad v'alida de fichas en una m'aquina tragamonedas
\item  gira los rodillos. 
\item 'Esta, luego de debitar el monto establecido con aterioridad correspondiente al pozo progresivo, consulta el tipo de jugada al casino 
\item le responde que es una jugada normal
\item la m'aquina genera el resultado de la jugada en base a las probabilidades que le establecieron. 
\item 'Esta termina arrojando un resultado ganador 
\item el jugador observa el resultado y tipo de jugada 
\item recibe el cobro.

\end{itemize}

}

\escenario
{
\begin{itemize}
  \item Un jugador con una mesa elegida para jugar inserta una cantidad v'alida de fichas en una m'aquina tragamonedas 
  \item gira los rodillos. 
  \item debita el monto establecido con aterioridad correspondiente al pozo progresivo
  \item consulta el tipo de jugada al casino
  \item le responde que es una jugada normal
  \item la m'aquina genera el resultado de la jugada en base a las probabilidades que le establecieron
  \item termina arrojando un resultado perdedor
  \item el jugador observa el resultado y tipo de jugada.
\end{itemize}
}

\escenario
{
Un jugador con una mesa elegida para jugar inserta una cantidad v'alida de fichas en una m'aquina tragamonedas y gira los rodillos. 'Esta, luego de debitar el monto establecido con aterioridad correspondiente al pozo progresivo, consulta el tipo de jugada al casino y le responde que es una jugada todos ponen. Luego la m'aquina genera el resultado de la jugada en base a las probabilidades que le establecieron. 'Esta termina arrojando un resultado ganador para el jugador qui'en luego observa el resultado y tipo de jugada, recibe el cobro y la m'aquina debita en el pozo feliz el importe correspondiente.
}

\escenario
{
Un jugador con una mesa elegida para jugar inserta una cantidad v'alida de fichas en una m'aquina tragamonedas y gira los rodillos. 'Esta, luego de debitar el monto establecido con aterioridad correspondiente al pozo progresivo, consulta el tipo de jugada al casino y le responde que es una jugada feliz. Luego la m'aquina genera el resultado de la jugada en base a las probabilidades que le establecieron. 'Esta termina arrojando un resultado ganador para el jugador. 'Este observa el resultado, el tipo de jugada y luego cobra la suma entre el pago por la apuesta ganada y el pozo feliz hasta ese momento. Luego el pozo feliz es reiniciado a su valor inicial. 
}

\escenario
{
Un jugador con una mesa elegida para jugar inserta la m'axima cantidad de fichas en una m'aquina tragamonedas luego de jugar en ella el mismo importe tantas veces como las necesarias para que pueda ganar el pozo progresivo y gira los rodillos. 'Esta, luego de debitar el monto establecido con aterioridad correspondiente al pozo progresivo, consulta el tipo de jugada al casino y le responde que es una jugada de pozo progresivo. Luego la m'aquina genera el resultado de la jugada en base a las probabilidades que le establecieron. 'Esta termina arrojando el resultado con pago m'aximo. El jugador observa el resultado y tipo de jugada. Luego cobra la suma entre el pago por la apuesta ganada y el monto total del pozo progresivo. Luego el pozo progresivo es reiniciado a su valor inicial.
}

% 
% 
% \subsubsection{Craps}
% \escenario{ Entrar Craps}{
Este escenario es bastante gen'erico. Se muestra como se valida cada cosa, como actua el sistema en cada caso
y que mensaje de error da.

El usuario puede o no estar en el casino en modo jugador.(incluye modo observador o no haber ingresado)
Puede estar en otra mesa o puede desear crearla.
}
\tam

\clearpage
\escenario{Resolverse Apuesta de Sitio a Ganar}{
La ronda esta en ``Est'an Saliendo'' sali'o un 4. La apuesta se resuelve, pasa a estar cerrada
}
% imagen
\imagenvertical{DS_Craps/CrapsResolverse_ApuestaDeSitioaGanar.png}{Resolverse Apuesta de Sitio a Ganar}{0.5}

\clearpage
\escenario{Resolverse Apuesta Venir}{
Se estableci'o el punto. Se le paga. La apuesta se cierra
}
% imagen

\imagenvertical{DS_Craps/CrapsResolverse_AuestaVenir.png}{Resolverse Apuesta Venir}{0.5}

\escenario{ \textbf{Jugador de Craps haciendo una apuesta} }{
     El usuario esta en una mesa de craps. Elige un valor de ficha de \$20
  Elige un valor de ficha de \$15. Elige el tipo de apuestatem por cada eleccion se ve un mensaje en el log. Si elige una apuesta antes de una ficha  da un error.

}
% imagen


% 
% 
