La arquitectura del casino se compone por:

\begin{description}
\item[Cliente Jugador] Sirve como interfaz gr'afica utilizada por los usuarios finales encargada de transformar las acciones de ellos en el pedido correspondiente y enviarlo al servidor. Adem'as es el encargado de reflejar la respuesta de 'este en la interfaz para informaci'on del usuario.
\item[Cliente Administrador] Es la interfaz que permite a administradores la configuraci'on del modo dirigido y el pedido de reportes.
\item[Servidor] Es la aplicaci'on adonde los clientes envian sus pedidos y la cual les responde con la informaci'on necesitada. Lleva el estado de todas las entidades del casino y contiene la l'ogica establecida por cada uno de ellos.
\end{description}

La comunicaci'on entre ellos ya fue resuelta por el documento de Arquitectura Conceptual y Protocolo brindado por los Timbalistas (ver documento adjunto) y por las extensiones realizadas al mismo (ver secci'on \ref{Seccion::ModificacionesAlProtocolo}). No existir'a comunicaci'on entre los componentes {\it Cliente} desde el punto de vista arquitectural, aunque s'i se relacionar'an indirectamente a trav'es de la l'ogica de negocio del servidor.
