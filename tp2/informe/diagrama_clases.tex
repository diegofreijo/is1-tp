\subsection{Gr'afico Diagrama de Clases}

El enfoque en este diagrama es sobre el \italica{Servidor}.

El grafico del modelo va en formato digital

% A continuaci'on tenemos el gr'afico del diagrama de clases para la resoluci'on del problema sobre el casino.
% 

% y
% \imagenvertical{DC_DiagramaCompleto_vista_control.png}{Diagrama de Clases}{0.18}
% 
% \imagenvertical{DC_DiagramaCompleto_modelo.png}{Diagrama de Clases}{0.18}

\clearpage

\subsection{Explicaci'on del Diagrama de Clases}

El m'odulo del casino se puede separar en dos partes: 

\begin{itemize}
\item componentes \italica{Cliente}
\item componente \italica{Servidor}
\end{itemize}  

La comunicaci'on entre el componente {\it Servidor} y los componentes {\it Cliente} ya fue resuelta por el documento de Arquitectura Conceptual y Protocolo brindado por los clientes (ver documento adjunto) y por las extensiones realizadas al mismo (ver \ref{ModificacionesAlProtocolo}).
No existir'a comunicaci'on entre los componentes {\it Cliente} desde el punto de vista arquitectural, aunque s'i se relacionar'an indirectamente a trav'es de la l'ogica de negocio del servidor.

A continuacion tenemos la explicaci'on de las clases (o conjunto de ellas) m'as relevantes del diagrama.

\begin{itemize}
 \item \italica{Clases AdministradorJugador}, \italica{AdministradorObservador} y \italica{FachadaUsuario}. 'Estas clases hacen el manejo del ingreso y egreso de jugadores y observadores. Adem'as de las consultas sobre el saldo, la existencia (entre otras) de jugadores y observadores

\item La \italica{Clase Usuarios} contiene la lista de jugadores (normales y vip) y la lista de observadores.
Sus operaciones est'an relacionadas con el manejo de las antedichas listas y obtenci'on de datos sobre el jugador, mediante \italica{ObtenerSaldoJugador}

\item \italica{Clase AdministradorMesaCraps} es la responsable del ingreso y egreso de jugadores en las mesas, decir qu'e mesas est'an abiertas, ultimos resultados, entre otros. B'asicamente se ocupa de administrar todo lo referente a las mesas y los tiros del juego craps.

\item \italica{Clase AdministradorMesaTragamonedas} al igual que la anterior administra lo referente a las mesas y los tiros, pero en este caso es sobre el juego de Tragamonedas.

\item La \italica{Clase Mesas} contiene la lista de mesas de craps y la lista de mesas tragamonedas. La \italica{Clase MesaTragamoneda} y la \italica{Clase MesaCraps} son las representaciones de las mesas del juego Tragamonedas y del juego Craps respectivamente. 

\item La \italica{Clase JugadaCraps} contiene el resultado de los dados, a traves de la \italica{Clase Dado}. Tambi'en contiene las apuestas efectudas y por qu'e importe.

\item La \italica{Clase JugadaTragamonedas} contiene el resultado de los rodillos, a traves de la \italica{Clase ResultadoTragamonedas}. Pues esta contiene la \italica{Clase RodilloTragamonedas} que tiene como atributo el valor del rodillo.
Tambi'en se relaciona con la \italica{Clase PozoProgresivo}, de la cual se obtendr'a el monto para realizar el pago y el consiguiente reseteo del mismo.

\item La \italica{Clase AdministradorPozos} muestra los saldos del pozo progresivo y del pozo feliz.

\item La \italica{Clase ReceptorPedidos} mediante su especializaci'on levantar'a los distintos pedidos de la manera que sea necesaria.

\item La \italica{Clase DespachadorRespuestas} es an'aloga a la anterior y tiene el mismo comportamiento diferenci'andose en que es para las respuestas. Hay una por puerto.

\item \italica{Clase DespachadorPedidos} interpretar'a el pedido y seg'un de qui'en provenga se lo dar'a para procesar a alguna de las Clases \italica{JuegoCraps}, \italica{AccesoYVistaCraps}, \italica{AccesoYVistaTragamonedas} o \italica{AccesoYVistaCasino}. 

\end{itemize}



