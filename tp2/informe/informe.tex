\documentclass[spanish, a4paper, 10pt, titlepage]{article}
\author{Echevarr'ia - Farjat - Freijo - Giusto}

% Incluyo los paquetes y configuraciones
% Archivo de configuracion del informe
% -------------------------------------------

\usepackage[spanish,activeacute]{babel}								% Idioma castellano
\usepackage{caratula}														% Caratula de Algo2
%\usepackage[a4paper=true,pagebackref=true]{hyperref}				% Agrega la TOC al PDF e hipervinculos
\usepackage{graphicx} 														% Permite insertar graficos
\usepackage{fancyhdr}														% Permite manejo de cabeceras de pagina
\usepackage{eufrak}															% Usado en el enunciado del trabajo
\usepackage{latexsym}
%\usepackage{algorithmic}													% Para escribir los algos
%\usepackage{dsfont}															% Para el simbolo de naturales
\usepackage[font=small,labelfont=bf]{caption}						% Para editar las captions
%\usepackage{array}

\usepackage[utf8]{inputenc}
\usepackage{amsmath}
\usepackage[x11names, rgb]{xcolor}
% \usepackage{tikz}
% 	\usetikzlibrary{snakes,arrows,shapes}
\usepackage{listings}
\usepackage{lastpage}
\usepackage{geometry}
  \geometry{left=1cm, right=1cm, top=2cm, bottom=2cm}

% Estilo de pagina para tener las cabeceras y pieseras
\pagestyle{fancy}
  \fancyhead[LO]{Ingenier\'ia de Software I}
  \fancyhead[C]{Trabajo Práctico}
  \fancyhead[RO]{Primer Cuatrimestre 2008}
  \renewcommand{\headrulewidth}{0.4pt}

  \fancyfoot[LO]{Echevarr\'ia - Farjat - Freijo - Giusto}
  \fancyfoot[C]{}
  \fancyfoot[RO]{P\'agina \thepage\ de \pageref{LastPage}}
  \renewcommand{\footrulewidth}{0.4pt}

\parindent = 1.5 em 
\parskip = 8 pt


%%%%%%%%%%%%%%%% COMANDOS %%%%%%%%%%%%%%%%%%%%
\newcommand{\todo}{{\large\textbf{TODO: }}}
\newcommand{\paso}{\textsc{Paso }}
\newcommand{\func}[1]{\verb"#1"}
\newcommand{\imagen}[3]
{
	\begin{figure}[p!hbt]
	  \centering
	    \includegraphics[scale=#3]{../img/#1}
	  \caption{#2}
	\end{figure}
}
\newcommand{\imagenvertical}[3]
{
	\begin{figure}[p!hbt]
	  \centering
	    \includegraphics[angle=90,scale=#3]{../img/#1}
	  \caption{#2}
	\end{figure}
}

\newcommand{\nat}{\mathds{N}}
\newcommand{\algoritmo}[3]{\noindent {\bf\underline{#1}:} #2 $\longrightarrow$ #3}
\newcommand{\superindice}[1]{$^\textrm{{\tiny #1}}$}
\newcommand{\subsubsubsection}[1]{\noindent\negrita{#1}

}
\newcommand{\negrita}[1]{{\bf #1}}
\newcommand{\italica}[1]{{\it #1}}

% -----------------------------------------------------------------------------------------------------
% Codigo para generar los casos de uso, basado en caratula.sty del DC-FCEN-UBA 
% Autor: Nicolas Rosner
% Modificado por Pablo Echevarria 25-5-08
% ToDo: ver como declarar primero titulo y despues operaciones, ver como numerar automaticamente las operaciones
% -----------------------------------------------------------------------------------------------------
% Token list para las instrucciones ----
\newtoks\oplist\oplist={}

% Comando para que el usuario agregue operaciones del CU
% Uso: \op{Caso normal}{Caso alternativo}
\newcommand{\op}[2]{\oplist=\expandafter{\the\oplist
\hline#1&#2\\ }}

% Comando para generar el CU con las operaciones ya pasadas y dandole la info que falta
% Uso: \cu{Nombre CU}{Actor primario}{Actores secundarios}{Precondicion}{Postcondicion}{Aclaraciones generales a todos los pasos}
\newcommand{\cu}[6]{
\begin{center}	
\begin{tabular}{|p{11cm}|p{5cm}|}
\hline
\multicolumn{2}{|l|}{\begin{large}\italica{\negrita{CASO DE USO:} #1}\end{large}}\\[0.2em]
\hline
\multicolumn{2}{|l|}{\negrita{Actor Primario:} #2}\\[0.2em]
\hline
\multicolumn{2}{|l|}{\negrita{Actor Secundario:} #3 }\\[0.2em]
\hline
\multicolumn{2}{|l|}{\negrita{Precondici'on:} #4 }\\[0.2em]
\hline
\multicolumn{2}{|l|}{\negrita{Postcondici'on:} #5}\\[0.2em]
\hline
\multicolumn{2}{|l|}{}\\[0.2em]
\hline
\negrita{Curso Normal} & \negrita{Curso Alternativo}\\[0.2em]
\the\oplist
\hline
\multicolumn{2}{|l|}{}\\[0.2em]
\hline
%\multicolumn{2}{|1|}{#6}\\[0.2em]
%\hline
\end{tabular}
\end{center}
\oplist={}
}




%%%%%%%%%%%%%%%% FIN COMANDOS %%%%%%%%%%%%%%%%%%%%


%%%%%%%%%%%%%%%%%%%%%%%%%%%%%%%%%%%%%%%%%%%%%%%%%%%%%%%%%%%%%%
%%%%%%%%%%%%%%%%%%%%%%%%%%%%%%%%%%%%%%%%%%%%%%%%%%%%%%%%%%%%%%%%%%%%%%%%%%%%%%%%%%%%
%%%%%   Inicio del documento
%%%%%%%%%%%%%%%%%%%%%%%%%%%%%%%%%%%%%%%%%%%%%%%%%%%%%%%%%%%%%%%%%%%%%%%%%%%%%%%%%%%%
%%%%%%%%%%%%%%%%%%%%%%%%%%%%%%%%%%%%%%%%%%%%%%%%%%%%%%%%%%%%%%
\begin{document}

% Caratula y tabla de contenidos
\materia{Ingenier'ia de Software I}
\submateria{Primer Cuatrimestre 2008}
\titulo{Trabajo Pr'actico - Parte III}
\subtitulo{Implementaci'on}
\grupo{Grupo 5}

	\integrante{Echevarria, Pablo}{133/00}{pablohe@gmail.com}
	\integrante{Farjat, Lucas}{468/05}{lacacks@gmail.com}
	\integrante{Freijo, Diego}{4/05}{giga.freijo@gmail.com}
	\integrante{Giusto, Maximiliano}{486/05}{maxi.giusto@gmail.com}

\maketitle
\clearpage



\tableofcontents
\clearpage

% ------------------------------------------------------
% Secciones
% ------------------------------------------------------


% Introduccion
\section{Introducci'on}
% Introduccion. Cambios con respecto al informe 1.

En esta secci'on describiremos los cambios con respecto al informe 1.


\clearpage
 
% Escenarios
\section{Escenarios y Diagramas de Secuencia}
Para la construcci'on de los escenarios y diagramas de secuencia fue basada en el echo de que contabamos con los mensajes de protocolo los cuales inducen interaci'on y modificaci'on de nuestro modelo.

Tambi'en decidimos hacer que nuestros escenarios sean lo m'as gen'ericos posible siempre y cuando esto no complique la lectura. 

En cuanto a la profundidad para algunos DS mostramos la interacci'on de punta 
a punta estos se encuentran en la secci'on del mismo nombre.
Factorizandolos en 2 o 3 DS's cada uno. El resto de los diagramas, salvo excepciones, comienzan con la llamada al m'etodo de la fachada del modelo.

Dado que la secci'on de recepci'on de pedidos y el despacho es muy parecida en estos diagramas 
decidimos obviarlos (salvo, claro, en los de punta a punta).


Por simpleza no usamos los \textit{ObtenerInstancia} de los Singletons. Estos estar'an implicitos.


Para tratar con los XML asumiremos que contamos con cierta funcionalidad, provista por los objetos, por ejemplo:
\begin{itemize}
\item  Para obtener el atributo \textbf{idMesa} de un XML usamos \textit{XML.idMesa}, donde idMesa es el tag que se encuentra en el XML, y  \textit{XML.idMesa}, nos devoveria un entero almacenado en ese tag.

\item As'i tambi'en para la lista de jugadores, \textit{XML.jugadoresEnMesa} devuelve una \textbf{lista$<$jugadores$>$}.
\end{itemize}

% 
% \escenario{ \textbf{Startup del Casino}
% \begin{itemize}
% \item se inicia el servidor
% \item se leen los achivos de configuraci'on
% \end{itemize}
% 
% }
% 
% \escenario{ \textbf{Startup Cliente}
% \begin{itemize}
% \item se incia el cliente
% \item ???????????????????????????????????
% \end{itemize}
% 
% }

\subsection{Diagramas de punta a punta}

En esta secci'on mostraremos DS desde que se leen los arhivos XML hasta que el mensajero de salida despacha los archivos XML de respuesta.

\subsubsection{Craps}




\textbf{Tirar Dados}


Este escenario es bastante gen'erico,

\escenario{
      \begin{itemize}
      \item El usuario puede o no estar en la mesa.
      \item El usuario puede o no ser el tirador.
      \item En caso de que sea el tirador y est'e en la mesa.
      \item Se tiran los dados.
      \item Si sali'o alg'un valor propio de punto este se setea.
      \item Hay alguna cantidad de apuestas de alg'un tipo que se resuelven o no con su l'ogica particular, dependiendo del contexto %$^1$
      \item Estamos en un ``est'an saliendo''% $^3$
      \item Es una jugada feliz %$^2$
    \end{itemize}
}

% 
% $^1$ Resoluci'on de distintas apuestas se ve en otros DS's (en sitio a perder y a venir)
% $^2$ En otro DS se ve la situacion con el caso de que el punto esté establecido
% $^3$ En otro DS se ve la jugada Todos ponen

El pozo feliz se reparte si o si el servidor de Jugadas no devuelve una feliz mientras el pozo no lleque al m'inimo.
La notificaci'on a de que una mesa cambió se pude ver en apostar, sin bien est'a instanciado en una mesa, aplica si la mesa fuera gen'erica.


Este DS se dividió en 3 secciones:
\begin{enumerate}
 \item Recepcion de pedido: es la recepcion de pedido,  y la respuesta hacia el modulo de comunciación, no se muestra lo que sucede en la llamada TirarCraps (usuario, unXML)
\item  TirarCraps: Hace todo lo concerniente a la validaci'on, no se hace zoom en TirarDados.
\item TirarDados: aqui puede verse lo que pasa cuando se hace un tirar dados de una mesa
\end{enumerate}

% \subsubsection{Tragamonedas}
% 


% 
% \subsection{Casino}
% %  \escenario{Entrar Casino}{
Un usuario desea entrar al casino, puede hacerlo en modo jugador o en modo observador.
Si ya ha ingresado en modo jugador no se lo dejar'a entrar nuevamente. Si est'a en modo observador y desea ingresar en el mismo modo tampoco podr'a hacerlo.

En cambio si quiere entrar como jugador (independientemente de si ingres'o como observador o si no ingres'o) se deber'a validar que sea un usuario autorizado por marketing:

\begin{itemize}
 \item En caso afirmativo quedar'a ingresado en modo jugador.
 \item En caso negativo quedar'a en modo observador o fuera del casino seg'un cual fuese su estado anterior.
\end{itemize}
 }
\imagen{DS_Casino/EntarCasino/DS_EntrarCasinoFueraDelModelo.png}{Entrar casino fuera del Modelo}{0.6}
%aca iria una imagen------------------------------------------------------------
Entrar casino dentro del Modelo \tam

\clearpage



%aca iria una imagen------------------------------------------------------------

\clearpage

\textbf{Pedir Estado Casino}

Un usuario desea informarse sobre el estado del casino. Se le informar'a s'olo si ha ingresado en el casino, m'as haya si es en modo jugador o en modo observador.

El estado del casino est'a formado por:

\begin{itemize}
 \item La lista de jugadores y observadores ingresados en el casino;
 \item El valor del pozo feliz y del pozo progresivo;
 \item El estado de las mesas del juegos de craps:
	\begin{itemize}
	 \item Los jugadores;
	 \item El 'ultimo tirador y el pr'oximo;
	 \item Si el siguiente es tiro de salida o ya est'a el punto establecido;
	 \item El valor de los dados en el 'ultimo tiro;
	\end{itemize}
 \item El estado de las mesas del juego tragamonedas:
	\begin{itemize}
 	 \item Los jugadores;
 	 \item El valor de los rodillos en el 'ultimo tiro;
 	 \item El 'ultimo tirador y el pr'oximo;
	\end{itemize}
\end{itemize}
\tam

%aca iria una imagen------------------------------------------------------------

\clearpage

\textbf{Salir Casino}

Usuario desea salir del casino.

Si ha ingresado como observador no tendr'a ning'un tipo de validaci'on, por consiguiente tampoco problemas.

Si ha ingresado como jugador, para poder salir deber'a estar fuera de toda mesa. Es decir, no puede pretender salir del casino si es que est'a dentro de una mesa jugando.

\imagen{DS_Casino/SalirCasino/DS_SalirCasino.png}{Salir del Casino}{0.6}

%aca iria una imagen------------------------------------------------------------




% 
% \subsection{Diagr'amas de que no son de punta a punta}
% Dado que la secci'on de recepci'on de pedidos y el despacho es muy parecida en estos diagramas decidimos obviarlo.
% 
% \subsubsection{Funcionalidades de inicializaci'on}
% Los escenarios aqui presentados son muy gen'ericos.


\escenario{ la Configuraci'on general del Casino}
{
Se setea el valor de las fichas, el saldo del casino y la pasword del aministrador
}
% imagen

\imagen{DS_InicioServidor/DS_InicializarConfiguracion.png}{Inicializar Configuraci'on}{0.5}

\escenario{Jugadores Registrados}{
En este DS se ve como se setea la lista de jugadores registrados del casino
}
\imagenvertical{DS_InicioServidor/DS_InicializarJugadoresRegistrados.png}{Inicializar Jugadores Registrados}{0.4}

% imagen


\escenario{Inicializar Mesas}{
En este DS puede verse como se inicalizan las mesas abiertas, se asigna el observador de cambios.
}
\imagen{DS_InicioServidor/DS_InicializarMesas.png}{Inicializar Mesas}{0.5}
% imagen


\escenario{Inicio del Servidor}{
En este DS se puede ver como se ``enciende'' el servidor,
como se crea el el Obtenerdor de pedidos, el receptor de pedidos de archivos,  

}
\imagen{DS_InicioServidor/DS_InicioServidor.png}{Inicio Servidor}{0.5}

% 
% \subsubsection{Funcionalidades generales de los administradores}
% \escenario{Modo Dirigido Craps} { 
Puede verse la secuencia de un seteo de el modo dirigido de craps, donde se setea alg'un valor de cada dado
y que es una jugada \textit{todosPonen}.}
\tam
% \imagen{DS_Admin/ConfigurarModoDirigidoCraps.png}{Modo Dirigido Craps}{0.4}


\escenario{ Modo Dirigido setear JugadaFeliz }{
Puede verse la secuencia de un seteo de el modo dirigido para la jugada Feliz.
}
\tam


\escenario{Pedir Reporte Ranking de Jugadores}{
Un administrador pide el reporte de ranking de jugadores el cual le informa cu'ales son los jugadores m'as ganadores y perdedores en lo que va del d'ia.

Para poder recibirlo debe introducir la password correcta.
}

\imagen{DS_Admin/PedirReporte/DS_PedidoReporteRankingDeJugadores.png}{Validaci'on de la password}{0.5}

\imagen{DS_Admin/Pedir Reporte/DS_RespuestaReporteRankingDeJugadores.png}{Armado del reporte}{0.5}


\escenario{Pedir Reporte de Movimientos}{
Un administrador pide el reporte de todos los movimientos por jugador (apuestas,
premios ganados, monto ganado) desde que ingresaron al casino.

Para poder recibirlo debe introducir la password correcta.
}

\imagen{DS_Admin/PedirReporte/DS_RespuestaReporteMovimientos.png}{Armado del reporte}{0.5}


\escenario{Pedir Estado Actual}{
Un administrador pide el reporte de movimientos el cual le informa cu'ales es el saldo de los jugadores y el saldo del casino.

Para poder recibirlo debe introducir la password correcta.
}

\imagen{DS_Admin/PedirReporte/DS_RespuestaReporteEstadoActual.png}{Armado del reporte}{0.5}



% 
% \subsubsection{Funcionalidades generales del casino}
%  \escenario{Entrar Casino}{
Un usuario desea entrar al casino, puede hacerlo en modo jugador o en modo observador.
Si ya ha ingresado en modo jugador no se lo dejar'a entrar nuevamente. Si est'a en modo observador y desea ingresar en el mismo modo tampoco podr'a hacerlo.

En cambio si quiere entrar como jugador (independientemente de si ingres'o como observador o si no ingres'o) se deber'a validar que sea un usuario autorizado por marketing:

\begin{itemize}
 \item En caso afirmativo quedar'a ingresado en modo jugador.
 \item En caso negativo quedar'a en modo observador o fuera del casino seg'un cual fuese su estado anterior.
\end{itemize}
 }
\imagen{DS_Casino/EntarCasino/DS_EntrarCasinoFueraDelModelo.png}{Entrar casino fuera del Modelo}{0.6}
%aca iria una imagen------------------------------------------------------------
Entrar casino dentro del Modelo \tam

\clearpage



%aca iria una imagen------------------------------------------------------------

\clearpage

\textbf{Pedir Estado Casino}

Un usuario desea informarse sobre el estado del casino. Se le informar'a s'olo si ha ingresado en el casino, m'as haya si es en modo jugador o en modo observador.

El estado del casino est'a formado por:

\begin{itemize}
 \item La lista de jugadores y observadores ingresados en el casino;
 \item El valor del pozo feliz y del pozo progresivo;
 \item El estado de las mesas del juegos de craps:
	\begin{itemize}
	 \item Los jugadores;
	 \item El 'ultimo tirador y el pr'oximo;
	 \item Si el siguiente es tiro de salida o ya est'a el punto establecido;
	 \item El valor de los dados en el 'ultimo tiro;
	\end{itemize}
 \item El estado de las mesas del juego tragamonedas:
	\begin{itemize}
 	 \item Los jugadores;
 	 \item El valor de los rodillos en el 'ultimo tiro;
 	 \item El 'ultimo tirador y el pr'oximo;
	\end{itemize}
\end{itemize}
\tam

%aca iria una imagen------------------------------------------------------------

\clearpage

\textbf{Salir Casino}

Usuario desea salir del casino.

Si ha ingresado como observador no tendr'a ning'un tipo de validaci'on, por consiguiente tampoco problemas.

Si ha ingresado como jugador, para poder salir deber'a estar fuera de toda mesa. Es decir, no puede pretender salir del casino si es que est'a dentro de una mesa jugando.

\imagen{DS_Casino/SalirCasino/DS_SalirCasino.png}{Salir del Casino}{0.6}

%aca iria una imagen------------------------------------------------------------




% 
% \subsubsection{Tragamonedas}
% \escenario
{
\begin{itemize}
 \item Un jugador con una mesa elegida para jugar
\item  inserta una cantidad v'alida de fichas en una m'aquina tragamonedas
\item  gira los rodillos. 
\item 'Esta, luego de debitar el monto establecido con aterioridad correspondiente al pozo progresivo, consulta el tipo de jugada al casino 
\item le responde que es una jugada normal
\item la m'aquina genera el resultado de la jugada en base a las probabilidades que le establecieron. 
\item 'Esta termina arrojando un resultado ganador 
\item el jugador observa el resultado y tipo de jugada 
\item recibe el cobro.

\end{itemize}

}

\escenario
{
\begin{itemize}
  \item Un jugador con una mesa elegida para jugar inserta una cantidad v'alida de fichas en una m'aquina tragamonedas 
  \item gira los rodillos. 
  \item debita el monto establecido con aterioridad correspondiente al pozo progresivo
  \item consulta el tipo de jugada al casino
  \item le responde que es una jugada normal
  \item la m'aquina genera el resultado de la jugada en base a las probabilidades que le establecieron
  \item termina arrojando un resultado perdedor
  \item el jugador observa el resultado y tipo de jugada.
\end{itemize}
}

\escenario
{
Un jugador con una mesa elegida para jugar inserta una cantidad v'alida de fichas en una m'aquina tragamonedas y gira los rodillos. 'Esta, luego de debitar el monto establecido con aterioridad correspondiente al pozo progresivo, consulta el tipo de jugada al casino y le responde que es una jugada todos ponen. Luego la m'aquina genera el resultado de la jugada en base a las probabilidades que le establecieron. 'Esta termina arrojando un resultado ganador para el jugador qui'en luego observa el resultado y tipo de jugada, recibe el cobro y la m'aquina debita en el pozo feliz el importe correspondiente.
}

\escenario
{
Un jugador con una mesa elegida para jugar inserta una cantidad v'alida de fichas en una m'aquina tragamonedas y gira los rodillos. 'Esta, luego de debitar el monto establecido con aterioridad correspondiente al pozo progresivo, consulta el tipo de jugada al casino y le responde que es una jugada feliz. Luego la m'aquina genera el resultado de la jugada en base a las probabilidades que le establecieron. 'Esta termina arrojando un resultado ganador para el jugador. 'Este observa el resultado, el tipo de jugada y luego cobra la suma entre el pago por la apuesta ganada y el pozo feliz hasta ese momento. Luego el pozo feliz es reiniciado a su valor inicial. 
}

\escenario
{
Un jugador con una mesa elegida para jugar inserta la m'axima cantidad de fichas en una m'aquina tragamonedas luego de jugar en ella el mismo importe tantas veces como las necesarias para que pueda ganar el pozo progresivo y gira los rodillos. 'Esta, luego de debitar el monto establecido con aterioridad correspondiente al pozo progresivo, consulta el tipo de jugada al casino y le responde que es una jugada de pozo progresivo. Luego la m'aquina genera el resultado de la jugada en base a las probabilidades que le establecieron. 'Esta termina arrojando el resultado con pago m'aximo. El jugador observa el resultado y tipo de jugada. Luego cobra la suma entre el pago por la apuesta ganada y el monto total del pozo progresivo. Luego el pozo progresivo es reiniciado a su valor inicial.
}

% 
% 
% \subsubsection{Craps}
% \escenario{ Entrar Craps}{
Este escenario es bastante gen'erico. Se muestra como se valida cada cosa, como actua el sistema en cada caso
y que mensaje de error da.

El usuario puede o no estar en el casino en modo jugador.(incluye modo observador o no haber ingresado)
Puede estar en otra mesa o puede desear crearla.
}
\tam

\clearpage
\escenario{Resolverse Apuesta de Sitio a Ganar}{
La ronda esta en ``Est'an Saliendo'' sali'o un 4. La apuesta se resuelve, pasa a estar cerrada
}
% imagen
\imagenvertical{DS_Craps/CrapsResolverse_ApuestaDeSitioaGanar.png}{Resolverse Apuesta de Sitio a Ganar}{0.5}

\clearpage
\escenario{Resolverse Apuesta Venir}{
Se estableci'o el punto. Se le paga. La apuesta se cierra
}
% imagen

\imagenvertical{DS_Craps/CrapsResolverse_AuestaVenir.png}{Resolverse Apuesta Venir}{0.5}

\escenario{ \textbf{Jugador de Craps haciendo una apuesta} }{
     El usuario esta en una mesa de craps. Elige un valor de ficha de \$20
  Elige un valor de ficha de \$15. Elige el tipo de apuestatem por cada eleccion se ve un mensaje en el log. Si elige una apuesta antes de una ficha  da un error.

}
% imagen


% 
% 

\clearpage

% - Pseudocodigo de operaciones mas complicadas en lo algoritmico.
\section{Pseudoc'odigo de operaciones m'as complicadas en lo algoritmo.}
\subsubsubsection{ObtenerJugadoresMasGanadores(): Lista<Nombre>}

'Este m'etodo pertenece a la clase \italica{AdministradorDeCasino}. Devuelve la lista ordenada en forma descendente de los jugadores que m'as dinero han ganado en el casino durante el presente d'ia.

\begin{verbatim}
AdministradorDeCasino::ObtenerJugadoresMasGanadores(): Lista<Nombre>
	
	jugadas = elHistorialDeJugadas.GetJugadas()
	jugadasTragamonedas = elHistorialDeJugadas.GetJugadasTragamonedas()
	premios = jugadas.GetPremios()
	premiosTragamonedas = jugadasTragamonedas.GetPremio()

	Para cada elemento de premios hacer
		gano = monto_normal + monto_feliz - monto_todosponen - monto_apostado
		Si al jugador asociado es nuevo (no aparecio en una iteracion anterior)
			se le guarda este valor
		si no
			se le suma al valor que ya tenia
		Fin si
	Fin para

Para cada elemento de premiosTragamonedas hacer
		gano = monto_normal + monto_feliz - monto_todosponen - monto_apostado + monto_progresivo
		Si al jugador asociado es nuevo (no aparecio en una iteracion anterior)
			se le guarda este valor
		si no
			se le suma al valor que ya tenia
		Fin si
	Fin para

	De los jugadores que menor valor tienen en gano se toma a los 3 primeros y se los retorna
\end{verbatim}


\subsubsubsection{ObtenerJugadoresMasPerdedores(): Lista<Nombre>}

'Este m'etodo pertenece a la clase \italica{AdministradorDeCasino}. Devuelve la lista ordenada en forma descendente de los jugadores que m'as dinero han perdido en el casino durante el presente d'ia.

\begin{verbatim}
AdministradorDeCasino::ObtenerJugadoresMasPerdedores(): Lista<Nombre>
	
	jugadas = elHistorialDeJugadas.GetJugadas()
	jugadasTragamonedas = elHistorialDeJugadas.GetJugadasTragamonedas()
	premios = jugadas.GetPremios()
	premiosTragamonedas = jugadasTragamonedas.GetPremio()

	Para cada elemento de premios hacer
		perdio = monto_normal + monto_feliz - monto_todosponen - monto_apostado
		Si al jugador asociado es nuevo (no aparecio en una iteracion anterior)
			se le guarda este valor
		si no
			se le suma al valor que ya tenia
		Fin si
	Fin para

	Para cada elemento de premiosTragamonedas hacer
		perdio = monto_normal + monto_feliz - monto_todosponen - monto_apostado + monto_progresivo
		Si al jugador asociado es nuevo (no aparecio en una iteracion anterior)
			se le guarda este valor
		si no
			se le suma al valor que ya tenia
		Fin si
	Fin para

	De los jugadores que menor valor tienen en perdio se toma a los 3 primeros y se los retorna
\end{verbatim}


\subsubsubsection{DetalleMovimientoJugadores(): Coleccion<Tupla<Nombre, Texto, Creditos, Creditos, Creditos, Creditos>>}

'Este m'etodo pertenece a la clase \italica{AdministradorDeCasino}. Devuelve la una coleccion con los jugadores sus apuestas y los premios ganados y perdidos durante el presente d'ia.

\begin{verbatim}
AdministradorDeCasino::DetalleMovimientoJugadores(): Coleccion<Tupla<Nombre, Texto, Creditos, Creditos, Creditos, Creditos>>

	jugadas = elHistorialDeJugadas.GetJugadas()
	jugadasTragamonedas = elHistorialDeJugadas.GetJugadasTragamonedas()
	premios = jugadas.GetPremios()
	premiosTragamonedas = jugadasTragamonedas.GetPremio()

	Para cada elemento p de premios hacer
		unaTupla = Tupla(p.apostador, p.nombre_tipo_apuesta, p.monto_normal, p.monto_feliz, p.monto_todosponen, Null)
		laListaDeRetorno.Agregar(unaTupla)
	Fin para

	Para cada elemento p de premiosTragamonedas hacer
		unaTupla = Tupla(p.apostador, p.nombre_tipo_apuesta, p.monto_normal, p.monto_feliz, p.monto_todosponen, p.montoProgresivo)
		laListaDeRetorno.Agregar(unaTupla)
	Fin para

	Retornar laListaDeRetorno
\end{verbatim}



% Diagrama de clases
\section{Diagrama de clases}
\subsection{Introducci'on}
El m'odulo del casino se puede separar en dos partes: 

\begin{itemize}
\item componentes \italica{Cliente}
\item componente \italica{Servidor}
\end{itemize}  

La comunicaci'on entre el componente {\it Servidor} y los componentes {\it Cliente} ya fue resuelta por el documento de Arquitectura Conceptual y Protocolo brindado por los clientes (ver documento adjunto) y por las extensiones realizadas al mismo (ver \ref{ModificacionesAlProtocolo}).

No existir'a comunicaci'on entre los componentes {\it Cliente} desde el punto de vista arquitectural, aunque s'i se relacionar'an indirectamente a trav'es de la l'ogica de negocio del servidor.


\subsection{Servidor}
\subsection{Gr'afico Diagrama de Clases}

El enfoque en este diagrama es sobre el \italica{Servidor}.

El grafico del modelo va en formato digital

% A continuaci'on tenemos el gr'afico del diagrama de clases para la resoluci'on del problema sobre el casino.
% 

% y
% \imagenvertical{DC_DiagramaCompleto_vista_control.png}{Diagrama de Clases}{0.18}
% 
% \imagenvertical{DC_DiagramaCompleto_modelo.png}{Diagrama de Clases}{0.18}

\clearpage

\subsection{Explicaci'on}
Para facilitar el dise'no, mantener un nivel bajo de acoplamiento e incluso fomentar la reutilizaci'on de los elementos utilizados en el diagrama, se agruparon ciertas clases en m'odulos. Cada m'odulo define una 'unica responsabilidad y las clases que lo contienen deben respetarla y ejecutarla.

A continuaci'on tenemos la explicaci'on detallada de los m'odulos junto con sus clases relevantes.


\subsubsection{Comunicaci'on}
Se encarga de mantener la comunicaci'on de bajo nivel contra los clientes. 'Esto incluye la recepci'on de pedidos y el envio de las respuestas y mensajes de estado. Adem'as es el encargado de abstraer el medio de comunicaci'on (por ejemplo, por archivo que es el utilizado en el presente trabajo). Su l'ogica es simple y genera, consistiendo de rutinas de escucha de mensajes

\subsubsubsection{Clases relevantes}

\begin{description}
\item[ReceptorPedidos] Es el encargado de \italica{poolear} el medio que le fue especificado en b'usqueda de nuevos pedidos. Posee un ReceptorPedidosConcreto qui'en es el encargado de especificar el medio por el cual debe escuchar pedidos (en nuestro caso, por archivo) y es establecido en el arranque de la aplicaci'on. Una vez que se invoca a ComenzarRecepcion, el servidor ya est'a listo para comenzar a recibir pedidos de clientes. Notar que no es un singleton porque no se requiere limitar la cantidad de receptores (pueden haber uno por thread si el sistema fuese multithreading) ni requiere ser referenciado. 
\item[DespachadorRespuestas] Similar al receptor, 'este se encarga de enviar la respuesta ya generada de vuelta a un cliente. Es un singleton para que se permita su uso a quien as'i lo requiera, pero no para limitar la cantidad de instancias (en principio, podrian ser m'as si se necesitan atender despachos de varios threads). Al igual que el receptor, posee un DespachadorRespuestasEspecifico que se encarga de hacer el env'io real porque es quien realmente conoce el medio por donde se debe enviar la respuesta (en nuestro caso, por archivo).
\end{description}


\subsubsection{MensajeroDeEntrada}
Tiene como tarea la de manejar el flujo de informaci'on dentro del servidor. 'Esto incluye distribuir los pedidos recibidos desde la capa de comunicaci'on a los respectivos encargados de atenderlos y de avisarle a los encargados de generar las respuestas que las generen (adem'as de informar cu'al de ellas deben generar). En su l'ogica, principalmente sabe recorrer los XML para obtener los valores que necesitan los responsables de atender al pedido. Adem'as, sabe comprender dada una respuesta del manejo del pedido a quien le debe informar que respuesta devolver al cliente.

\subsubsubsection{Clases relevantes}

\begin{description}
\item[DespachadorPedidos] Es el responsable de despachar cada pedido recien entrado a su manejador correspondiente.
\item[\italica{Manejadores}] Cada manejador posee la responsabilidad de atender a pedidos de un conector del lado del cliente y posee un m'etodo por cada pedido que pueda llegar. Dentro de cada uno de 'estos m'etodos se encuentra la l'ogica de invocaci'on al Modelo y MensajeroDeSalida.
\end{description}


\subsubsection{MensajeroDeSalida}
Se encarga de generar las respuestas a los clientes, sabiendo de donde obtener la informaci'on necesaria para lograrlo. Adem'as es quien atiende eventos generados por la l'ogica del casino y sabe como actual ante cada uno de ellos. Tiene l'ogica para conseguir los datos necesarios para generar las respuestas, al igual que sabe como 'estas deben ser armadas.

\subsubsubsection{Clases relevantes}

\begin{description}
\item[\italica{Manejadores}] 
\end{description}


\subsubsection{Modelo}
Contiene la l'ogica de negocio junto con las estructuras asociadas a ella. Expone interfaces (llamadas \italica{fachadas}) mediantes las cuales los demas m'odulos pueden comunicarse con 'el. Su l'ogica esta compuesta por las validaciones ante cada mensaje y las consultas y modificaciones que 'estos generan. 

\subsubsubsection{Clases relevantes}

\begin{description}
\item[AdministradorJugador], \italica{AdministradorObservador} y \italica{FachadaUsuario}. 'Estas clases hacen el manejo del ingreso y egreso de jugadores y observadores. Adem'as de las consultas sobre el saldo, la existencia (entre otras) de jugadores y observadores

\item[Usuarios] contiene la lista de jugadores (normales y vip) y la lista de observadores.
Sus operaciones est'an relacionadas con el manejo de las antedichas listas y obtenci'on de datos sobre el jugador, mediante \italica{ObtenerSaldoJugador}

\item[AdministradorMesaCraps] es la responsable del ingreso y egreso de jugadores en las mesas, decir qu'e mesas est'an abiertas, ultimos resultados, entre otros. B'asicamente se ocupa de administrar todo lo referente a las mesas y los tiros del juego craps.

\item[AdministradorMesaTragamonedas] al igual que la anterior administra lo referente a las mesas y los tiros, pero en este caso es sobre el juego de Tragamonedas.

\item[Mesas] la lista de mesas de craps y la lista de mesas tragamonedas. La \italica{Clase MesaTragamoneda} y la \italica{Clase MesaCraps} son las representaciones de las mesas del juego Tragamonedas y del juego Craps respectivamente. 

\item[JugadaCraps] contiene el resultado de los dados, a traves de la \italica{Clase Dado}. Tambi'en contiene las apuestas efectudas y por qu'e importe.

\item[JugadaTragamonedas] contiene el resultado de los rodillos, a traves de la \italica{Clase ResultadoTragamonedas}. Pues esta contiene la \italica{Clase RodilloTragamonedas} que tiene como atributo el valor del rodillo.
Tambi'en se relaciona con la \italica{Clase PozoProgresivo}, de la cual se obtendr'a el monto para realizar el pago y el consiguiente reseteo del mismo.

\item[AdministradorPozos] muest]a los saldos del pozo progresivo y del pozo feliz.

\item[ReceptorPedidos] media]te su especializaci'on levantar'a los distintos pedidos de la manera que sea necesaria.

\item[DespachadorRespuestas] es an]aloga a la anterior y tiene el mismo comportamiento diferenci'andose en que es para las respuestas. Hay una por puerto.

\item[DespachadorPedidos] inter]retar'a el pedido y seg'un de qui'en provenga se lo dar'a para procesar a alguna de las Clases \italica{JuegoCraps}, \italica{AccesoYVistaCraps}, \italica{AccesoYVistaTragamonedas} o \italica{AccesoYVistaCasino}. 

\end{description}



\subsection{Cliente Jugador}
Debido a la necesidad de una interfaz gr'afica de usuario para la utilizaci'on del producto, decidimos adoptar como estrategia para el modelado una aproximaci'on a un sistema de eventos y ventanas tal como los masivamente usados en casi todas las plataformas. En este tipo de sistemas el desarrollador de aplicaciones consta con bibliotecas y APIs que le permiten interactuar con el sistema de ventanas y eventos. Hemos realizado la simplificaci'on y abstracci'on que creemos conveniente para que nuestros diagramas sean simples y al mismo tiempo lo suficientemente expresivos para no dejar de lado cuestiones que hacen al mundo de la programaci'on orientada a eventos.
Presentamos a continuaci'on la descripci'on de los m'odulos componentes.

\subsubsection{GUI}
Es el conjunto de clases que modelan la interfaz gr'afica presentada al usuario.

\subsubsubsection{Clases relevantes}

\begin{description}
\item[\italica{Ventana}] Modela la clase base a toda representaci'on subyacente de un gr'afico de ventana. Generalmente esta clase es provista por una biblioteca. Es de esta clase que, mediante herencia, podemos definir las ventanas personalizadas que utilizan nuestros clientes.
\end{description}



\subsubsection{Controladores GUI}
Consisten en el conjunto de clases que controlan los eventos desencadenados por acciones del usuario en su interacci'on con el sistema de ventanas o generados exclusivamente por el sistema de ventanas. Existe un controlador por cada ventana a excepci'on de ventanas particularmente sencillas de las cuales no resulta imprescindible manejar de una forma interesante los eventos que podr'ia generar un usuario al interaccionar con las mismas.

Nuestro diagrama puede dar una noci'on parcial referente a la apariencia visual de las ventanas gr'aficas definidas si inferimos y caracterizamos de alguna forma intuitiva los nombres de los m'etodos de las clases de ventanas y controladores. Cre'imos apropiado incluir una descripci'on aproximada pero mucho m'as precisa de c'omo lucir'an nuestras interfases gr'aficas en el producto final. Adem'as, es en estas interfases en las que pensamos al momento de dise'nar el cliente, con lo cual hay una correspondencia 1 a 1 (salvando algunos casos) con las clases presentadas en el diagrama.

% \subsubsubsection{LogIn}
\imagen{PrototiposPantalla/PlayerClient_SignIn.png}{Prototipo de pantalla de ingreso al casino. Clase \italica{VentanaLogin}}{0.5}
% 
% 
% % \subsubsubsection{Lobby}
\imagen{PrototiposPantalla/PlayerClient_Lobby.png}{Prototipo de pantalla del lobby del casino. Clase \italica{VentanaLobby}}{0.5}
% 
% 
% % \subsubsubsection{Selecci'on de ficha}
\imagen{PrototiposPantalla/PlayerClient_SelectCoinValue.png}{Prototipo de pantalla de selecci'on de ficha. Clase \italica{VentanaSeleccionarValorFicha}}{0.5}
% 
% 
% % \subsubsubsection{Selecci'on de mesa}
\imagen{PrototiposPantalla/PlayerClient_SelectTable.png}{Prototipo de pantalla de selecci'on de mesa. Clase \italica{VentanaSeleccionarMesa}}{0.5}
% 
% % \subsubsubsection{Tragamonedas}
\imagen{PrototiposPantalla/PlayerClient_Tragamonedas.png}{Prototipo de pantalla del juego Tragamonedas. Clase \italica{VentanaTragamonedas}}{0.5}
% \clearpage
% 
% % \subsubsubsection{Craps}
\imagenvertical{PrototiposPantalla/PlayerClient_Craps.png}{Prototipo de pantalla del juego Craps. Clase \italica{VentanaCraps}}{0.65}
% \clearpage

\subsubsection{Comunicaci'on}
Esquema semejante al que utilizamos para el dise'no del servidor aunque con algunos cambios que vale la pena mencionar.


\subsection{Cliente Administrador}
\todo{Hacer esto}
\clearpage

\subsubsubsection{Panel de administraci'on}
\imagen{PrototiposPantalla/PlayerAdmin_AdminPanel.png}{Prototipo de pantalla del panel de administradores}{1}
\clearpage

\subsubsubsection{Verificaci'on de identidad}
\imagen{PrototiposPantalla/PlayerAdmin_VerifyPassword.png}{Prototipo de pantalla de verificaci'on de password de administrador}{1}
\clearpage

\subsubsubsection{Modo Dirigido :: Tragamonedas}
\imagen{PrototiposPantalla/PlayerAdmin_Tragamonedas.png}{Prototipo de pantalla de configuraci'on del modo dirigido para el juego Tragamonedas}{1}
\clearpage

\subsubsubsection{Modo Dirigido :: Craps}
\imagen{PrototiposPantalla/PlayerAdmin_Craps.png}{Prototipo de pantalla de configuraci'on del modo dirigido para el juego Craps}{1}
\clearpage

\subsubsubsection{Modo Dirigido :: Jugada Feliz}
\imagen{PrototiposPantalla/PlayerAdmin_HappyMove.png}{Prototipo de pantalla de configuraci'on de la jugada feliz en modo dirigido}{1}
\clearpage


\clearpage

% - Justificacion, analisis y explicacion de su diseño, utilizando desde principios de diseño, hasta patterns, pasando por depedencias, acoplamiento y cohesion, hasta mas secuencias de ejemplo para explicar porque su diseño es bueno y elegante al resolver los problemas que se les presentaron.
\section{Justificaci'on, an'alisis y explicaci'on del dise'no}
Se model'o basandonos en el patr'on de diseño MVC. Con lo cual tenemos tres grandes paquetes:

\begin{itemize}
 \item Modelo
 \item Controlador
 \item Vista
\end{itemize}

La recepci'on de los pedidos se hace con la Clase ReceptorPedidos, que est'a dentro del paquete Vista. Esta clase ser'a especializada de la manera que sea necesaria. Esto brinda una mayor flexibilidad a la hora de tomar los pedidos y nos liga menos al tipo en el que llega el pedido.

Al levantarse los datos son guardados en una clase Pedido la cual tiene un diccionario en el cual las las claves son del tipo texto y el resultado es un objeto. Esto es bueno, ya que permite flexiblidad ante los cambios de par'ametros de entrada. Tanto de tipo como de cantidad.

La Clase DespachadorPedidos (paquete controlador) toma los par'ametros de entrada y multiplexa seg'un corresponda. Esta clase es singleton debido a que solo es necesaria una sola instancia para no tener problemas de concurrencia.

Las clases JuegoCraps, AccesoYVistaCraps, AccesoYVistaTragamonedas y AccesoYVistaCasino procesan los datos seg'un corresponda. Los datos son obtenidos de los administradores que est'an en el paquete Modelo. Esto crea acoplamiento sobre dichas clases, pero no las crea sobre el resto del paquete modelo. Si bien el acoplamiento est'a, 'este es entre paquetes que es menor que entre clases de distintos paquetes. 'Estas clases son singleton.

Tenemos las Clases Singleton para la emisi'on de la respuesta: JuegoCraps, AccesoYVistaCraps, AccesoYVistaTragamonedas y AccesoYVistaCasino.

Estas clases se corresponden uno a uno con las del controlador, brindando las respuestas a los pedidos. En el caso de necesitar datos, que no fueron brindados por el controlador, tambi'en le hacen el pedido a las clases administrador del paquete Modelo.

Como son para emitir la respuesta ent'an dentro del paquete Vista.

Para emitir la respuesta utilizar'an un tipo llamado Respuesta. Esto permite desacoplar del tipo de respuesta y dejar a la clase especializada del Despachador de respuestas, la responsabilidad del formato de respuesta. Como es especializaci'on en caso en que se modifique el modo en que se envia la respuesta se hereda una clase nueva. Favoreciendo el open-close.


\clearpage

\section{\label{ModificacionesAlProtocolo}Modificaciones al protocolo}

% % \input{../doc/Extensiones\ al\ protocolo.txt}
\clearpage


% - Trazabilidad completa con informe 1. Ademas todo el informe va dirigido al area de sistemas de la empresa de los socios.
\section{Trazabilidad}
\input{trazabilidad.tex}
\clearpage

% - Conclusiones.
\section{Conclusiones}
\subsection{Comentarios al corrector}
\subsubsection{La ficci'on supera a la realidad}
Docente de IS1, en la seccion de requerimientos del sistema nos comprometimos, entre otros, a que nuestro sistema sea escalable y seguro. Lo hicimos porque son requerimientos importantes y como tales deberemos cumplir para que los stakeholders est'en satisfechos con la soluci'on que les ofreceremos. Pero en el contexto del TP, no obtuvimos ninguna m'etrica acerca de que tan escalable y qu'e tan seguro deber'ia ser (a'un as'i, no son variables que deber'ian obviarse por las caracter'isticas del producto). Es por eso que en realidad, en el TP, no nos comprometemos a cumplir estos requerimientos. 

De forma similar ocurre con el requerimiento de ser r'apido. En realidad nos comprometemos a qu'e el sistema sea ``utilizable'' pero no ofrecemos ninguna garant'ia real, principalmente por que no se nos ofrecieron m'etricas a cumplir.

El dise'no no ser'a inclu'ido no porque tengamos dise'nadores en huelga si no porque ning'un integrante de nuestro grupo se destaca por poseer cualidades art'isticas...



\clearpage


\end{document}
