% Archivo de configuracion del informe
% -------------------------------------------

\usepackage[spanish,activeacute]{babel}								% Idioma castellano
\usepackage{caratula}														% Caratula de Algo2
%\usepackage[a4paper=true,pagebackref=true]{hyperref}				% Agrega la TOC al PDF e hipervinculos
\usepackage{graphicx} 														% Permite insertar graficos
\usepackage{fancyhdr}														% Permite manejo de cabeceras de pagina
\usepackage{eufrak}															% Usado en el enunciado del trabajo
\usepackage{latexsym}
%\usepackage{algorithmic}													% Para escribir los algos
%\usepackage{dsfont}															% Para el simbolo de naturales
\usepackage[font=small,labelfont=bf]{caption}						% Para editar las captions
%\usepackage{array}

\usepackage[utf8]{inputenc}
\usepackage{amsmath}
\usepackage[x11names, rgb]{xcolor}
% \usepackage{tikz}
% 	\usetikzlibrary{snakes,arrows,shapes}
\usepackage{listings}
\usepackage{lastpage}
\usepackage{geometry}
  \geometry{left=1cm, right=1cm, top=2cm, bottom=2cm}

% Estilo de pagina para tener las cabeceras y pieseras
\pagestyle{fancy}
  \fancyhead[LO]{Ingenier\'ia de Software I}
  \fancyhead[C]{Trabajo Pr\'actico - Dise\~no}
  \fancyhead[RO]{Primer Cuatrimestre 2008}
  \renewcommand{\headrulewidth}{0.4pt}

  \fancyfoot[LO]{Echevarr\'ia - Farjat - Freijo - Giusto}
  \fancyfoot[C]{}
  \fancyfoot[RO]{P\'agina \thepage\ de \pageref{LastPage}}
  \renewcommand{\footrulewidth}{0.4pt}

\parindent = 1.5 em 
\parskip = 8 pt


%%%%%%%%%%%%%%%% COMANDOS %%%%%%%%%%%%%%%%%%%%
\newcommand{\todo}{{\large\textbf{TODO: }}}
\newcommand{\paso}{\textsc{Paso }}
\newcommand{\func}[1]{\verb"#1"}
\newcommand{\imagen}[3]
{
	\begin{figure}[p!hbt]
	  \centering
	  \includegraphics[scale=#3]{../img/#1}
	  \caption{#2}
          \label{fig_#1}
	\end{figure}
}
\newcommand{\imagenvertical}[3]
{
	\begin{figure}[p!hbt]
	  \centering
	  \includegraphics[angle=90,scale=#3]{../img/#1}
	  \caption{#2}
          \label{fig_#1}
	\end{figure}
}




\newcommand{\nat}{\mathds{N}}
\newcommand{\algoritmo}[3]{\noindent {\bf\underline{#1}:} #2 $\longrightarrow$ #3}
\newcommand{\superindice}[1]{$^\textrm{{\tiny #1}}$}
\newcommand{\subsubsubsection}[1]{\noindent\negrita{#1}

}
\newcommand{\negrita}[1]{{\bf #1}}
\newcommand{\italica}[1]{{\it #1}}

% Comandos y seteos iniciales para los escenarios
% Uso para crear un escenario
% 		\escenario{descrpcion}
\newcounter{contadorEscenarios}
\setcounter{contadorEscenarios}{1}
\newcommand{\escenario}[1]
{\negrita{Escenario \arabic{contadorEscenarios}}
#1
\stepcounter{contadorEscenarios}
}

% -----------------------------------------------------------------------------------------------------
% Codigo para generar los casos de uso, basado en caratula.sty del DC-FCEN-UBA 
% Autor: Nicolas Rosner
% Modificado por Pablo Echevarria 25-5-08
% ToDo: ver como declarar primero titulo y despues operaciones, ver como numerar automaticamente las operaciones
% -----------------------------------------------------------------------------------------------------
% Token list para las instrucciones ----
\newtoks\oplist\oplist={}

% Comando para que el usuario agregue operaciones del CU
% Uso: \op{Caso normal}{Caso alternativo}
\newcommand{\op}[2]{\oplist=\expandafter{\the\oplist
\hline#1&#2\\ }}

% Comando para generar el CU con las operaciones ya pasadas y dandole la info que falta
% Uso: \cu{Nombre CU}{Actor primario}{Actores secundarios}{Precondicion}{Postcondicion}{Aclaraciones generales a todos los pasos}
\newcommand{\cu}[6]{
\begin{center}	
\begin{tabular}{|p{11cm}|p{5cm}|}
\hline
\multicolumn{2}{|l|}{\begin{large}\italica{\negrita{CASO DE USO:} #1}\end{large}}\\[0.2em]
\hline
\multicolumn{2}{|l|}{\negrita{Actor Primario:} #2}\\[0.2em]
\hline
\multicolumn{2}{|l|}{\negrita{Actor Secundario:} #3 }\\[0.2em]
\hline
\multicolumn{2}{|l|}{\negrita{Precondici'on:} #4 }\\[0.2em]
\hline
\multicolumn{2}{|l|}{\negrita{Postcondici'on:} #5}\\[0.2em]
\hline
\multicolumn{2}{|l|}{}\\[0.2em]
\hline
\negrita{Curso Normal} & \negrita{Curso Alternativo}\\[0.2em]
\the\oplist
\hline
\multicolumn{2}{|l|}{}\\[0.2em]
\hline
%\multicolumn{2}{|1|}{#6}\\[0.2em]
%\hline
\end{tabular}
\end{center}
\oplist={}
}




%%%%%%%%%%%%%%%% FIN COMANDOS %%%%%%%%%%%%%%%%%%%%
