Luego de haber realizado el proceso de relevamiento de requerimientos y especificaci'on del problema en el TP1, en el presente trabajo planteamos una soluci'on al dise'no de los componentes de software solicitados por nuestros clientes, los Timbalistas. 

\noindent La soluci'on aqu'i expuesta consiste en:
\begin{quote}
\begin{description}
\item[Diagramas de clase] Uno para el componente Servidor y dos para el Cliente, siendo 'estos los correspondientes al cliente a utilizar por los jugadores y observadores del casino y otro a utilizar por el 'area de administraci'on para realizar las consultas y modificaciones ya establecidas anteriormente. El sentido de ellos es mostrar en detalle el dise'no de cada componente.
\item[Diagramas de secuencia] Se explicar'a como ciertas funcionalidades importantes y complejas son desarrolladas por cada entidad del sistema, desde que el usuario del sistema genera una 'orden hasta que obtiene una actualizaci'on en su interfaz gr'afica con los cambios ordenados por el servidor. Para las dem'as funcionalidades se explicar'an los puntos cr'iticos y que son diferentes a cualquier otra (los pasos obviados ser'an muy similares a otros ya resueltos). Con 'estos queremos mostrar no solo el uso del diagrama de clases si no adem'as la validez del dise'no para con los requerimientos de los Timbalistas.
\end{description}
\end{quote} 
Adem'as, en la secci'on de Comentarios mostraremos todas las alternativas que se nos plantearon como v'alidas y el an'alisis correspondiente a las ventajas y desventajas entre ellas (al igual de la justificaci'on por la soluci'on finalmente elegida).

Y para el corrector, puede observar en la secci'on de Conclusiones las principales experiencias vividas y sobrevividas por el grupo en la realizaci'on del trabajo.
