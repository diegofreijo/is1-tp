\escenario{ \textbf{Jugador de Craps haciendo una apuesta}
  \begin{itemize}
    \item El usuario esta en una mesa de craps 
  \item elige un valor de ficha de \$20
  \item elige un valor de ficha de \$15
  \item elige el tipo de apuesta
  \item por cada eleccion se ve un mensaje en el log
  \item si elige una apuesta antes de una ficha  da un error
  \end{itemize}
}


\subsubsection{TirarDados}

Este escenario es bastante gen'erico,

\escenario{
      \begin{itemize}
      \item El usuario puede o no estar en la mesa.
      \item El usuario puede o no ser el tirador.
      \item En caso de que sea el tirador y esté en la mesa.
      \item Se tiran los dados.
      \item Si sali'o alg'un valor propio de punto este se setea.
      \item Hay alguna cantidad de apuestas de alg'un tipo que se resuelven o no con su l'ogica particular, dependiendo del contexto $^1$
      \item Estamos en un ``est'an saliendo'' $^3$
      \item Es una jugada feliz $^2$
    \end{itemize}
}


$^1$ Resoluci'on de distintas apuestas se ve en otros DS's (en sitio a perder y a venir)
$^2$ En otro DS se ve la situacion con el caso de que el punto esté establecido
$^3$ En otro DS se ve la jugada Todos ponen

El pozo feliz se reparte si o si el servidor de Jugadas no devuelve una feliz mientras el pozo no lleque al m'nimo

La notificaci'on a de que una mesa cambió se pude hacer generico sin complicar demasiado el diagrama.
