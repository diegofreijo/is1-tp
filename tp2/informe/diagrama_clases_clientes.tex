Los diagrama de clases se encuentran aparte.

Debido a la necesidad de una interfaz gr'afica de usuario para la utilizaci'on del producto, decidimos adoptar como estrategia para el modelado una aproximaci'on a un sistema de eventos y ventanas tal como los masivamente usados en casi todas las plataformas (celulares, pcs, agendas electr'onicas, etc). En este tipo de sistemas el desarrollador de aplicaciones consta con bibliotecas y APIs que le permiten interactuar con el sistema de ventanas y eventos. Hemos realizado la simplificaci'on y abstracci'on que creemos conveniente para que nuestros diagramas sean simples y al mismo tiempo lo suficientemente expresivos para no dejar de lado cuestiones que hacen al mundo de la programaci'on orientada a eventos.
Presentamos a continuaci'on la descripci'on de los m'odulos componentes.
\clearpage

\subsubsection{GUI}
Es el conjunto de clases que modelan la interfaz gr'afica presentada al usuario.

\subsubsubsection{Clases relevantes}

\begin{description}
\item[\italica{Ventana}] Modela la clase base a toda representaci'on subyacente de un gr'afico de ventana. Generalmente esta clase es provista por una biblioteca. Es de esta clase que, mediante herencia, podemos definir las ventanas personalizadas que utilizan nuestros clientes. Se destaca principalmente el m'etodo \italica{MostrarModal} el cual muestra el objeto visual ventana de forma modal (el usuario s'olo podr'a interactuar con la ventana modal y no con el resto hasta que esta se cierre) y espera a que el usuario la cierre presionando un bot'on, devolviendo un identificador que permite conocer qu'e bot'on fue presionado al cerrar la ventana y de esa forma tomar decisiones. El m'etodo \italica{MostrarMensajeDeTexto} modela el conocido MessageBox modal que permite mostrar un texto al usuario en una ventana popup.
\end{description}

Las clases de ventanas concretas proveen m'etodos para acceder y modificar los distintos controles de las ventanas, como ser recuadros de texto, listas de opciones, radiobuttons, etc. Hemos tratado de ser gen'ericos y usar controles razonables.

\subsubsection{Controladores GUI}
Consisten en el conjunto de clases que controlan los eventos desencadenados por acciones del usuario en su interacci'on con el sistema de ventanas o generados exclusivamente por el sistema de ventanas.

Existe una 'unica ventana activa en todo momento, cuyo controlador ser'a quien maneje los pedidos del usuario hasta que se transfiera el control.

Son los responsables de actualizar la interfaz de usuario acorde a las respuestas que da el servidor. Existe un controlador por cada ventana a excepci'on de ventanas particularmente sencillas de las cuales no resulta imprescindible manejar de una forma interesante los eventos que podr'ia generar un usuario al interaccionar con las mismas.

Los controladores se encargan de contruir la/s ventanas correspondientes en su constructor e inicializan las ventanas solicitando la informaci'on pertinente al servidor.

Hemos simplificado nuestro dise'no asumiendo que los eventos desencadenados por acciones sobre la interfaz de usuario eran 'unicamente clicks de botones. Esto nos permitir'a elegir dentro de un amplio rango de bibliotecas gr'aficas, ya que casi todas proveen un control del estilo bot'on pulsable y permiten asociar el evento del pulsado con una funci'on o m'etodo. Aqu'i realizamos una simplificaci'on referida a la forma de asociar las interacciones del usuario sobre el entorno gr'afico con los m'etodos del objeto de la clase que se encarga de manejar dichos eventos: ser'a el sistema de ventanas qui'en de alguna manera que no modelamos asociar'a los m'etodos de un controlador con los eventos correspondientes en la creaci'on de una ventana.

Nuestro diagrama puede dar una noci'on parcial referente a la apariencia visual de las ventanas gr'aficas definidas si inferimos y caracterizamos de alguna forma intuitiva los nombres de los m'etodos de las clases de ventanas y controladores. Cre'imos apropiado incluir una descripci'on aproximada pero mucho m'as precisa de c'omo lucir'an nuestras interfases gr'aficas en el producto final. Adem'as, es en estas interfases en las que pensamos al momento de dise'nar el cliente, con lo cual hay una correspondencia 1 a 1 (salvando algunos casos) con las clases presentadas en el diagrama.

\subsubsection{Comunicaci'on}
Esquema semejante al que utilizamos para el dise'no del servidor aunque con algunos cambios que vale la pena mencionar.

El env'io y recepci'on de respuesta de los mensajes que los clientes env'ian a trav'es de los puertos de comunicaci'on se dise'n'o para que se realice de forma sincr'onica.
Esto es que cuando un usuario desencadena una acci'on que requiere respuesta del servidor el cliente se bloquear'a hasta recibir respuesta. Es por eso que los m'etodos de las clases de los puertos devuelven un Xml consistente de la respuesta que el servidor proporciona.

\subsubsection{Cliente Jugador}
\subsubsubsection{Clases relevantes}
\begin{description}
\item[\italica{Sesion}] Contiene la informaci'on de la sesi'on del usuario logueado. Tiene la informaci'on del saldo del usuario, lo cual es lo 'unico que debe mantener el cliente localmente como informaci'on, ya que todo lo dem'as es notificado por el servidor en sus notificaciones. El nombre de usuario y el modo son necesarios para la construcci'on de los mensajes de pedidos y para la configuraci'on de ciertas ventanas.
\end{description}

\subsubsubsection{Controladores GUI}
Hay en determinados casos que vimos la necesidad de pollear para saber si hay mensajes de notificaci'on del servidor. Es el caso de la notificaci'on de actualizaci'on de estado del juego Craps. Para resolver esta cuesti'on decidimos agregar un m'etodo \italica{OnIdle} que ser'a llamado peri'odicamente por el sistema de eventos y ventanas el cual nos permitir'a chequear regularmente si hay una notificaci'on de estado del juego. El flujo de control de atenci'on de este m'etodo termina utilizando el receptor de mensajes as'incrono en la capa de comunicaci'on.

De la misma forma fue necesario agregar un m'etodo \italica{OnActualizarInfoCasino} que permite obtener la informaci'on referida a los pozos especiales que deben mostrarse al usuario en todo momento. El protocolo no notifica acerca de cambios a estos pozos, con lo cual la 'unica forma que encontramos para mantener las interfases de usuario actualizadas fue realizar un pedido del estado del casino a intervalos regulares razonables, como ser por ejemplo un par de segundos, de manera de no saturar al servidor ya que 'este debe dar respuesta a cada uno de estos pedidos en cada uno de los clientes que los soliciten. Adem'as, estos pedidos son bloqueantes, con lo cual tampoco queremos que la interfaz de usuario se bloquee demasiado seguido en caso de tener que esperar un tiempo considerable la respuesta del servidor si 'este est'a ocupado.

\begin{description}
\item[\italica{ControladorVentanaCraps}] Referido al protocolo: El mensaje de notificaci'on de estado para el juego Craps definido por el protocolo no informa del tipo de jugada (feliz, normal, todosponen) del 'ultimo tiro, con lo cual no pudimos reflejar esto en las interfases de usuario, aunque mostramos los pagos correspondientes a cada tipo de jugada, informaci'on que s'i proporciona el protocolo. Tampoco da una descripci'on de por qu'e fall'o una apuesta de craps al enviar un mensaje de apuestaCraps, por lo cual mostramos un msgbox con alguna informaci'on de error.
Asumimos que el tag ``ultimoTiro'' solo ser'a completado por el servidor cuando la actualizaci'on corresponda a efectos de una nueva tirada de dados. De no hacerlo as'a no habr'ia manera de saber cu'ando actualizar los saldos de los jugadores en los clientes.
\end{description}

\subsubsubsection{Comunicaci'on}
\subsubsubsection{Clases relevantes}
\begin{description}
\item[\italica{IReceptorMensajes}] Permite la recepci'on en forma sincr'onica o as'incrona de mensajes del servidor. Posee dos m'etodos a tales fines. Ambos toman como par'ametro el tipo del mensaje a recibir y el id de la terminarl virtual del receptor. El m'etodo as'incrono se utiliza exclusivamente para la recepci'on del mensaje de notificaci'on de actualizaci'on del estado de un juego Craps, devolviendo el control inmediatamente al llamador e informando en el valor de retorno si hab'ia un mensaje del tipo solicitado disponible. De existir un mensaje al momento de realizar la llamada el m'etodo devuelve el mensaje en cuesti'on en el par'ametro pasado por referencia en forma de Xml.
\end{description}

\subsubsubsection{Prototipos de pantalla}

\subsubsubsection{LogIn}
Permite al usuario elegir el modo en que desea ingresar al casino y su nombre. En el prototipo presentado hemos inclu'ido un password feliz para usuarios jugadores que har'ia que los usuarios del sistema sientan mayor seguridad, aunque este aspecto no fue modelado (no es un requerimiento).
\imagen{PrototiposPantalla/PlayerClient_SignIn.png}{Prototipo de pantalla de ingreso al casino. Clase \italica{VentanaLogin}}{1}
\clearpage

\subsubsubsection{Lobby}
El Lobby del casino permite seleccionar el tipo de juego al que se desea jugar (u observar).
\imagen{PrototiposPantalla/PlayerClient_Lobby.png}{Prototipo de pantalla del lobby del casino. Clase \italica{VentanaLobby}}{1}
\clearpage

\subsubsubsection{Selecci'on de ficha}
Esta ventana se usa exclusivamente para elegir el valor de la ficha cuando un usuario jugador quiere abrir una mesa para el juego Tragamonedas. Mostrar'a la lista de valores de ficha existentes en la jornada actual del casino.
\imagen{PrototiposPantalla/PlayerClient_SelectCoinValue.png}{Prototipo de pantalla de selecci'on de ficha. Clase \italica{VentanaSeleccionarValorFicha}}{1}
\clearpage

\subsubsubsection{Selecci'on de mesa}
Muestra la lista de ids de mesas a las que el usuario puede ingresar. Incluye una opci'on para crear una nueva mesa en los casos que corresponda, ya que esta ventana es utilizada tanto para la selecci'on de mesa por parte de jugadores como de observadores.
\imagen{PrototiposPantalla/PlayerClient_SelectTable.png}{Prototipo de pantalla de selecci'on de mesa. Clase \italica{VentanaSeleccionarMesa}}{1}
\clearpage

\subsubsubsection{Tragamonedas}
Corresponde al juego Tragamonedas y muestra toda la informaci'on relevante referido al mismo. Permite seleccionar la cantidad de fichas a apostar y girar los rodillos.
\imagen{PrototiposPantalla/PlayerClient_Tragamonedas.png}{Prototipo de pantalla del juego Tragamonedas. Clase \italica{VentanaTragamonedas}}{0.8}
\clearpage

\subsubsubsection{Craps}
El juego Craps. La ventana muestra todos los datos que se solicit'o. Utiliza recuadros de texto para mostrar las apuestas por jugador para cada tipo de jugada.
\imagenvertical{PrototiposPantalla/PlayerClient_Craps.png}{Prototipo de pantalla del juego Craps. Clase \italica{VentanaCraps}}{0.65}
\clearpage


\subsubsection{Cliente Administrador}

El Diagrama de Clases se encuentra a parte, dado el tama~ño.

% \imagenvertical{DC_ClienteAdministrador/DC_ClienteAdministrador.png}{Diagrama de clases del cliente}{0.3}
% 

% \todo{Hacer esto}
% % \clearpage
% 
% % \subsubsubsection{Panel de administraci'on}
\imagen{PrototiposPantalla/PlayerAdmin_AdminPanel.png}{Prototipo de pantalla del panel de administradores}{0.5}
% % \clearpage
% 
% % \subsubsubsection{Verificaci'on de identidad}
\imagen{PrototiposPantalla/PlayerAdmin_VerifyPassword.png}{Prototipo de pantalla de verificaci'on de password de administrador}{0.5}
% % \clearpage
% 
% % \subsubsubsection{Modo Dirigido :: Tragamonedas}
\imagen{PrototiposPantalla/PlayerAdmin_Tragamonedas.png}{Prototipo de pantalla de configuraci'on del modo dirigido para el juego Tragamonedas}{0.5}
% % \clearpage
% 
% % \subsubsubsection{Modo Dirigido :: Craps}
\imagen{PrototiposPantalla/PlayerAdmin_Craps.png}{Prototipo de pantalla de configuraci'on del modo dirigido para el juego Craps}{0.5}
% % \clearpage
% 
% % \subsubsubsection{Modo Dirigido :: Jugada Feliz}
\imagen{PrototiposPantalla/PlayerAdmin_HappyMove.png}{Prototipo de pantalla de configuraci'on de la jugada feliz en modo dirigido}{0.5}
% % \clearpage

