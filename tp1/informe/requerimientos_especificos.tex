Los siguientes son los diferentes diagramas que describen el comportamiento del sistema.

% Diagrama de contexto
\subsection{Diagrama de contexto \label{DC}}
En la Fig.\ref{fig_DC_Diagrama.png} se pueden ver los principales fen'omenos encontrados
fuera y para con el sistema.
\imagen{DC_Diagrama.png}{Diagrama de contexto}{0.6}
\clearpage


% Modelo de objetivos
\subsection{Modelo de objetivos \label{MO}}
\imagenvertical{DO_Diagrama.png}{Diagrama de Objetivos}{0.21}
\clearpage


% Requerimientos
\subsection{Requerimientos\label{REQ}}
% reqs referencing specific custom commands definition
\newcounter{reqsCounter}
\setcounter{reqsCounter}{1}
\newcommand{\req}[1]{\newcounter{#1}\setcounter{#1}{\arabic{reqsCounter}}\stepcounter{reqsCounter}\negrita{REQ\#\arabic{#1}}}
\newcommand{\rref}[1]{\negrita{REQ\#\arabic{#1}}}

A continuaci'on se detallan los requerimientos del sistema a construir. Se ha identificado cada requerimiento con un c'odigo 'unico que permite referenciar a dichos requerimientos desde el resto del documento.

\subsubsection{Requerimientos funcionales}


% requerimientos esenciales
\subsubsubsection{Esenciales}

\begin{itemize}

\item El sistema permite abrir el casino

\negrita{C'odigo de identificaci'on}: \req{req:abrir_casino}

\negrita{Descripci'on}: El casino podr'a ser abierto cuando se lo desee por un administrador del sistema.

\item El sistema permite cerrar el casino cuando no hay jugadores en el mismo

\negrita{C'odigo de identificaci'on}: \req{req:cerrar_casino}

\negrita{Descripci'on}: El casino podr'a ser cerrado cuando se lo desee por un administrador del sistema, con la condici'on de que no existan jugadores dentro al momento de hacerlo.

\item El sistema consta con juegos de tipo Tragamonedas

\negrita{C'odigo de identificaci'on}: \req{req:existe_juego_tragamoneda}

\negrita{Descripci'on}: Un casino sin juegos no es un casino. Para poder conseguir que los jugadores juegen hay, entre otros, juegos de tipo Tragamonedas.

\item El sistema consta con juegos de tipo Craps

\negrita{C'odigo de identificaci'on}: \req{req:existe_juego_craps}

\negrita{Descripci'on}: Existir'an juegos de tipo Craps.

\item El sistema permite a los jugadores ingresar al casino

\negrita{C'odigo de identificaci'on}: \req{req:ingreso_al_casino}

\negrita{Descripci'on}: Los jugadores deben poder ingresar al casino utilizando para ello los datos de su cuenta.

\item El sistema permite a los jugadores crear una nueva mesa de juego

\negrita{C'odigo de identificaci'on}: \req{req:creacion_de_mesa}

\negrita{Descripci'on}: Los jugadores que no est'en en una mesa de juego deben poder crear una nueva mesa de juego del tipo que elijan.

\item El sistema permite a los jugadores ingresar a una mesa de juego existente

\negrita{C'odigo de identificaci'on}: \req{req:ingreso_a_mesa}

\negrita{Descripci'on}: Los jugadores que no est'en en una mesa de juego deben poder ingresar a una mesa existente cuando lo deseen.

\item El sistema permite a los jugadores abandonar una mesa de juego

\negrita{C'odigo de identificaci'on}: \req{req:salir_de_mesa}

\negrita{Descripci'on}: Los jugadores que est'en en una mesa de juego deben poder salir de la misma.

\item El sistema se encargar'a de cerrar las mesas que no tengan jugadores

\negrita{C'odigo de identificaci'on}: \req{req:auto_cerrar_mesas}

\negrita{Descripci'on}: Cuando una mesa de juego haya sido abandonada por el 'ultimo jugador que la utilizaba, la misma ser'a cerrada de forma autom'atica por el sistema.

\item El sistema permite a los jugadores apostar a los juegos si poseen suficiente cr'edito

\negrita{C'odigo de identificaci'on}: \req{req:apuesta_a_juegos}

\negrita{Descripci'on}: Los jugadores que se encuentren en una mesa de juego y posean el saldo necesario deben poder apostar.

\item El sistema permite siempre a los jugadores vip apostar a los juegos

\negrita{C'odigo de identificaci'on}: \req{req:apuesta_vip_a_juegos}

\negrita{Descripci'on}: Los jugadores vip que se encuentren en una mesa de juego deben poder apostar siempre, incluso si su saldo es insuficiente para realizar una apuesta (los jugadores vip pueden tener saldo negativo).

\end{itemize}


% requerimientos importantes
\subsubsubsection{Importantes}

\begin{itemize}

%% \item El sistema permite la configuraci�n de los juegos
%% \negrita{C'odigo de identificaci'on}:\req{}
%% \negrita{Descripci'on}: Poder ajustar las variables de probabilidad y ganancias de cada juego al igual que las probabilidades de ocurrencia de jugadas felices y todos ponen. De 'esta forma se puede asegurar la rentabilidad del casino.

\item El sistema permite entrar y salir del ``modo dirigido''

\negrita{C'odigo de identificaci'on}: \req{req:existe_modo_dirigido}

\negrita{Descripci'on}: Un administrador de sistema puede activar y desactivar el ``modo dirigido'', lo que le permite controlar el azar y asegurar las ganancias del casino.

\item El sistema permite controlar los resultados de los juegos

\negrita{C'odigo de identificaci'on}: \req{req:controlar_resultados}

\negrita{Descripci'on}: Un administrador del sistema en ``modo dirigido'' puede controlar los resultados de los juegos.

\item El sistema permite controlar el tipo de jugada de los juegos

\negrita{C'odigo de identificaci'on}: \req{req:controlar_tipo_jugada}

\negrita{Descripci'on}: Un administrador del sistema en ``modo dirigido'' puede controlar el tipo de jugada (``normal'', ``feliz'' 'o ``todos ponen'') de los juegos.

\item El sistema permite la configuraci'on de las probabilidades de los resultados de los juegos

\negrita{C'odigo de identificaci'on}: \req{req:conf_probabilidades_resultados}

\negrita{Descripci'on}: Un administrador de sistema puede configurar la probabilidad de ocurrencia de los distintos resultados de los juegos.

\item El sistema permite la configuraci'on de las probabilidades de ocurrencia de las jugadas ``feliz'' y ``todos ponen''

\negrita{C'odigo de identificaci'on}: \req{req:conf_probabilidades_jugadas_especiales}

\negrita{Descripci'on}: Un administrador de sistema puede configurar la probabilidad de ocurrencia de las jugadas especiales.

\item El sistema permite la configuraci'on de la probabilidad de ganar el ``premio gordo progresivo'' de las m'aquinas tragamonedas

\negrita{C'odigo de identificaci'on}: \req{req:conf_probabilidades_premio_progresivo}

\negrita{Descripci'on}: Un administrador de sistema puede configurar la probabilidad de ganar el ``premio gordo progresivo'' de un jugador que satisface los requisitos necesarios para intentar ganarlo.

\item El sistema permite la configuraci'on del pozo progresivo de las Tragamonedas

\negrita{C'odigo de identificaci'on}: \req{req:ajuste_de_pozo_progresivo}

\negrita{Descripci'on}: Un administrador de sistema puede ajustar el valor m'inimo posible para el pozo com'un progresivo de los juegos de m'aquinas tragamonedas.

\item El sistema permite generar reportes con informaci'on del estado del casino y sus jugadores

\negrita{C'odigo de identificaci'on}: \req{req:generar_reportes}

\negrita{Descripci'on}: Un administrador de sistema puede pedir reportes del estado del casino y sus jugadores para mantener control de la rentabilidad y ajustar correctamente los juegos y pozos.

\item El sistema permite acceder al casino a m'as de un jugador a la vez desde la misma terminal

\negrita{C'odigo de identificaci'on}: \req{req:acceso_multiple}

\negrita{Descripci'on}: Dos o m'as jugadores deben poder acceder al casino desde una misma terminal.

\item El sistema permite ser extendido mediante la incorporaci'on de nuevos juegos

\negrita{C'odigo de identificaci'on}: \req{req:agregar_juegos}

\negrita{Descripci'on}: Cuando los SOCIOS lo deseen, podr'an agregar nuevos juegos al sistema, ampliando las opciones para sus clientes.

\end{itemize}

\clearpage

% requerimientos deseables
\subsubsubsection{Deseables}

\begin{itemize}

\item El sistema permite la comunicaci'on v'ia chat entre jugadores de una misma mesa

\negrita{C'odigo de identificaci'on}: \req{req:chat}

\negrita{Descripci'on}: Los jugadores podr'an expresar entre s'i su felicidad por estar jugando.

\end{itemize}



\subsubsection{Requerimientos no funcionales}

\begin{itemize}

\item El sistema es seguro

\negrita{C'odigo de identificaci'on}: \req{req:sistema_seguro}

\negrita{Descripci'on}: Se necesita que las transacciones sean seguras para brindar mayor confianza a los jugadores y as'i fomentar el uso del casino.

\item El sistema permite suficientes jugadores

\negrita{C'odigo de identificaci'on}: \req{req:suficientes_jugadores}

\negrita{Descripci'on}: El sistema deber'a poder adaptarse f'acilmente para permitir que el n'umero de jugadores sea el requerido para satisfacer las necesidades de rentabilidad.

\item Interfaz intuitiva y sencilla

\negrita{C'odigo de identificaci'on}: \req{req:cool_GUI}

\negrita{Descripci'on}: Es necesario para que los jugadores se sientan c'omodos y no frustrados mientras juegan y as'i incentivarlos a que realicen muchas apuestas (y as'i, que logren su felicidad).

\item Dise'no gr'afico atractivo

\negrita{C'odigo de identificaci'on}: \req{req:graficos_buena_onda}

\negrita{Descripci'on}: Al igual que el anterior, fomentan el uso de los juegos.

\item El sistema es eficaz

\negrita{C'odigo de identificaci'on}: \req{req:sistema_eficaz}

\negrita{Descripci'on}: Se necesita que las operaciones hechas al sistema respondan lo suficientemente r'apido. 'Esto ayuda a lograr una mejor experiencia de usuario, con lo que se fomenta el uso de los juegos.

\end{itemize}


\subsubsection{Compromisos por parte del grupo}

El requerimiento \rref{req:chat} no ser'a cumplido por no llegar a un acuerdo en el presupuesto asignado al desarrollo del producto.

Nos comprometemos en el cumplimiento del resto de los requerimientos por parte del sistema que desarrollaremos.

\clearpage


% Casos de uso
\subsection{Casos de uso \label{CU}}
\subsubsection{Actores}
Se modelaron cuatro actores en los casos de uso del sistema

\begin{description}
\item[Invitado]: un posible futuro jugador al cual se le permite acceder al casino con el objetivo de observar las mesas donde se desarrollan los juegos.
\item[Jugador]: aquel que aparece en la lista de jugadores ingresadada al sistema. Tiene la posibilidad de jugar a cualquier juego asi como cierto saldo de donde se debitan las apuestas y acreditan las ganancias.
\item[Administrador]: un representante del equipo de Timbalistas. Es aquel que realiza tareas administrativas en el sistema.
\item[Jefe de contabilidad]: es aquel capaz de realizar las tareas de Administrador y a su vez ''Dirigir el azar''.
\end{description}

\imagen{CU_Actores.png}{Diagrama de herencia de los actores}{0.6}


\subsubsection{Diagrama}
\imagen{CU_CasosDeUso.png}{Diagrama de casos de uso}{0.6}


\subsubsection{Descripci'on de cada caso de uso}

% --------------- Jugando en una mesa ---------------------
\subsubsubsection{CU: Jugando en una mesa}
\negrita{Pre condici'on}: - \newline
\indent\negrita{Post condici'on}: -

\negrita{Actor primario}: Jugador \newline
\indent\negrita{Actores secundarios}: -

\negrita{Desarrollo normal}
\begin{enumerate}
\item El sistema le muestra al jugador los tipos de juego que dispone el casino.
\item El jugador elige el tipo de juego al que desea jugar y si desea unirse a una mesa existente o crear una nueva.
\item Si elige unirse a una mesa, EXTIENDE ''Eligiendo mesa''. Si en cambio elige crear una mesa, EXTIENDE ''Creando mesa''.
\item Si eligi'o jugar a Craps, EXTIENDE ''Jugando Craps''. Si en cambio eligi'o jugar a Tragamonedas, EXTIENDE ''Tragamonedas''.\label{label_jugar}
\item Si el jugador desea volver a jugar, ir a PASO \ref{label_jugar}.
\item Fin CU.
\end{enumerate}



% --------------- Eligiendo mesa ---------------------
\subsubsubsection{CU: Eligiendo mesa}
\negrita{Pre condici'on}: - \newline
\indent\negrita{Post condici'on}: -

\negrita{Actor primario}: Eligidor de mesa \newline
\indent\negrita{Actores secundarios}: -

\negrita{Desarrollo normal}
\begin{enumerate}
\item El sistema le muestra al eligidor de mesa las mesas abiertas disponibles del juego seleccionado.
\item El eligidor de mesa elige una mesa.
\item Fin CU.
\end{enumerate}



% --------------- Creando mesa ---------------------
\subsubsubsection{CU: Creando mesa}
\negrita{Pre condici'on}: - \newline
\indent\negrita{Post condici'on}: -

\negrita{Actor primario}: Creador de mesa \newline
\indent\negrita{Actores secundarios}: -

\negrita{Desarrollo normal}
\begin{enumerate}
\item El sistema le muestra
\item El creador de mesa 
\item Fin CU.
\end{enumerate}



% --------------- Jugando a craps ---------------------
\subsubsubsection{CU: Jugando a craps}
\negrita{Pre condici'on}: - \newline
\indent\negrita{Post condici'on}: -

\negrita{Actor primario}: Jugador de craps \newline
\indent\negrita{Actores secundarios}: -

\negrita{Desarrollo normal}
\begin{enumerate}
\item Fin CU.
\end{enumerate}



% --------------- Jugando a tragamonedas ---------------------
\subsubsubsection{CU: Jugando a tragamonedas}
\negrita{Pre condici'on}: - \newline
\indent\negrita{Post condici'on}: -

\negrita{Actor primario}: Jugador de tragamonedas \newline
\indent\negrita{Actores secundarios}: -

\negrita{Desarrollo normal}
\begin{enumerate}
\item Fin CU.
\end{enumerate}




% --------------- Mirando mesa ---------------------
\subsubsubsection{CU: Mirando mesa}
\negrita{Pre condici'on}: El invitado no esta mirando ninguna mesa. \newline
\indent\negrita{Post condici'on}: El invitado esta mirando una mesa.

\negrita{Actor primario}: Invitado \newline
\indent\negrita{Actores secundarios}: -

\negrita{Desarrollo normal}
\begin{enumerate}
\item El sistema le muestra al invitado los tipos de juego que dispone el casino.
\item El invitado elige el tipo de juego que desea mirar.
\item El sistema le muestra al invitado las mesas abiertas del juego elegido.
\item El invitado elige la mesa que desea mirar.
\item El sistema le muestra al invitad la mesa elegida.
\item Fin CU.
\end{enumerate}




% --------------- Saliendo de mesa ---------------------
\subsubsubsection{CU: Saliendo de mesa}
\negrita{Pre condici'on}: El invitado esta mirando una mesa. \newline
\indent\negrita{Post condici'on}: El invitado no esta mirando ninguna mesa.

\negrita{Actor primario}: Invitado \newline
\indent\negrita{Actores secundarios}: -

\negrita{Desarrollo normal}
\begin{enumerate}
\item El jugador le informa al sistema que no desea mirar mas la mesa que esta mirando.
\item El sistema le deja de mostrar la mesa.
\item Fin CU.
\end{enumerate}




% --------------- Abriendo casino ---------------------
\subsubsubsection{CU: Abriendo casino}
\negrita{Pre condici'on}: El casino est'a cerrado. \newline
\indent\negrita{Post condici'on}: El casino est'a abierto.

\negrita{Actor primario}: Administrador \newline
\indent\negrita{Actores secundarios}: -

\negrita{Desarrollo normal}
\begin{enumerate}
\item El sistema carga la lista de clientes con sus saldos y las configuraciones de craps, tragamonedas y pozo feliz.
\item El sistema abre el casino.
\item Fin CU.
\end{enumerate}




% --------------- Cerrando casino ---------------------
\subsubsubsection{CU: Cerrando casino}
\negrita{Pre condici'on}: El casino est'a abierto. \newline
\indent\negrita{Post condici'on}: El casino est'a cerrado.

\negrita{Actor primario}: Administrador \newline
\indent\negrita{Actores secundarios}: -

\negrita{Desarrollo normal}
\begin{enumerate}
\item El sistema verifica que no hayan mesas abiertas.
\item El sistema cierra el casino. \label{label_mesas_abiertas}
\item Fin CU. \label{label_fincucc}
\end{enumerate}

\negrita{Desarrollo alternativo}

%\begin{enumerate}
\ref{label_mesas_abiertas}.1 Si hay mesas abiertas, el sistema le advierte al administrador que la operaci'on no puede ser realizada. Ir a PASO \ref{label_fincucc}.
%\end{enumerate}




% --------------- Pidiendo reporte ---------------------
\subsubsubsection{CU: Pidiendo reporte}
\negrita{Pre condici'on}: - \newline
\indent\negrita{Post condici'on}: -

\negrita{Actor primario}: Administrador \newline
\indent\negrita{Actores secundarios}: -

\negrita{Desarrollo normal}
\begin{enumerate}
\item El sistema le muestra al administrador los posibles reportes a pedir: Ranking de jugadores, Estado actual y Detalle movimientos por jugador.
\item El administrador elige el reporte que desea.
\item El sistema le muestra al administrador el reporte solicitado.
\item Fin CU.
\end{enumerate}




% --------------- Dirigiendo el azar ---------------------
\subsubsubsection{CU: Dirigiendo el azar}
\negrita{Pre condici'on}: - \newline
\indent\negrita{Post condici'on}: La/s mesa/s seleccionadas por el jefe de contabilidad se comportar'an de la forma se'nalada por 'este.

\negrita{Actor primario}: Jefe de contabilidad \newline
\indent\negrita{Actores secundarios}: -

\negrita{Desarrollo normal}
\begin{enumerate}
\item El sistema le muestra al jefe de contabilidad las opciones de
	\begin{itemize}
	\item Activar/desactivar modo dirigido de craps.
	\item Activar/desactivar modo dirigido de tragamonedas.
	\item Activar ''todos ponen''.
	\item Activar ''jugada feliz''.
	\end{itemize}
\item El jefe de contabilidad elige la opci'on deseada e ingresa los parametros necesarios.
\item El sistema efect'ua los cambios correspondientes sobre las mesas correspondientes.
\item Fin CU.
\end{enumerate}



\clearpage


% Diagramas de actividad
\subsection{Diagramas de actividad \label{DA}}
\subsubsection{Tragamonedas}
\imagen{DA_Tragamonedas}{Desarrollo de una jugada de tragamonedas}{0.45}
\clearpage


\subsubsection{Craps}
\imagen{DA_Craps}{Desarrollo de una jugada de craps}{0.32}
\clearpage


\subsubsection{Tipo de juego}
\imagen{DA_TipoDeJugada}{Desici'ones tomadas para decidir si una jugada es feliz o todos ponen junto a las acciones que 'estas desencadenan}{0.47}


\clearpage

\subsection{M'aquinas de estado finitas \label{FSM} }

Con las m'aquinas de estado modelaremos cierta funcionalidad:
 
\begin{center}
\begin{tabular}{p{3.5cm}|p{8cm}}
        
        \multicolumn{2}{c}{ \negrita{CRAPS}  }\\
        \hline 
        Ronda & modelamos los estados de una ronda de Craps \\
        \hline
        Crupier & esta ser'a una vista de una de las funcionalidades provista por el sistema, aqu'i se podra ver parte de la interaci'on de los jugadores con el sistema. \\
         \hline 
        Jugardor i & modela un jugador en particular en su rol de tirador \\
        \hline 
        Apuesta de un tiro& modela jugador haciendo/cancelando una o varias apuestas que duran un tiro \\
        \hline 
        Apuesta de una ronda& \italica{'idem} para una ronda. \\
        \hline 
        Apuesta de m'as de una ronda& \italica{'idem} para m'as de una ronda. \\
\end{tabular}
\end{center}

\begin{center}
\begin{tabular}{p{3cm}|p{8cm}}
        \multicolumn {2}{c}{\negrita{TRAGAMONEDAS}}  \\
        \hline
        Mesa i & din'amica de la mesa i donde juega el jugador i\\
        \hline
        Jugador i & din'amica del jugador i
\end{tabular}
\end{center}

\begin{table}[p!hbt]
 \begin{center}

 \begin{tabular}{p{3cm}|p{3cm}|p{3cm}|p{5cm}}
    
    \multicolumn{4}{c}{Donde se modela cada apuesta}  \\
    \hline
    \italica{Apuesta} & \negrita{Cuando} & \negrita{se resuelve} & \negrita{Duraci'on / lo modela} \\ 
    \hline
    \vskip0.05cm \negrita{Linea de pase o linea de no pase } &\vskip0.05cm  Antes del tiro de salida &\vskip0.05cm Craps o Natural, \negrita{cuando sale el punto}, 7 o punto &\vskip0.05cm  una ronda / FSM una ronda  \\ 
    \hline
    \vskip0.05cm \negrita{Venir ó No venir}&\vskip0.05cmDespues del punto   &\vskip0.05cm Natural o Craps, \negrita{cuando sale el punto}, 7 o punto   & $n$ rondas / FSM n rondas\\
    \hline
    \vskip0.05cm \negrita{Campo}&\vskip0.05cmAntes de Cualquier tiro   &\vskip0.05cmCon los dados que salier'on   & un tiro / FSM un tiro\\  
    \hline
    \vskip0.05cm \negrita{Sitio}&\vskip0.05cmAntes de Cualquier tiro   &\vskip0.05cmCon los dados que salier'on  & $n$ rondas / FSM n rondas  \\
    

\end{tabular}

\end{center}

\end{table}




Glosario de estiquetas de las FSM's:


\begin{center}
    \begin{tabular}{p{4cm}|p{8cm}}
    
    \multicolumn{2}{c}{Referencias de las etiquetas y Variable de la \negrita{FSM jugador i}} \\
    \hline
    \negrita{Etiqueta de la transici'on} & \negrita{Acci'on} \\
    \hline
    serTirador i & el jugador i pasa a ser tirador.\\
    \hline
    estanSaliendo i & \italica{el tirador i tira} los dados la primera vez en esa ronda \\
    \hline
    natural i& ''  un 7 u 11 \\
    \hline
    craps i & ''   2, 3 o 12\\
    \hline
    punto i & ''  establece el punto \\ 
    \hline
    tirarDados i & '' los dados \\
    \hline
    sevenOut i & ''  un 7 antes de repetir el punto \\
    \hline 
    ganoCraps i & '' acierta el punto\\
    \hline
    dejarDados i  & '' deja los dados  

    \label{glosarioFSMjugadori}
    \end{tabular}
\end{center}

\begin{center}
    \begin{tabular}{p{4cm}|p{8cm}}
    
\multicolumn{2}{c}{Referencias de las etiquetas de la \textbf{FSM Crupier} } \\

            \hline
\negrita{Etiqueta de la 
transici'on} & \negrita{Acci'on} \\
    \hline
    darleDadosAl ProximoTirador & los dados pasan al pr'oximo jugador  \\
\hline
     \italica{Variable} tirador & almancena el n'umero de tirador actual o candidato a tirador si decidiera dejar los dados\\
        \multicolumn{2}{c}{ }  \\
    
    \end{tabular}
\end{center}

\begin{center}
    \begin{tabular}{p{4cm}|p{8cm}}
    
     \multicolumn{2}{c}{Referencias de las etiquetas de la \textbf{FSM Apuesta m'as de una ronda i } } \\
    \hline
    \negrita{Etiqueta de la transici'on} & \negrita{Acci'on} \\

    apuestaDeMasDeUna Ronda i & el jugador i hace una apuesta  \\
\hline
    perdi'oApuestaDeUna Ronda i & el jugador i gano su apuesta \\

\hline
    perdi'oApuestaDeUna Ronda i & el jugador i perdió su apuesta    \\  
        
\end{tabular}
\end{center}


\begin{center}
    \begin{tabular}{p{4cm}|p{8cm}}
    
     \multicolumn{2}{c}{Referencias de las etiquetas de la \textbf{FSM Apuesta de un tiro i } } \\
    \hline
    \negrita{Etiqueta de la transici'on} & \negrita{Acci'on} \\

    apuestaDeMasDeUna Ronda i & el jugador i hace una apuesta  \\
\hline
    perdi'oApuestaDeUna Ronda i & el jugador i gano su apuesta \\

\hline
    perdi'oApuestaDeUna Ronda i & el jugador i perdió su apuesta    \\  
        
\end{tabular}
\end{center}

\imagen{FSM_Ronda.png}{FSM Ronda}{0.5}
\imagen{FSM_Crupier.png}{FSM Crupier}{0.5}
\imagen{FSM_Jugador_i.png}{FSM Jugador i}{0.5}

\imagenvertical{FSM_Apuesta_mas_de_una_ronda.png}{FSM Apuesta de más de una Ronda}{0.6}
FALTA LA DE UNA RONDA... MIRANDOLA FIJOOOOOOO
% \imagenvertical{FSM_Apuesta_una_ronda.png}{FSM Apuesta de una Ronda}{0.4}
\imagenvertical{FSM_Apuesta_un_tiro.png}{FSM Apuesta de un Tiro}{0.8}



Limitaciones del modelo:

FSM apuesta de una ronda.

Por la traza, el jugador podria apostar en una ronda y en la siguiente ronda cancelar la apuesta, 
esta es una limitacion del modelo. Entendemos que una apuesta una vez que se tiraron los dados no se puede retirar ni modificar. Lo que podrá cancelar serán las n apuestas anteriores antes de que se tiren los dados.


\clearpage

% Modelo conceptual y restricciones ocl
\subsection{Modelo conceptual\label{MC}}
\subsubsection{Diagrama de clases conceptuales}
Para comprender mejor las relaciones estructurales entre las entidades del sistema, aqu'i se muestra el modelo conceptual del mismo. 

\imagenvertical{MC_Diagrama.png}{Diagrama de clases conceptuales}{0.5}


\subsubsection{Detalles de las clases}
A continuaci'on se comenta el significado y raz'on de ser de las clases principales en el diagrama
\begin{description}
\item [Jugada] Para poder obtener los reportes de:
\begin{itemize}
 \item \italica{Ranking de jugadores} 
\item \italica{Detalle de movimientos por jugador} 
\end{itemize}
 es necesario guardar el historial de jugadas realizadas por cada jugador en la jornada. 
La clase \negrita{jugada}: una jugada realizada por alg'un jugador en alguna mesa en alg'un momento de la jornada del casino. Puede ser de tipo Tragamonedas o Craps; reflejando en el primer caso una apuesta de fichas y la posibilidad de haber girado los rodillos y en el segundo a las apuestas de los jugadores con la posibilidad de haber realizado un tiro (notar que una jugada de Craps \negrita{no} es una ronda sino un tiro de dados). Puede estar asociada a su resultado si es que ya se jug'o, o no tenerlo si es que todav'ia no se tiraron los dados en el caso de Craps o no se giraron los rodillos en el caso de Tragamonedas. 

\item [Invitado] Es un visitante del casino. Puede mirar las mesas para ver jugar a sus jugadores.

\item [Jugador] Es aquel capaz de realizar jugadas en una mesa. Un jugador \italica{es} a la vez un invitado. Es decir, puede mirar una mesa sin necesidad de jugar.
\item [Jornada] Representa una jornada del casino, desde que abre hasta que cierra.
\item [Mesa] Es en donde se realizan las jugadas. Puede ser de Tragamonedas o Craps y en una jornada puede recibir cualquier cantidad de jugadas.
\end{description}


\subsubsection{Restricciones OCL al modelo conceptual}
\begin{itemize}



\item \textit{Solo hay una jugada feliz como m'aximo}

\textbf{context} Jugada \\ \textbf{inv:}
    Jugada.allInstances()$\rightarrow$select(j $|$ j.tipo = TipoJugada::feliz \textbf{and} j.result'oEn$\rightarrow$isEmpty())$\rightarrow$size() $\leq$ 1)



\item \textit{Para cada mesa de Craps hay exactamente un tirador}

% self.jugadas->select(j | j.result'oEn->isEmpty() and j.oclAsType(JugadaCraps).tirador = true)->size() = 1)
\textbf{context}  MesaCraps \\ \textbf{inv:} 
    self.jugadas$\rightarrow$select(j $|$ j.result'oEn$\rightarrow$isEmpty() \textbf{and} j.oclAsType(JugadaCraps).tirador = true)$\rightarrow$size() = 1)



\item \textit{En las mesas de Craps s'olo se realizan apuestas de tipo Crap} 

% self.jugadas->forAll(j | j.oclIsTypeOf(JugadaCraps))
\textbf{context}  MesaCraps \\ \textbf{inv:} 
    self.jugadas$\rightarrow$forAll(j $|$ j.oclIsTypeOf(JugadaCraps))



\item \textit{En las mesas de Tragamonedas s'olo se realizan apuestas de tipo Tragamoneda}

% self.jugadas->forAll(j | j.oclIsTypeOf(JugadaTragamonedas))
\textbf{context}  MesaTragamonedas \\ \textbf{inv:} 
    self.jugadas$\rightarrow$forAll(j $|$ j.oclIsTypeOf(JugadaTragamonedas))



\item \textit{Los jugadores comunes no tienen saldo negativo}

% self.saldo >= 0
\textbf{context} JugadorNormal \\ \textbf{inv:}
    self.saldo $\geq$ 0
  
  

\item \textit{No hay jugadores con el mismo dni}

% Jugador.allInstances()->forAll(j1, j2 : Jugador | j1 != j2 ---> j1.dni != j2.dni)
\textbf{context}  Jugador \\ \textbf{inv:}
    Jugador.allInstances()$\rightarrow$forAll($j_{1}$, $j_{2}$ : Jugador $|$ $j_{1} < > j_{2}$ \textbf{implies} $j_{1}.dni < > j_{2}.dni$)



\item \textit{El tipo de jugada (normal, feliz o todosPonen) tiene que ser el mismo para las jugadas de una misma mesa}

% self.jugadas->select(j | j.result'oEn->isEmpty())->forAll(j1, j2 : Jugada | j1.tipo = j2.tipo)
\textbf{context} Mesa \\ \textbf{inv:}
    self.jugadas$\rightarrow$select(j $|$ j.result'oEn$\rightarrow$isEmpty())$\rightarrow$forAll($j_{1}$, $j_{2}$ : Jugada $|$ $j_{1}$.tipo = $j_{2}$.tipo)
  
  

\item \textit{El resultado de una jugada de Craps debe ser del mismo tipo que la jugada}

% self.result'oEn->notEmpty ---> self.result'oEn.oclIsTypeOf(ResultadoCraps)
\textbf{context} JugadaCraps \\ \textbf{inv:}
    self.result'oEn$\rightarrow$notEmpty() \textbf{implies} self.result'oEn.oclIsTypeOf(ResultadoCraps)



\item \textit{El resultado de una jugada de Tragamonedas debe ser del mismo tipo que la jugada}

% self.result'oEn->notEmpty ---> self.result'oEn.oclIsTypeOf(ResultadoTragamonedas)
\textbf{context} JugadaTragamonedas \\ \textbf{inv:}
    self.result'oEn$\rightarrow$notEmpty() \textbf{implies} self.result'oEn.oclIsTypeOf(ResultadoTragamonedas)



\item \textit{El n'umero salido en un resultado de craps debe ser entre 2 y 12}

% 2 <= self.n'umeroSalido <= 12
\textbf{context} ResultadoCraps \\ \textbf{inv:}
    2 $\leq$ self.n'umeroSalido $\leq$ 12



\item \textit{Definici'on del atributo derivado Jugada::pago para una JugadaTragamonedas}

\textbf{context} JugadaTragamonedas::pago : Cr'edito \\ \textbf{derive:}
    \textbf{let} result : ResultadoTragamonedas = self.result'oEn.oclAsType(ResultadoTragamonedas)

    \textbf{let} esBar(figura : FiguraRodillo) : Boolean = figura = FiguraRodillo::barSimple \textbf{or} figura = FiguraRodillo::barDoble \textbf{or} figura = FiguraRodillo::barTriple

    \textbf{let} correspondePremioDinosaurio : Boolean = result.rodillo1 = FiguraRodillo::dinosaurio and result.rodillo2 = FiguraRodillo::dinosaurio and result.rodillo3 = FiguraRodillo::dinosaurio
    
    \textbf{let} correspondePremioTresCerezas : Boolean = result.rodillo1 = FiguraRodillo::cereza and result.rodillo2 = FiguraRodillo::cereza and result.rodillo3 = FiguraRodillo::cereza
    
    \textbf{let} correspondePremioBarTriple : Boolean = result.rodillo1 = FiguraRodillo::barTriple and result.rodillo2 = FiguraRodillo::barTriple and result.rodillo3 = FiguraRodillo::barTriple
    
    \textbf{let} correspondePremioBarDoble : Boolean = result.rodillo1 = FiguraRodillo::barDoble and result.rodillo2 = FiguraRodillo::barDoble and result.rodillo3 = FiguraRodillo::barDoble
    
    \textbf{let} correspondePremioBarSimple : Boolean = result.rodillo1 = FiguraRodillo::barSimple and result.rodillo2 = FiguraRodillo::barSimple and result.rodillo3 = FiguraRodillo::barSimple
    
    \textbf{let} correspondePremioTresBares : Boolean = esBar(result.rodillo1) and esBar(result.rodillo2) and esBar(result.rodillo3)
    
    \textbf{let} correspondePremioDosCerezas : Boolean = Bag{}->including(result.rodillo1)$\rightarrow$including(result.rodillo2)$\rightarrow$including(result.rodillo3)$\rightarrow$count(FiguraRodillo::cereza) = 2
    
    \textbf{let} correspondePremioUnaCereza : Boolean = result.rodillo1 = FiguraRodillo::cereza \textbf{or} result.rodillo2 = FiguraRodillo::cereza \textbf{or} result.rodillo3 = FiguraRodillo::cereza

    \textbf{let} premioDinosaurio(fichas : FichasApostadas) : Cr'edito =\\
        \textbf{if} fichas = FichasApostadas::unaFicha \textbf{then} 1000\\
        \textbf{else} \textbf{if} fichas = FichasApostadas::dosFichas \textbf{then} 2000\\
        \textbf{else} \textbf{if} fichas = FichasApostadas::tresFichas \textbf{then} 5000\\
        \textbf{endif}

    \textbf{let} premioTresCerezas(fichas : FichasApostadas) : Cr'edito =\\
        \textbf{if} fichas = FichasApostadas::unaFicha \textbf{then} 160\\
        \textbf{else} \textbf{if} fichas = FichasApostadas::dosFichas \textbf{then} 320\\
        \textbf{else} \textbf{if} fichas = FichasApostadas::tresFichas \textbf{then} 480\\
        \textbf{endif}

    \textbf{let} premioBarTriple(fichas : FichasApostadas) : Cr'edito =\\
        \textbf{if} fichas = FichasApostadas::unaFicha \textbf{then} 80\\
        \textbf{else} \textbf{if} fichas = FichasApostadas::dosFichas \textbf{then} 160\\
        \textbf{else} \textbf{if} fichas = FichasApostadas::tresFichas \textbf{then} 240\\
        \textbf{endif}

    \textbf{let} premioBarDoble(fichas : FichasApostadas) : Cr'edito =\\
        \textbf{if} fichas = FichasApostadas::unaFicha \textbf{then} 40\\
        \textbf{else} \textbf{if} fichas = FichasApostadas::dosFichas \textbf{then} 80\\
        \textbf{else} \textbf{if} fichas = FichasApostadas::tresFichas \textbf{then} 120\\
        \textbf{endif}

    \textbf{let} premioBarSimple(fichas : FichasApostadas) : Cr'edito =\\
        \textbf{if} fichas = FichasApostadas::unaFicha \textbf{then} 20\\
        \textbf{else} \textbf{if} fichas = FichasApostadas::dosFichas \textbf{then} 40\\
        \textbf{else} \textbf{if} fichas = FichasApostadas::tresFichas \textbf{then} 60\\
        \textbf{endif}

    \textbf{let} premioTresBares(fichas : FichasApostadas) : Cr'edito =\\
        \textbf{if} fichas = FichasApostadas::unaFicha \textbf{then} 10\\
        \textbf{else} \textbf{if} fichas = FichasApostadas::dosFichas \textbf{then} 20\\
        \textbf{else} \textbf{if} fichas = FichasApostadas::tresFichas \textbf{then} 30\\
        \textbf{endif}

    \textbf{let} premioDosCerezas(fichas : FichasApostadas) : Cr'edito =\\
        \textbf{if} fichas = FichasApostadas::unaFicha \textbf{then} 5\\
        \textbf{else} \textbf{if} fichas = FichasApostadas::dosFichas \textbf{then} 10\\
        \textbf{else} \textbf{if} fichas = FichasApostadas::tresFichas \textbf{then} 15\\
        \textbf{endif}

    \textbf{let} premioUnaCereza(fichas : FichasApostadas) : Cr'edito =\\
        \textbf{if} fichas = FichasApostadas::unaFicha \textbf{then} 2\\
        \textbf{else} \textbf{if} fichas = FichasApostadas::dosFichas \textbf{then} 4\\
        \textbf{else} \textbf{if} fichas = FichasApostadas::tresFichas \textbf{then} 6\\
        \textbf{endif}

    \textbf{if} self.result'oEn$\rightarrow$isEmpty() \textbf{then}\\
        0\\
    \textbf{else}\\
        \textbf{if} correspondePremioDinosaurio \textbf{then} premioDinosaurio(self.apuesta)\\
        \textbf{else} \textbf{if} correspondePremioTresCerezas \textbf{then} premioTresCerezas(self.apuesta)\\
        \textbf{else} \textbf{if} correspondePremioBarTriple \textbf{then} premioBarTriple(self.apuesta)\\
        \textbf{else} \textbf{if} correspondePremioBarDoble \textbf{then} premioBarDoble(self.apuesta)\\
        \textbf{else} \textbf{if} correspondePremioBarSimple \textbf{then} premioBarSimple(self.apuesta)\\
        \textbf{else} \textbf{if} correspondePremioTresBares \textbf{then} premioTresBares(self.apuesta)\\
        \textbf{else} \textbf{if} correspondePremioDosCerezas \textbf{then} premioDosCerezas(self.apuesta)\\
        \textbf{else} \textbf{if} correspondePremioUnaCereza \textbf{then} premioUnaCereza(self.apuesta)\\
        \textbf{else} 0\\
        \textbf{endif}\\
    \textbf{endif}

\clearpage



\item\textit{Definir Jugada.pago}
FALTA!!!!


\item\textit{QUE ONDA CON LAS APUESTAS DE CRAPS QUE DURAN MAS DE 1 TIRO}
FALTA!!!!!!!!!!!!!!!!!!!!!!!!!!!!!!!!!!!!!


\item\textit{ En una jugada Crap las apuestas en sitio a ganar de un jugador son a distintos n'umeros}

\textbf{context}  JugadaCrap \\ \textbf{inv:} 
  self.ApuestaEnSitioAGanar$\rightarrow$forAll( $a_{1}, a_{2}$ : ApuestaEnSitioAGanar $ | $ $ a_{1} <> a_{2} $  \textbf{implies} $ a_{1}.valor <> a_{2}.valor $ )


\item\textit{ En una jugada Crap las apuestas en sitio a perder de un jugador son a distintos n'umeros}

\textbf{context}  JugadaCrap \\ \textbf{inv:} 
  self.ApuestaEnSitioAPerder$\rightarrow$forAll($a_{1}$, $a_{2}$ : ApuestaEnSitioAPerder  $ | $ $a_{1} <> a_{2} $ \textbf{implies} $a_{1}$.valor $<>$ $a_{2}$.valor)

\end{itemize}

\clearpage

\subsection{Prototipos de pantallas\label{PROTO}}



	\begin{figure}[p!hbt]
		\centering
		\includegraphics[width=0.8\textwidth]{../img/PP_Login.png}
		\caption{Prototipo de pantalla: Login }
		\label{fig:login}
	\end{figure}


        \begin{figure}[p!hbt]

		\centering
		\includegraphics[width=0.8\textwidth]{../img/PP_Lobby.png}
		\caption{Prototipo de pantalla: Lobby }
		\label{fig:lobby}
	\end{figure}

	\begin{figure}[p!hbt]
		\centering
		\includegraphics[width=0.8\textwidth]{../img/PP_Traga.png}
		\caption{Prototipo de pantalla: Tragamonedas }
		\label{fig:traga}
	\end{figure}

	

	\begin{figure}[p!hbt]
		\centering
		\includegraphics[width=0.8\textwidth]{../img/PP_Craps.png}
		\caption{Prototipo de pantalla: Craps }
		\label{fig:craps}
	\end{figure}

% \imagen{img/PP_Lobby.png}{Prototipo de pantalla: Lobby}{0.6}
% \imagen{img/PP_Traga.png}{Prototipo de pantalla: Tragamonedas}{0.6}
% \imagen{img/PP_Craps.png}{Prototipo de pantalla: Craps}{0.6}

\clearpage

\subsection{Requerimientos No alcanzados por ning'un modelo \label{RND}}

\begin{itemize}
\item \rrefImportante{req:acceso_multiple} El sistema permite acceder al casino a m'as de un jugador a la vez desde la misma terminal, en ningun lugar se restringe esta funcionalidad, el equipo se compromete a cumplirla.
\item \rrefImportante{req:agregar_juegos} El sistema permite ser extendido mediante la incorporaci'on de nuevos juegos, en ningun lado se acotan la cantidad de juegos a incorporar, por lo que el equipo tambien se compromete a cumplir con este requerimiento.

 
\end{itemize}
