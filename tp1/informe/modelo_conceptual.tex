\subsubsection{Diagrama de clases conceptuales}
Para comprender mejor las relaciones estructurales entre las entidades del sistema, aqu'i se muestra el modelo conceptual del mismo. 

\imagenvertical{MC_Diagrama.png}{Diagrama de clases conceptuales}{0.5}


\subsubsection{Detalles de las clases}
A continuaci'on se comenta el significado y raz'on de ser de las clases principales en el diagrama
\begin{description}
\item [Jugada] Para poder obtener los reportes de \italica{Ranking de jugadores} y \italica{Detalle de movimientos por jugador} es necesario guardar el historial de jugadas realizadas por cada jugador en la jornada. La clase jugada representa exactamente eso: una jugada realizada por alg'un jugador en alguna mesa en alg'un momento de la jornada del casino. Una jugada puede ser de Tragamonedas o Craps; reflejando en el primer caso una apuesta de fichas y la posibilidad de haber girado los rodillos y en el segundo a las apuestas de los jugadores con la posibilidad de haber realizado un tiro (notar que una jugada de Craps \negrita{no} es una ronda si no un tiro de dados). Puede estar asociada a su resultado si es que ya se jug'o, o no tenerlo si es que todavia no se tiraron los dados en el caso de Craps o no se giraron los rodillos en el caso de Tragamonedas. 
\item [Invitado] Es un visitante del casino (se le podr'ia decir cliente del mismo). Puede mirar las mesas para ver jugar a sus jugadores.
\item [Jugador] Es aquel capaz de realizar jugadas en una mesa. Siendo coherentes con el modelo de casos de uso, un jugador \italica{es} a la vez un invitado. Es decir, puede mirar una mesa sin necesidad de jugar.
\item [Jornada] Representa una jornada del casino, desde que abre hasta que cierra.
\item [Mesa] Es en donde se realizan las jugadas. Puede ser de Tragamonedas o Craps y en una jornada puede recibir cualquier cantidad de jugadas.
\end{description}


\subsubsection{Restricciones OCL al modelo conceptual}
\begin{itemize}
 \item \textit{Solo hay una jugada feliz como m'aximo \\}
  \textbf{context} Jugada 
  \\ \textbf{inv:}
  Jugada.allInstances()$\rightarrow$select(tipo = TipoJugada::feliz)$\rightarrow$size() $\leq$ 1)


\item \textit{ Para cada mesa de Craps hay exactamente un tirador \\}
\textbf{context}  MesaDeCraps \\ \textbf{inv:} 
  self.Jugada$\rightarrow$select(j $|$ j.oclAsType(JugadaCrap).tirador = true)$\rightarrow$size() = 1)


\item \textit{En las mesas de Craps s'olo se realizan apuestas de tipo Crap} 

\textbf{context}  MesaDeCraps \\ \textbf{inv:} 
  self.Jugada$\rightarrow$forAll(j $|$ j.oclIsTypeOf(JugadaCrap))


\item \textit{En las mesas de Tragamonedas s'olo se realizan apuestas de tipo Tragamoneda}

\textbf{context}  MesaDeTragamonedas \\ \textbf{inv:} 
  self.Jugada$\rightarrow$forAll(j $|$ j.oclIsTypeOf(JugadaTragamoneda))

\item \textit{Los jugadores comunes no tienen saldo negativo}

\textbf{context}  JugadorComun \\ \textbf{inv:} 
  self.saldo $\geq$ 0

\item \textit{ No hay jugadores con el mismo dni}

\textbf{context}  Jugador \\ \textbf{inv:} 
  Jugador.allInstances()$\rightarrow$forAll($j_{1}$, $j_{2}$ : Jugador $|$ $j_{1} < > j_{2}$ \textbf{implies} $j_{1}.dni < > j_{2}.dni$)


\item\textit{ El tipo de jugada (normal, feliz o todosPonen) tiene que ser el mismo para las jugadas de una misma mesa}

\textbf{context}  Mesa \\ \textbf{inv:} 
  self.Jugada$\rightarrow$forAll($j_{1}$, $j_{2}$ : Jugada $|$ $j_{1}$.tipo = $j_{2}$.tipo)

  
\item\textit{El resultado de una jugada debe ser del mismo tipo que la jugada}
FALTA!!!!!


\item\textit{El numero salido en un resultado de craps debe ser entre 2 y 12}
FALTA!!!!!


\item\textit{Definir Jugada.pago}
FALTA!!!!


\item\textit{QUE ONDA CON LAS APUESTAS DE CRAPS QUE DURAN MAS DE 1 TIRO}
FALTA!!!!!!!!!!!!!!!!!!!!!!!!!!!!!!!!!!!!!


\item\textit{ En una jugada Crap las apuestas en sitio a ganar de un jugador son a distintos n'umeros}

\textbf{context}  JugadaCrap \\ \textbf{inv:} 
  self.ApuestaEnSitioAGanar$\rightarrow$forAll( $a_{1}, a_{2}$ : ApuestaEnSitioAGanar $ | $ $ a_{1} <> a_{2} $  \textbf{implies} $ a_{1}.valor <> a_{2}.valor $ )


\item\textit{ En una jugada Crap las apuestas en sitio a perder de un jugador son a distintos n'umeros}

\textbf{context}  JugadaCrap \\ \textbf{inv:} 
  self.ApuestaEnSitioAPerder$\rightarrow$forAll($a_{1}$, $a_{2}$ : ApuestaEnSitioAPerder  $ | $ $a_{1} <> a_{2} $ \textbf{implies} $a_{1}$.valor $<>$ $a_{2}$.valor)

\end{itemize}
