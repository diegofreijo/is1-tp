
\newcommand{\ronda}{ \italica{ FSM Ronda} }
\newcommand{\crupier}{ \italica{ FSM Crupier} }
\newcommand{\tirador}{ \italica{ FSM tirador i} }
\newcommand{\unaRonda}{\italica{FSM Jugador i haciendo apuestas de \textbf{una ronda}}}
\newcommand{\muchasRondas}{\italica{FSM Jugador i haciendo apuestas en\textbf{ m'as de una ronda}}}
\newcommand{\unTiro}{\italica{FSM Jugador i haciendo apuestas de \textbf{un tiro}}}

\subsubsection{Craps}

El juedo de craps consta de varios tipos de apuestas, decidimos usar
un nombre genérico para los nombres de las máquinas en pos de buscar mayor generalidad.
Con estas de estado modelaremos cierta funcionalidad:

\begin{center}
\begin{tabular}{p{4cm}|p{8cm}}        
        \ronda & modelamos los estados de una ronda de Craps \\
        \hline
        \crupier & esta ser'a una vista de una de las funcionalidades provista por el sistema, aqu'i se pod'a ver parte de la interaCci'on de los jugadores con el sistema. \\
         \hline 
         \tirador  & modela un jugador en particular en su rol de tirador \\
        \hline 
        \unTiro & modela un jugador en particular haciendo/cancelando una o varias apuestas que duran un tiro \\
        \hline 
        \unaRonda& \italica{'idem} para apuestas que duran una ronda. \\
        \hline 
        \muchasRondas & \italica{'idem} para apuestas para m'as de una ronda. \\
\end{tabular}
\end{center}

 \begin{center}
 \begin{tabular}{p{3cm}|p{2cm}|p{3cm}|p{2cm}|p{4cm}}
    \multicolumn{5}{c}{\negrita{Donde se modela cada apuesta}}  \\
    \hline
    \italica{Apuesta} & \negrita{Cuando} & \negrita{Se resuelve} & \negrita{Duraci'on } & \negrita{La modela} \\ 
    \hline
     \vskip0.05cm \negrita{L'inea de pase o L'inea de no pase } &\vskip0.05cm  Antes del tiro de salida &\vskip0.05cm Craps o Natural, \negrita{despues de que sale el punto}, 7 'o punto &\vskip0.05cm  una ronda & \unaRonda  \\

     \hline
     \vskip0.05cm \negrita{Venir ó No venir}&\vskip0.05cmDespues del punto   &\vskip0.05cm Natural o Craps, \negrita{despues de que sale el punto}, 7 'o el punto   & $n$ rondas & \muchasRondas\\
     \hline
     \vskip0.05cm \negrita{Campo}&\vskip0.05cmAntes de cualquier tiro   &\vskip0.05cmCon los dados que salier'on   & un tiro & \unTiro \\  
     \hline
     \vskip0.05cm \negrita{Sitio}&\vskip0.05cmAntes de cualquier tiro   &\vskip0.05cmCon los dados que salier'on  & $n$ rondas & \muchasRondas  \\
     

  \end{tabular}
\end{center}

En al Glosario se encuentran las definiciones de cada etiqueda usada.


\imagen{FSM_Ronda.png}{FSM Ronda}{0.5}
\imagen{FSM_Crupier.png}{FSM Crupier}{0.5}
\imagen{FSM_Jugador_i.png}{FSM Jugador i}{0.5}

\imagenvertical{FSM_Apuesta_mas_de_una_ronda.png}{FSM Apuesta de más de una Ronda}{0.6}
FALTA LA DE UNA RONDA... MIRANDOLA FIJOOOOOOO
% \imagenvertical{FSM_Apuesta_una_ronda.png}{FSM Apuesta de una Ronda}{0.4}
\imagenvertical{FSM_Apuesta_un_tiro.png}{FSM Apuesta de un Tiro}{0.7}



\textbf{Limitaciones del modelo:}

\underline{\unaRonda} :
Analizando la traza, el jugador podr'ia apostar en una ronda y en la siguiente ronda cancelar la apuesta, 
esta es una limitacion del modelo. Entendemos que una apuesta una vez que se tiraron los dados no se puede retirar ni modificar. Lo que podrá cancelar serán las n apuestas anteriores antes de que se tiren los dados, una vez tirados estos no se podr'a cancelar.
