\newcommand{\ronda}{\italica{FSM Ronda}}
\newcommand{\crupier}{\italica{FSM Crupier}}
\newcommand{\tirador}{\italica{FSM Tirador i}}
\newcommand{\dados}{\italica{FSM Dados i}}
\newcommand{\campo}{\italica{FSM Jugador i haciendo apuesta de Campo }}
\newcommand{\sitio}{\italica{FSM Jugador i haciendo apuesta de sitio}}
\newcommand{\pase}{\italica{FSM Jugador i haciendo apuesta linea de pase / barra No Pase}}
\newcommand{\venir}{\italica{FSM Jugador i haciendo apuesta venir / no venir}}

Tanto para Craps como para Tragamonedas, modelaremos c'omo los jugadores interact'uan con el sistema.

\subsubsection{Tragamonedas}

En la figura \ref{fig_FSM/FSM_Jugador_tragamonedas_i.png} vemos la din'amica del juego de tragamonedas (\rrefEsencial{req:existe_juego_tragamoneda}) y qu'e opciones tiene un jugador.
Al jugador $i$ se le asigna la mesa $i$. No es posible elegir jugar en una mesa previamente creada, ya que todas las mesas de tragamonedas existentes est'an ocupadas (en una mesa de tragamonedas solo puede jugar un 'unico jugador) y al abandonar los jugadores las mesas de tragamonedas 'estas son cerradas autom'aticamente por el sistema (una mesa sin jugadores es cerrada autom'aticamente, ver \rrefEsencial{req:auto_cerrar_mesas}), lo que imposibilita el poder elegir una mesa existente.

\imagen{FSM/FSM_Jugador_tragamonedas_i.png}{FSM Jugador tragamonedas i}{0.5}
\clearpage


\subsubsection{Craps}
\label{FSM:Craps}
Para modelar el juego de craps (\rrefEsencial{req:existe_juego_craps}) usaremos varias m'aquinas, en particular: el tirador, el crupier (sistema) y los jugadores haciendo los varios tipos de apuestas posibles. 

En la secci'on \ref{etiquetasFSMs} del Glosario se encuentran las definiciones de cada etiqueta usada.

 
\begin{center}
\begin{tabular}{p{4cm}|p{12cm}}        
         \multicolumn{2}{c}{M'AQUINAS USADAS PARA MODELAR EL CRAPS}     \\
        \hline
        \ronda & Modelamos los estados de una ronda de Craps \\
        \hline
        \crupier & Esta ser'a una vista de una de las funcionalidades provista por el sistema, aqu'i se podr'a ver parte de la interacci'on de los jugadores con el sistema. \\
         \hline 
         \tirador  & Modela un jugador en particular en su rol de tirador. Usamos una variable temporal para mostrar que un tirador no puede quedarse con los dados indefinidamente. Las constantes $k1$ y $k2$ son constantes de tiempo a definir en la etapa de dise\~{n}o.\\
 \hline
        \campo  & el jugador i haciendo apuesta de campo \\
\hline 
       \sitio& el jugador i haciendo apuestas de sitio \\
\hline
        \pase& el jugador i haciendo apuestas de pase o no pase \\
\hline        
        \venir& el jugador i haciendo apuestas de venir o no venir
\end{tabular}
\end{center}

\begin{center}
    \begin{tabular}{p{4cm}|p{1cm}|p{6cm}}        
        \multicolumn{3}{c}{Variables}\\
        \hline
        dados & 2...12 & el valor de los dados\\
        \hline
        nroAS & 2...12 & el valor de los dados vinculado a la apuesta de Sitio\\
        \hline
        tirador & 1...n & el tirador actual\\
        \hline
        punto & 2...12 & el valor del punto\\
        \hline
        puntoVenirNoVenir i & 2...12 & el valor de los dados vinculado al punto de la apuesta venir / noVenir\\
        \hline
    \end{tabular}
\end{center}
 

\imagen{FSM/FSM_Ronda.png}{\ronda}{0.5}
\imagen{FSM/FSM_Crupier.png}{\crupier}{0.5}
\imagen{FSM/FSM_Jugador_i.png}{ \tirador }{0.5}
\imagen{FSM/FSM_Dados.png}{\dados  }{0.5}

\imagenvertical{FSM/FSM_Jugador_i_haciendo_apuesta_de_Campo.png}{ \campo}{0.5}
\imagenvertical{FSM/FSM_Jugador_i_haciendo_apuesta_de_sitio.png}{ \sitio }{0.5}
\imagenvertical{FSM/FSM_Jugador_i_haciendo_apuesta_linea_se_pase_Barra_No_Pase.png}{\pase }{0.4}
\imagenvertical{FSM/FSM_Jugador_i_haciendo_apuesta_venir_novenir.png}{\venir }{0.5}



La FSM del juego de craps se logra componiendo en paralelo:
\crupier $||$ \tirador $||$ \campo $||$ \sitio $||$ \pase $||$ \venir \\ con i de 1 a n

