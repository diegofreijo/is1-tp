\subsubsection{Actores}
Se modelaron cuatro actores en los casos de uso del sistema

\begin{description}
\item[Invitado]: un posible futuro jugador al cual se le permite acceder al casino con el objetivo de observar las mesas donde se desarrollan los juegos.
\item[Jugador]: aquel que aparece en la lista de jugadores ingresada al sistema. Tiene la posibilidad de jugar a cualquier juego as'i como cierto saldo de donde se debitan las apuestas y acreditan las ganancias. Si lo desea, tambi'en puede observar las mesas de juego tal como lo har'ia un invitado.
\item[Administrador]: un representante del equipo de los SOCIOS. Es aquel que realiza tareas administrativas en el sistema.
\item[Jefe de contabilidad]: es aquel capaz de realizar las tareas de Administrador y a su vez ``Dirigir el azar''.
\end{description}

\imagen{CU_Actores.png}{Diagrama de actores}{0.7}

\clearpage
\subsubsection{Diagrama de casos de uso}

\imagen{CU_Diagrama.png}{Diagrama de casos de uso}{0.6}



\subsubsection{Descripci'on de los casos de uso}

% --------------- Mirando mesa ---------------------
\op{1. El sistema le consulta al invitado si desea mirar una mesa de craps o de tragamonedas.}{}
\op{2. El invitado elige el tipo de juego que desea mirar.}{}
\op{3. El sistema le muestra al invitado las mesas abiertas del juego elegido.}{}
\op{4. El invitado elige la mesa que desea mirar.}{}
\op{5. El sistema le muestra al invitado la mesa elegida.}{}
\op{6. Fin CU.}{}
\cu{Mirando mesa}{Invitado}{-}{El invitado no est'a mirando ninguna mesa.}{El invitado est'a mirando una mesa.}{}{\rrefDeseable{req:inv_mirar_y_salir_mesas}}




% --------------- Saliendo de mesa ---------------------
\op{1. El invitado le informa al sistema que no desea mirar m'as la mesa que esta mirando.}{}
\op{2. El sistema le deja de mostrar la mesa.}{}
\op{3. Fin CU.}{}
\cu{Saliendo de mesa}{Invitado}{-}{El invitado est'a mirando una mesa.}{El invitado no est'a mirando ninguna mesa.}{}{\rrefEsencial{req:creacion_ingreso_y_salida_de_mesa}}




% --------------- Creando mesa ---------------------
\op{1. Si el juego elegido es tragamonedas, el sistema le pide al jugador el valor de la ficha. Si en cambio el juego elegido es craps, IR A 3.}{}
\op{2. El sistema valida el valor de ficha ingresado.}{}
\op{3. El sistema crea la mesa requerida del juego elegido.}{3.1 El sistema le advierte al jugador que el valor de ficha es inv'alido. IR A 1.}{}
\op{4. Fin CU.}{}
\cu{Creando mesa}{Jugador}{-}{El casino est'a abierto.}{Se crea la mesa requerida y el jugador tiene como 'unica mesa elegida a la reci'en creada.}{}{\rrefEsencial{req:creacion_ingreso_y_salida_de_mesa}}




% --------------- Eligiendo mesa ---------------------
\op{1. El sistema le muestra al jugador las mesas abiertas.}{}
\op{2. El jugador elige una mesa.}{}
\op{3. Fin CU.}{}
\cu{Eligiendo mesa}{Jugador}{-}{El casino est'a abierto.}{El jugador tiene como 'unica mesa elegida para jugar a la mesa seleccionada.}{}{\rrefEsencial{req:creacion_ingreso_y_salida_de_mesa}}




% --------------- Jugando a craps ---------------------
\op{1. El sistema le pregunta al jugador qu'e apuesta desea realizar y el monto de la misma.}{}
\op{2. El jugador informa sus decisiones.}{}
\op{3. Si al jugador le tocaba tirar los dados, el sistema le da la posibilidad de hacerlo. Si no, IR A 5.}{}
\op{4. El jugador tira los dados.}{}
\op{5. El sistema muestra el n'umero obtenido, debita del saldo del jugador sus apuestas perdidas y le acredita las apuestas ganadas.}{}
\op{6. Si la ronda no finaliz'o y el jugador es el siguiente en tirar, o si el jugador realiz'o una apuesta venir o no venir, IR A PASO 1.}{}
\op{7. Fin CU.}{}
\cu{Jugando a Craps}{Jugador}{-}{El jugador eligi'o una mesa de craps para jugar.}{-}{En cualquier momento, el jugador podr'a chatear con los dem'as jugadores en la misma mesa. Tambi'en podr'a cancelar su apuesta (ver FSM Craps secci'on {\ref{FSM:Craps}}).}{\rrefEsencial{req:apuesta_a_juegos}}



% --------------- Jugando a tragamonedas ---------------------
\op{1. El sistema le pregunta al jugador si quiere jugar 1, 2 o 3 fichas.}{}
\op{2. El jugador elige cu'antas fichas quiere jugar y acciona los rodillos.}{}
\op{3. El sistema le muestra el resultado de la tirada, debita la apuesta de la cuenta del jugador de tragamonedas y acredita la ganancia correspondiente (de existir alguna)}{}
\op{4. Fin CU.}{}
\cu{Jugando a Tragamonedas}{Jugador}{-}{El jugador eligi'o una mesa de tragamonedas para jugar.}{-}{}{\rrefEsencial{req:apuesta_a_juegos}}



% --------------- Abriendo casino ---------------------
\op{1. El sistema carga la lista de clientes con sus saldos. Adem'as carga las siguientes configuraciones:
	\begin{itemize}
	\item probabilidad de ocurrencia de cereza, bar simple, doble, triple y dinosaurio en los rodillos de las m'aquinas tragamonedas.
	\item probabilidad asignada a cada n'umero del 2 al 12 en los dados de las mesas de craps.
	\item el porcentaje que se le descuenta a cada jugador del monto ganado en una jugada \italica{todos ponen}.
	\item el porcentaje que se le descuenta a cada jugador en cada apuesta para engordar el pozo progresivo.
	\item valores de fichas admitidos para realizar apuestas y configurar nuevas mesas de tragamonedas.
	\item la cantidad de jugadas que debe un jugador apostar la apuesta m'axima para que tenga posibilidad de sacar el \italica{gordo progresivo}.
	\item probabilidad de ocurrencia de una jugada \italica{feliz} y de una jugada \italica{todos ponen}.
	\item monto m'inimo del pozo progresivo y feliz.
	\end{itemize}
}{}
\op{2. El sistema abre el casino y se lo informa al administrador.}{2.1 Si alguna configuraci'on fue incorrecta, ir a PASO 3.}
\op{3. Fin CU.}{}
\cu{Abriendo casino}{Administrador}{-}{El casino est'a cerrado.}{El casino est'a abierto.}{La configuraci'on puede ser brindada por el Administrador al momento de la apertura del casino o bien el sistema puede obtenerla desde alg'un medio ya predefinido (un archivo o una base de datos por ejemplo).}{\rrefEsencial{req:abrir_y_cerrar_casino}, \rrefEsencial{req:carga_clientes}, \rrefImportante{req:conf_juegos_fichas_jugadas_y_pozos}}




% --------------- Cerrando casino ---------------------
\op{1. El sistema verifica que no hayan mesas abiertas.}{}
\op{2. El sistema persiste la informaci'on de los saldos de cada jugador y del casino, le informa al administrador que las acciones fueron realizadas y cierra el casino.}{2.1. Si hay mesas abiertas, el sistema le advierte al administrador que la operaci'on no puede ser realizada. Ir a PASO 3.}
\op{3. Fin CU.}{}
\cu{Cerrando casino}{Administrador}{-}{El casino est'a abierto.}{El casino est'a cerrado.}{}{\rrefEsencial{req:abrir_y_cerrar_casino}, \rrefEsencial{req:persistir_saldos_al_cierre}}




% --------------- Pidiendo reporte ---------------------
\op{1. El sistema le muestra al administrador los posibles reportes a pedir: Ranking de jugadores, Estado actual y Detalle movimientos por jugador.}{}
\op{2. El administrador elige el reporte que desea.}{}
\op{3. El sistema le mostrar'a al administrador, a partir de la informaci'on guardada en las jugadas sucesivas:
	\begin{itemize}
	\item si elige \negrita{Ranking de jugadores}, los jugadores que m'as dinero ganaron y perdieron en el d'ia.
	\item si elige \negrita{Estado actual}, el informe del estado actual del casino y los clientes, especialmente los saldos respectivos.
	\item si elige \negrita{Detalle de movimientos por jugador}, el detalle de todos los movimientos (apuestas, premios ganados, monto ganado) desde que ingresaron al casino.
	\end{itemize}
}{}
\op{4. Fin CU.}{}
\cu{Pidiendo reporte}{Administrador}{-}{-}{-}{}{\rrefImportante{req:reportes}}




% --------------- Dirigiendo el azar ---------------------
\op{1. El sistema le muestra al jefe de contabilidad las opciones de
	\begin{itemize}
	\item Activar modo dirigido de craps/tragamonedas, donde podr'a seleccionar una mesa y un resultado. 'Este saldr'a en todas las jugadas de esa mesa.
	\item Desactivar modo dirigido de craps/tragamonedas, donde podr'a seleccionar una mesa. 'Esta manejar'a los resultados azarosamente con las probabilidades especificadas al comienzo de la jornada.
	\item Activar \italica{todos ponen}/\italica{jugada feliz}, donde podr'a seleccionar una mesa. Saldr'a siempre 'ese tipo de jugada en todas las jugadas de esa mesa.
	\item Desactivar \italica{todos ponen}/\italica{jugada feliz}, donde podr'a seleccionar una mesa. La decisi'on acerca de si sale o no 'ese tipo de jugada en todas las jugadas de esa mesa quedar'a a cargo del azar en base a las probabilidades especificadas al comienzo de la jornada.
	\item Anular \italica{todos ponen}/\italica{jugada feliz}, donde podr'a seleccionar una mesa. Nunca saldr'a 'ese tipo de jugada en todas las jugadas de esa mesa.
	\end{itemize}
}{}
\op{2. El jefe de contabilidad elige la opci'on deseada e ingresa los par'ametros necesarios.}{}
\op{3. El sistema efect'ua los cambios correspondientes sobre las mesas especificadas a partir de la jugada siguiente a la que estaba en curso.}{}
\op{4. Fin CU.}{}
\cu{Dirigiendo el azar}{Jefe de contabilidad}{-}{-}{La/s mesa/s seleccionadas por el jefe de contabilidad se comportar'an de la forma se'nalada por 'este.}{}{\rrefImportante{req:modo_dirigido}}
