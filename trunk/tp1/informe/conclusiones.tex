\subsection{Comentarios al corrector}

Como conclusi'on podemos decir que vimos la utilidad de 
iterar, sobre el documento, como ejemplo podemos citar
el caso de uso ``Dejando de jugar en mesa'', caso de uso 
que incialmente nos parecia innecesario con el correr
de la iteraciones pudimos convencernos de la necesidad de su existencia
tanto para cumplir con el requerimiento () como para 
que el jugardor pue


% \subsubsection{La ficci'on supera a la realidad}
% Docente de IS1, en la seccion de requerimientos del sistema nos comprometimos, entre otros, a que nuestro sistema sea escalable y seguro. Lo hicimos porque son requerimientos importantes y como tales deberemos cumplir para que los stakeholders est'en satisfechos con la soluci'on que les ofreceremos. Pero en el contexto del TP, no obtuvimos ninguna m'etrica acerca de que tan escalable y qu'e tan seguro deber'ia ser (a'un as'i, no son variables que deber'ian obviarse por las caracter'isticas del producto). Es por eso que en realidad, en el TP, no nos comprometemos a cumplir estos requerimientos. 
% 
% De forma similar ocurre con el requerimiento de ser r'apido. En realidad nos comprometemos a qu'e el sistema sea ``utilizable'' pero no ofrecemos ninguna garant'ia real, principalmente por que no se nos ofrecieron m'etricas a cumplir.
% 
% 
