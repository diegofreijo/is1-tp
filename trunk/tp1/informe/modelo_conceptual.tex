\subsubsection{Diagrama de clases conceptuales}
Para comprender mejor las relaciones estructurales entre las entidades del sistema, aqu'i se muestra el modelo conceptual del mismo. 

\imagenvertical{MC_Diagrama.png}{Diagrama de clases conceptuales}{0.5}


\subsubsection{Detalles de las clases}
A continuaci'on se comenta el significado y raz'on de ser de las clases principales en el diagrama
\begin{description}
\item [Jugada] una jugada realizada por alg'un jugador en alguna mesa en alg'un momento de la jornada del casino. Puede ser de tipo Tragamonedas o Craps; reflejando en el primer caso una apuesta de fichas y la posibilidad de haber girado los rodillos y en el segundo a las apuestas de los jugadores con la posibilidad de haber realizado un tiro (notar que una jugada de Craps \negrita{no} es una ronda sino un tiro de dados). Puede estar asociada a su resultado si es que ya se jug'o, o no tenerlo si es que todav'ia no se tiraron los dados en el caso de Craps o no se giraron los rodillos en el caso de Tragamonedas.
El atributo derivado \italica{pago} refleja el monto otorgado al jugador por parte del casino al resolverse la jugada, incluyendo todas las alteraciones por jugadas y pozos especiales de los distintos tipos de juegos.
Para poder obtener los reportes de:
\begin{itemize}
    \item \italica{Ranking de jugadores} 
    \item \italica{Detalle de movimientos por jugador} 
\end{itemize}
es necesario guardar el historial de jugadas realizadas por cada jugador en la jornada.

\item [Invitado] Es un visitante del casino. Puede mirar las mesas para ver jugar a sus jugadores.

\item [Jugador] Es aquel capaz de realizar jugadas en una mesa. Un jugador \italica{es} a la vez un invitado. Es decir, puede mirar una mesa sin necesidad de jugar.
\item [Jornada] Representa una jornada del casino, desde que abre hasta que cierra. El factorTodosPonen es el factor por el que se multiplican los premios de los jugadores en una jugada todosPonen para descontar dicha cantidad del premio y destinarla al pozo feliz.
\item [Mesa] Es en donde se realizan las jugadas. Puede ser de Tragamonedas o Craps y en una jornada puede recibir cualquier cantidad de jugadas.
\item [Pozo Progresivo] Modela el pozo progresivo de las m'aquinas tragamonedas. Su asociaci'on con JugadaTragamonedas permite conocer, por medio del atributo monto, el valor del pozo al momento de realizarse la jugada (para las jugadas resueltas indica cu'al era el valor del pozo un instante antes de la resoluci'on de la jugada).
\item [Pozo Feliz] Modela el pozo feliz del casino. Su asociaci'on con Jugada permite conocer, por medio del atributo monto, el valor del pozo al momento de realizarse la jugada (al igual que el caso anterior, un instante antes de la resoluci'on de la jugada).

\end{description}


\subsubsection{Restricciones OCL al modelo conceptual}
\begin{itemize}



\item \textit{Solo hay una jugada feliz como m'aximo}

\textbf{context} Jugada \\ \textbf{inv:}
    Jugada.allInstances()$\rightarrow$select(j $|$ j.tipo = TipoJugada::feliz \textbf{and} j.result'oEn$\rightarrow$isEmpty())$\rightarrow$size() $\leq$ 1)



\item \textit{Para todas las jugadas no resueltas el pozo feliz tiene el mismo monto}

\textbf{context} Jugada \\ \textbf{inv:}
    Jugada.allInstances()$\rightarrow$select(j $|$ j.result'oEn$\rightarrow$isEmpty())$\rightarrow$forAll($j_{1}$, $j_{2}$ : Jugada $|$ $j_{1}$.pozoFeliz.monto\\ = $j_{2}$.pozoFeliz.monto)



\item \textit{Para cada mesa de Craps hay exactamente un tirador}

% self.jugadas->select(j | j.result'oEn->isEmpty() and j.oclAsType(JugadaCraps).tirador = true)->size() = 1)
\textbf{context}  MesaCraps \\ \textbf{inv:} 
    self.jugadas$\rightarrow$select(j $|$ j.result'oEn$\rightarrow$isEmpty() \textbf{and} j.oclAsType(JugadaCraps).tirador = true)$\rightarrow$size() = 1)



\item \textit{En las mesas de Craps s'olo se realizan apuestas de tipo Crap} 

% self.jugadas->forAll(j | j.oclIsTypeOf(JugadaCraps))
\textbf{context}  MesaCraps \\ \textbf{inv:} 
    self.jugadas$\rightarrow$forAll(j $|$ j.oclIsTypeOf(JugadaCraps))



\item \textit{En las mesas de Tragamonedas s'olo se realizan apuestas de tipo Tragamoneda}

% self.jugadas->forAll(j | j.oclIsTypeOf(JugadaTragamonedas))
\textbf{context}  MesaTragamonedas \\ \textbf{inv:} 
    self.jugadas$\rightarrow$forAll(j $|$ j.oclIsTypeOf(JugadaTragamonedas))



\item \textit{Los jugadores comunes no tienen saldo negativo}

% self.saldo >= 0
\textbf{context} JugadorNormal \\ \textbf{inv:}
    self.saldo $\geq$ 0
  
  

\item \textit{No hay jugadores con el mismo dni}

% Jugador.allInstances()->forAll(j1, j2 : Jugador | j1 != j2 ---> j1.dni != j2.dni)
\textbf{context}  Jugador \\ \textbf{inv:}
    Jugador.allInstances()$\rightarrow$forAll($j_{1}$, $j_{2}$ : Jugador $|$ $j_{1} < > j_{2}$ \textbf{implies} $j_{1}.dni < > j_{2}.dni$)



\item \textit{El tipo de jugada (normal, feliz o todosPonen) tiene que ser el mismo para las jugadas de una misma mesa}

% self.jugadas->select(j | j.result'oEn->isEmpty())->forAll(j1, j2 : Jugada | j1.tipo = j2.tipo)
\textbf{context} Mesa \\ \textbf{inv:}
    self.jugadas$\rightarrow$select(j $|$ j.result'oEn$\rightarrow$isEmpty())$\rightarrow$forAll($j_{1}$, $j_{2}$ : Jugada $|$ $j_{1}$.tipo = $j_{2}$.tipo)
  
  

\item \textit{El resultado de una jugada de Craps debe ser del mismo tipo que la jugada}

% self.result'oEn->notEmpty ---> self.result'oEn.oclIsTypeOf(ResultadoCraps)
\textbf{context} JugadaCraps \\ \textbf{inv:}
    self.result'oEn$\rightarrow$notEmpty() \textbf{implies} self.result'oEn.oclIsTypeOf(ResultadoCraps)



\item \textit{El resultado de una jugada de Tragamonedas debe ser del mismo tipo que la jugada}

% self.result'oEn->notEmpty ---> self.result'oEn.oclIsTypeOf(ResultadoTragamonedas)
\textbf{context} JugadaTragamonedas \\ \textbf{inv:}
    self.result'oEn$\rightarrow$notEmpty() \textbf{implies} self.result'oEn.oclIsTypeOf(ResultadoTragamonedas)



\item \textit{El n'umero salido en un resultado de craps debe ser entre 2 y 12}

% 2 <= self.n'umeroSalido <= 12
\textbf{context} ResultadoCraps \\ \textbf{inv:}
    2 $\leq$ self.n'umeroSalido $\leq$ 12



\item \textit{Definici'on del atributo derivado Jugada::pago para una JugadaTragamonedas}

\textbf{context} JugadaTragamonedas::pago : Cr'edito \\ \textbf{derive:}
    \textbf{let} result : ResultadoTragamonedas = self.result'oEn.oclAsType(ResultadoTragamonedas)

    \textbf{let} esBar(figura : FiguraRodillo) : Boolean = figura = FiguraRodillo::barSimple \textbf{or} figura = FiguraRodillo::barDoble \textbf{or} figura = FiguraRodillo::barTriple

    \textbf{let} correspondePremioDinosaurio : Boolean = result.rodillo1 = FiguraRodillo::dinosaurio and result.rodillo2 = FiguraRodillo::dinosaurio and result.rodillo3 = FiguraRodillo::dinosaurio
    
    \textbf{let} correspondePremioTresCerezas : Boolean = result.rodillo1 = FiguraRodillo::cereza and result.rodillo2 = FiguraRodillo::cereza and result.rodillo3 = FiguraRodillo::cereza
    
    \textbf{let} correspondePremioBarTriple : Boolean = result.rodillo1 = FiguraRodillo::barTriple and result.rodillo2 = FiguraRodillo::barTriple and result.rodillo3 = FiguraRodillo::barTriple
    
    \textbf{let} correspondePremioBarDoble : Boolean = result.rodillo1 = FiguraRodillo::barDoble and result.rodillo2 = FiguraRodillo::barDoble and result.rodillo3 = FiguraRodillo::barDoble
    
    \textbf{let} correspondePremioBarSimple : Boolean = result.rodillo1 = FiguraRodillo::barSimple and result.rodillo2 = FiguraRodillo::barSimple and result.rodillo3 = FiguraRodillo::barSimple
    
    \textbf{let} correspondePremioTresBares : Boolean = esBar(result.rodillo1) and esBar(result.rodillo2) and esBar(result.rodillo3)
    
    \textbf{let} correspondePremioDosCerezas : Boolean = Bag{}->including(result.rodillo1)$\rightarrow$including(result.rodillo2)$\rightarrow$including(result.rodillo3)$\rightarrow$count(FiguraRodillo::cereza) = 2
    
    \textbf{let} correspondePremioUnaCereza : Boolean = result.rodillo1 = FiguraRodillo::cereza \textbf{or} result.rodillo2 = FiguraRodillo::cereza \textbf{or} result.rodillo3 = FiguraRodillo::cereza

    \textbf{let} premioDinosaurio(fichas : FichasApostadas) : Cr'edito =\\
        \textbf{if} fichas = FichasApostadas::unaFicha \textbf{then} 1000\\
        \textbf{else} \textbf{if} fichas = FichasApostadas::dosFichas \textbf{then} 2000\\
        \textbf{else} \textbf{if} fichas = FichasApostadas::tresFichas \textbf{then} 5000\\
        \textbf{endif}

    \textbf{let} premioTresCerezas(fichas : FichasApostadas) : Cr'edito =\\
        \textbf{if} fichas = FichasApostadas::unaFicha \textbf{then} 160\\
        \textbf{else} \textbf{if} fichas = FichasApostadas::dosFichas \textbf{then} 320\\
        \textbf{else} \textbf{if} fichas = FichasApostadas::tresFichas \textbf{then} 480\\
        \textbf{endif}

    \textbf{let} premioBarTriple(fichas : FichasApostadas) : Cr'edito =\\
        \textbf{if} fichas = FichasApostadas::unaFicha \textbf{then} 80\\
        \textbf{else} \textbf{if} fichas = FichasApostadas::dosFichas \textbf{then} 160\\
        \textbf{else} \textbf{if} fichas = FichasApostadas::tresFichas \textbf{then} 240\\
        \textbf{endif}

    \textbf{let} premioBarDoble(fichas : FichasApostadas) : Cr'edito =\\
        \textbf{if} fichas = FichasApostadas::unaFicha \textbf{then} 40\\
        \textbf{else} \textbf{if} fichas = FichasApostadas::dosFichas \textbf{then} 80\\
        \textbf{else} \textbf{if} fichas = FichasApostadas::tresFichas \textbf{then} 120\\
        \textbf{endif}

    \textbf{let} premioBarSimple(fichas : FichasApostadas) : Cr'edito =\\
        \textbf{if} fichas = FichasApostadas::unaFicha \textbf{then} 20\\
        \textbf{else} \textbf{if} fichas = FichasApostadas::dosFichas \textbf{then} 40\\
        \textbf{else} \textbf{if} fichas = FichasApostadas::tresFichas \textbf{then} 60\\
        \textbf{endif}

    \textbf{let} premioTresBares(fichas : FichasApostadas) : Cr'edito =\\
        \textbf{if} fichas = FichasApostadas::unaFicha \textbf{then} 10\\
        \textbf{else} \textbf{if} fichas = FichasApostadas::dosFichas \textbf{then} 20\\
        \textbf{else} \textbf{if} fichas = FichasApostadas::tresFichas \textbf{then} 30\\
        \textbf{endif}

    \textbf{let} premioDosCerezas(fichas : FichasApostadas) : Cr'edito =\\
        \textbf{if} fichas = FichasApostadas::unaFicha \textbf{then} 5\\
        \textbf{else} \textbf{if} fichas = FichasApostadas::dosFichas \textbf{then} 10\\
        \textbf{else} \textbf{if} fichas = FichasApostadas::tresFichas \textbf{then} 15\\
        \textbf{endif}

    \textbf{let} premioUnaCereza(fichas : FichasApostadas) : Cr'edito =\\
        \textbf{if} fichas = FichasApostadas::unaFicha \textbf{then} 2\\
        \textbf{else} \textbf{if} fichas = FichasApostadas::dosFichas \textbf{then} 4\\
        \textbf{else} \textbf{if} fichas = FichasApostadas::tresFichas \textbf{then} 6\\
        \textbf{endif}

    \textbf{if} self.result'oEn$\rightarrow$isEmpty() \textbf{then}\\
        0\\
    \textbf{else}\\
        \textbf{if} correspondePremioDinosaurio \textbf{then} premioDinosaurio(self.apuesta)\\
        \textbf{else} \textbf{if} correspondePremioTresCerezas \textbf{then} premioTresCerezas(self.apuesta)\\
        \textbf{else} \textbf{if} correspondePremioBarTriple \textbf{then} premioBarTriple(self.apuesta)\\
        \textbf{else} \textbf{if} correspondePremioBarDoble \textbf{then} premioBarDoble(self.apuesta)\\
        \textbf{else} \textbf{if} correspondePremioBarSimple \textbf{then} premioBarSimple(self.apuesta)\\
        \textbf{else} \textbf{if} correspondePremioTresBares \textbf{then} premioTresBares(self.apuesta)\\
        \textbf{else} \textbf{if} correspondePremioDosCerezas \textbf{then} premioDosCerezas(self.apuesta)\\
        \textbf{else} \textbf{if} correspondePremioUnaCereza \textbf{then} premioUnaCereza(self.apuesta)\\
        \textbf{else} 0\\
        \textbf{endif}\\
    \textbf{endif}

\clearpage

\item \textit{Definici'on del atributo derivado Jugada::pago para una JugadaCraps}

\textbf{context} JugadaCraps::pago : Cr'edito \\ \textbf{derive:}

Describimos como se modifica el saldo de un jugador seg'un que apuesta haya ganado. 
(cuando el jugador apuesta se le descuenta de saldo el importe)

\begin{verbatim}
  [Linea de pase]
      jugador.saldo += apuestaCraps.valor * 2  (1 a 1)

   [Barra no pase]
      jugador.saldo += apuestaCraps.valor * 2  (1 a 1)

   [Venir]
      jugador.saldo += apuestaCraps.valor * 2  (1 a 1)

   [No venir] 
      jugador.saldo += apuestaCraps.valor * 2  (1 a 1)

   [De Campo]
      if (resultoEn.numeroSalido = 4, 9, 10 u 11) 
         jugador.saldo += apuestaCraps.valor * 2  (1 a 1)
      elseif (resultoEn.numeroSalido = 2 o 12)}
         jugador.saldo += apuestaCraps.valor * 3  (2 a 1)
          endif
      endif

   [De Sitio] (gana si sale el numero del la primer columna, 
    para la segunda columna y devuelve la apuesta realizada)
      a ganar 
              4      5:11  
              5      5:8  
              6      4:5  
              8      4:5  
              9      5:8  
              10     5:11 


      [a perder]

            4        9:5  
            5        7:5  
            6        7:6  
            8        7:6  
            9        7:5  
            10       9:5  
      
\end{verbatim}


\clearpage


\item\textit{ En una jugada Crap las apuestas en sitio a ganar de un jugador son a distintos n'umeros}

\textbf{context}  JugadaCrap \\ \textbf{inv:} 
  self.ApuestaEnSitioAGanar$\rightarrow$forAll( $a_{1}, a_{2}$ : ApuestaEnSitioAGanar $ | $ $ a_{1} <> a_{2} $  \textbf{implies} $ a_{1}.valor <> a_{2}.valor $ )


\item\textit{ En una jugada Crap las apuestas en sitio a perder de un jugador son a distintos n'umeros}

\textbf{context}  JugadaCrap \\ \textbf{inv:} 
  self.ApuestaEnSitioAPerder$\rightarrow$forAll($a_{1}$, $a_{2}$ : ApuestaEnSitioAPerder  $ | $ $a_{1} <> a_{2} $ \textbf{implies} $a_{1}$.valor $<>$ $a_{2}$.valor)


\item\textit{El factor todos ponen de una jornada va de 0 a 1}

\item\textit{Operaci'on reporte \italica{Ranking de jugadores}: dada una jornada puede obtenerse el conjunto de jugadores y todas las jugadas que los mismos realizaron durante la misma, pudi'endose obtener los que m'as plata ganaron y perdieron }

\item\textit{Operaci'on reporte \italica{Estado actual}: puede deducirse de observar el saldo de una jornada (saldo del casino) y el de todos los jugadores asociados a 'esta }

\item\textit{Operaci'on reporte \italica{Detalle movimientos por jugador}: al haber modelado todas las jugadas asociadas a los jugadores puede obtenerse el listado de las misma y verificar los premios involucrados }

\end{itemize}
