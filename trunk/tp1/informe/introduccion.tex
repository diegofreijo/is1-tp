\subsection{ Objetivo del documento }

El presente documento ser'a una herramienta en el proceso 
de desarrollo de la informatizaci'on del ``Casino On line''. En el mismo 
vamos a especificar el funcionamiento, operatoria, responsabilidades y alcance
del mismo.


% \subsection{ Convenciones de notaci'on	}
% TODO: SI NO HAY CONVENCIONES SACAR ESTA SECCION

\subsection{ Destinatarios del documento	}
Los destinatarios del documento son los SOCIOS. En particular:

\begin{itemize}
    \item Armando Paredes (Jefe)
    \item Lic. Galinardi (Marketing)
    \item Claudio Gallo (Contador)
    \item Dr. Foronga (Contador y mano derecha del Jefe)
\end{itemize}


\subsection{ Descripci'on del problema }
El problema consiste en especificar un sistema que describa el funcionamiento
de un casino online, el cual constar'a inicialmente de 2 tipos de juegos: ``Craps'' y ``M'aquina Tragamonedas''. Habr'a distintos tipos de jugadas y pozos especiales. Existir'an clientes V.I.P. quienes podr'an apostar todo lo que quieran (su saldo podr'a ser negativo).

 Podr'an ``visitar'' el casino personas que no posean una cuenta de usuario, a las que denominaremos ``invitados'', quienes podr'an mirar cualquier juego pero no podr'an realizar apuestas. Los invitados y el p'ublico en general podr'an darse de alta, para lo cual deber'an realizar los tr'amites necesarios con la secretaria. 
Al igual que en un casino convencional los jugadores podr'an cambiar sus fichas o comprar más, para lo cual tambi'en realizar'an la gesti'on por fuera del aplicativo, con la misma.
Los jugadores podr'an jugar a ambos juegos. Podr'a haber cualquier cantidad de mesas de ambos
juegos.


\subsection{ Documentos relacionados}
El presente informe se basa en las minutas de las reuniones con clientes del casino y los varios e-mails enviados a la lista \textit{isoft1-alu@googlegroups.com} hasta el d'ia de la fecha.


\subsection{ Organizaci'on del informe	}
Se espera que el documento pueda ser le'ido por cualquier individuo con conocimientos t'ecnicos m'inimos con excepci'on de algunos diagramas correspondientes a la secci'on de requerimientos espec'ificos. 



