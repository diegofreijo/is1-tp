
\newcommand{\ronda}{ \italica{ FSM Ronda} }
\newcommand{\crupier}{ \italica{ FSM Crupier} }
\newcommand{\tirador}{ \italica{ FSM Tirador i} }
\newcommand{\dados}{ \italica{ FSM Dados i} }
\newcommand{\campo}{\italica{FSM Jugador i haciendo apuesta de Campo }}
\newcommand{\sitio}{\italica{FSM Jugador i haciendo apuesta de sitio}}
\newcommand{\pase}{\italica{FSM Jugador i haciendo apuesta linea de pase / Barra NoPase}}
\newcommand{\venir}{\italica{FSM Jugador i haciendo apuesta venir novenir}}

Tanto con para Craps como para el Tragamonedas, modelares como los jugadores interactuan con el sistema.

\subsubsection{Tragamonedas}

En la figura \ref{fig_FSM/FSM_Jugador_tragamonedas_i.png}veremos la din'amica y qu'e opciones tiene un jugador en el juego de tragamonedas.
Al jugador $i$ se le asigna la mesa $i$. Esto se debe a que no es posible elegir jugar en una mesa ya existente, pues 'esta est'a ocupada. Adem'as al abandonarla la misma ser'a cerrada, lo que imposibilita el poder elegirla.

\imagen{FSM/FSM_Jugador_tragamonedas_i.png}{FSM Jugador tragamonedas i}{0.5}
\clearpage


\subsubsection{Craps}
\label{FSM:Craps}
Para modelar el juego de craps usaremos varias m'aquinas, en particular:
el tirador, el crupier (sistema) y los jugadores haciendo los varios tipos de apuestas
posibles. 

En la seccion \ref{etiquetasFSMs} del Glosario se encuentran las definiciones de cada etiqueda usada.

 
\begin{center}
\begin{tabular}{p{4cm}|p{12cm}}        
         \multicolumn{2}{c}{M'AQUINAS USADAS PARA MODELAR EL CRAPS}     \\
        \hline
        \ronda & Modelamos los estados de una ronda de Craps \\
        \hline
        \crupier & Esta ser'a una vista de una de las funcionalidades provista por el sistema, aqu'i se pod'a ver parte de la interacci'on de los jugadores con el sistema. \\
         \hline 
         \tirador  & Modela un jugador en particular en su rol de tirador. Usamos una variable temporal para mostrar que un tirador no puede quedarse con los dados indefinidamente. Las constantes $k1$ y $k2$ son constantes de tiempo a definir en la etapa de dise\~{n}o.\\
        \hline 
         & Modela un jugador en particular haciendo/cancelando una o varias apuestas que duran un tiro. \\
        \hline  
        & \italica{'idem} para apuestas que duran una ronda. Analizando la traza, el jugador podr'ia apostar en una ronda y en la siguiente ronda cancelar la apuesta, 'esta es una limitacion del modelo. Entendemos que una apuesta una vez que se tiraron los dados no se puede retirar ni modificar. Lo que podr'a cancelar ser'an las n apuestas anteriores antes de que se tiren los dados, una vez tirados estos no se podr'a cancelar. \\
        \hline 
        & \italica{'idem} para apuestas para m'as de una ronda. \\

\end{tabular}
\end{center}

 

\imagen{FSM/FSM_Ronda.png}{\ronda}{0.5}
\imagen{FSM/FSM_Crupier.png}{\crupier}{0.5}
\imagen{FSM/FSM_Jugador_i.png}{ \tirador }{0.5}
\imagen{FSM/FSM_Dados.png}{\dados  }{0.5}

\imagenvertical{FSM/FSM_Jugador_i_haciendo_apuesta_de_Campo.png}{ \campo}{0.5}
\imagenvertical{FSM/FSM_Jugador_i_haciendo_apuesta_de_sitio.png}{ \sitio }{0.5}
\imagenvertical{FSM/FSM_Jugador_i_haciendo_apuesta_linea_se_pase_Barra_No_Pase.png}{\pase }{0.5}
\imagenvertical{FSM/FSM_Jugador_i_haciendo_apuesta_venir_novenir.png}{\venir }{0.5}



La FSM del juego de craps se logra componiendo en paralelo:


