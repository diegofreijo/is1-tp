\subsection{Comentarios al corrector}
\subsubsection{La ficci'on supera a la realidad}
Docente de IS1, en la seccion de requerimientos del sistema nos comprometimos, entre otros, a que nuestro sistema sea escalable y seguro. Lo hicimos porque son requerimientos importantes y como tales deberemos cumplir para que los stakeholders est'en satisfechos con la soluci'on que les ofreceremos. Pero en el contexto del TP, no obtuvimos ninguna m'etrica acerca de que tan escalable y que tan seguro deber'ia ser (a'un as'i, no son variables que deber'ian obviarse por las caracter'isticas del producto). Es por eso que en realidad, en el TP, no nos comprometemos a cumplir estos requerimientos. 

De forma similar ocurre con el requerimiento de ser r'apido. En realidad nos compromentemos a que el sistema sea ''utilizable'' pero no ofrecemos ninguna garant'ia real, principalmente por que no se nos ofrecieron m'etricas a cumplir.

El dise'no no ser'a inclu'ido porque ningun integrante de nuestro grupo se destaca por poseer cualidades art'isticas...





\subsection{Decisiones tomadas por el grupo}
\begin{itemize}
%\item En CU ''Cerrando casino'' se decidi'o modelar el hecho que no se puede cerrar el casino como una precondici'on en lugar de un condicional dentro del CU para describir mejor la funcionalidad en lugar de los detalles de implementaci'on.
\item 
\item 

\end{itemize}
