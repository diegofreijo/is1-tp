\subsubsection{Actores}
Se modelaron cuatro actores en los casos de uso del sistema

\begin{description}
\item[Invitado]: un posible futuro jugador al cual se le permite acceder al casino con el objetivo de observar las mesas donde se desarrollan los juegos.
\item[Jugador]: aquel que aparece en la lista de jugadores ingresada al sistema. Tiene la posibilidad de jugar a cualquier juego as'i como cierto saldo de donde se debit'an las apuestas y acreditan las ganancias.
\item[Administrador]: un representante del equipo de SOCIOS. Es aquel que realiza tareas administrativas en el sistema.
\item[Jefe de contabilidad]: es aquel capaz de realizar las tareas de Administrador y a su vez ``Dirigir el azar''.
\end{description}

\imagen{CU_Actores.png}{Diagrama de herencia de los actores}{0.6}


\subsubsection{Diagrama}
\imagen{CU_CasosDeUso.png}{Diagrama de casos de uso}{0.6}


\subsubsection{Descripci'on de cada caso de uso}

% --------------- Jugando en una mesa ---------------------
\subsubsubsection{CU: Jugando en una mesa}
\negrita{Pre condici'on}: - \newline
\indent\negrita{Post condici'on}: -

\negrita{Actor primario}: Jugador \newline
\indent\negrita{Actores secundarios}: -

\negrita{Desarrollo normal}
\begin{enumerate}
\item El sistema le pregunta al jugador si quiere jugar a tragamonedas o a craps.
\item El jugador elige el tipo de juego al que desea jugar.
\item Si elige jugar al tragamonedas, EXTIENDE ``Creando mesa'' e IR A PASO \ref{label_jugar}. Si no, el sistema le pregunta si desea unirse a una mesa existente o crear una nueva.
\item Si elige unirse a una mesa, EXTIENDE ``Eligiendo mesa''. Si en cambio elige crear una mesa, EXTIENDE ``Creando mesa''.
\item Si eligi'o jugar a Craps, EXTIENDE ``Jugando a Craps''. Si en cambio eligi'o jugar a Tragamonedas, EXTIENDE ``Jugando a Tragamonedas''.\label{label_jugar}
\item Fin CU.
\end{enumerate}



% --------------- Eligiendo mesa ---------------------
\subsubsubsection{CU: Eligiendo mesa}
\negrita{Pre condici'on}: - \newline
\indent\negrita{Post condici'on}: -

\negrita{Actor primario}: Eligidor de mesa \newline
\indent\negrita{Actores secundarios}: -

\negrita{Desarrollo normal}
\begin{enumerate}
\item El sistema le muestra al eligidor de mesa las mesas de craps abiertas.
\item El eligidor de mesa elige una mesa.
\item Fin CU.
\end{enumerate}



% --------------- Creando mesa ---------------------
\subsubsubsection{CU: Creando mesa}
\negrita{Pre condici'on}: - \newline
\indent\negrita{Post condici'on}: Se crea la mesa requerida.

\negrita{Actor primario}: Creador de mesa \newline
\indent\negrita{Actores secundarios}: -

\negrita{Desarrollo normal}
\begin{enumerate}
\item Si el juego elegido es tragamonedas, el sistema le pide al creador de mesa el valor de la ficha. Si en cambio el juego elegido es craps, IR A \ref{label_val_ficha_invalido}. \label{cu_pedir_valor}
\item El sistema valida el valor de ficha ingresado.
\item El sistema crea la mesa requerida del juego elegido. \label{label_val_ficha_invalido}
\item Fin CU.
\end{enumerate}

\negrita{Desarrollo alternativo}

\ref{label_val_ficha_invalido}.1 El sistema le advierte al creador de mesa que el valor de ficha es invalido. IR A \ref{cu_pedir_valor}.





% --------------- Jugando a craps ---------------------
\subsubsubsection{CU: Jugando a craps}
\negrita{Pre condici'on}: - \newline
\indent\negrita{Post condici'on}: -

\negrita{Actor primario}: Jugador de craps \newline
\indent\negrita{Actores secundarios}: -

\negrita{Desarrollo normal}
\begin{enumerate}
\item El sistema le pregunta al jugador de craps si desea apostar, cuantas fichas y de qu'e tipo. \label{cu_jugar_craps}
\item El jugador de craps informa sus decisiones.
\item Al jugador de craps que le toque tirar, tira.
\item El sistema muestra el n'umero salido; debita del saldo del jugador de craps sus apuestas perdidas y le acredita las apuestas ganadas.
\item El sistema guarda la informaci'on de la jugada para futuras referencias.
\item Si la ronda no finaliz'o y el jugador de craps es el siguiente en tirar, o si el jugador realiz'o una apuesta venir o no venir, o si desea seguir jugando, IR A PASO \ref{cu_jugar_craps}.
\item Fin CU.
\end{enumerate}



% --------------- Jugando a tragamonedas ---------------------
\subsubsubsection{CU: Jugando a tragamonedas}
\negrita{Pre condici'on}: - \newline
\indent\negrita{Post condici'on}: -

\negrita{Actor primario}: Jugador de tragamonedas \newline
\indent\negrita{Actores secundarios}: -

\negrita{Desarrollo normal}
\begin{enumerate}
\item El sistema le pregunta al jugador de tragamonedas si quiere jugar 1, 2 o 3 fichas. \label{cu_jugar_traga}
\item El jugador elige cuantas fichas quiere jugar y acciona los rodillos.
\item El sistema le muestra el resultado de la tirada, debita la apuesta de la cuenta del jugador de tragamonedas y acredita la ganancia correspondiente.
\item El sistema guarda la informaci'on de la jugada para futuras referencias.
\item Si el jugador de tragamonedas desea seguir jugando, IR A PASO \ref{cu_jugar_traga}.
\item Fin CU.
\end{enumerate}




% --------------- Mirando mesa ---------------------
\subsubsubsection{CU: Mirando mesa}
\negrita{Pre condici'on}: El invitado no esta mirando ninguna mesa. \newline
\indent\negrita{Post condici'on}: El invitado esta mirando una mesa.

\negrita{Actor primario}: Invitado \newline
\indent\negrita{Actores secundarios}: -

\negrita{Desarrollo normal}
\begin{enumerate}
\item El sistema le consulta al invitado si desea mirar una mesa de craps o de tragamonedas.
\item El invitado elige el tipo de juego que desea mirar.
\item El sistema le muestra al invitado las mesas abiertas del juego elegido.
\item El invitado elige la mesa que desea mirar.
\item El sistema le muestra al invitado la mesa elegida.
\item Fin CU.
\end{enumerate}




% --------------- Saliendo de mesa ---------------------
\subsubsubsection{CU: Saliendo de mesa}
\negrita{Pre condici'on}: El invitado esta mirando una mesa. \newline
\indent\negrita{Post condici'on}: El invitado no esta mirando ninguna mesa.

\negrita{Actor primario}: Invitado \newline
\indent\negrita{Actores secundarios}: -

\negrita{Desarrollo normal}
\begin{enumerate}
\item El jugador le informa al sistema que no desea mirar mas la mesa que esta mirando.
\item El sistema le deja de mostrar la mesa.
\item Fin CU.
\end{enumerate}




% --------------- Abriendo casino ---------------------
\subsubsubsection{CU: Abriendo casino}
\negrita{Pre condici'on}: El casino est'a cerrado. \newline
\indent\negrita{Post condici'on}: El casino est'a abierto.

\negrita{Actor primario}: Administrador \newline
\indent\negrita{Actores secundarios}: -

\negrita{Desarrollo normal}
\begin{enumerate}
\item El sistema carga la lista de clientes con sus saldos y las configuraciones de craps, tragamonedas y pozo feliz.
\item El sistema abre el casino.
\item Fin CU.
\end{enumerate}





% --------------- Cerrando casino ---------------------
\subsubsubsection{CU: Cerrando casino}
\negrita{Pre condici'on}: El casino est'a abierto. \newline
\indent\negrita{Post condici'on}: El casino est'a cerrado.

\negrita{Actor primario}: Administrador \newline
\indent\negrita{Actores secundarios}: -

\negrita{Desarrollo normal}
\begin{enumerate}

\item El sistema verifica que no hayan mesas abiertas.
\item El sistema cierra el casino. \label{label_mesas_abiertas}
\item Fin CU. \label{label_fincucc}
\end{enumerate}

\negrita{Desarrollo alternativo}

\ref{label_mesas_abiertas}.1 Si hay mesas abiertas, el sistema le advierte al administrador que la operaci'on no puede ser realizada. Ir a PASO \ref{label_fincucc}.




% --------------- Pidiendo reporte ---------------------
\subsubsubsection{CU: Pidiendo reporte}
\negrita{Pre condici'on}: - \newline
\indent\negrita{Post condici'on}: -

\negrita{Actor primario}: Administrador \newline
\indent\negrita{Actores secundarios}: -

\negrita{Desarrollo normal}
\begin{enumerate}
\item El sistema le muestra al administrador los posibles reportes a pedir: Ranking de jugadores, Estado actual y Detalle movimientos por jugador.
\item El administrador elige el reporte que desea.
\item El sistema le mostrar'a al administrador, a partir de la informaci'on guardada en las jugadas sucesivas:
	\begin{itemize}
	\item si elige \negrita{Ranking de jugadores}, los jugadores que m'as dinero ganaron en el d'ia.
	\item si elige \negrita{Estado actual}, el informe del estado actual del casino y los clientes, especialmente los saldos respectivos.
	\item si elige \negrita{Detalle movimientos por jugador}, el detalle de todos los movimientos (apuestas, premios ganados, monto ganado) desde que ingresaron al casino.
	\end{itemize}
\item Fin CU.
\end{enumerate}




% --------------- Dirigiendo el azar ---------------------
\subsubsubsection{CU: Dirigiendo el azar}
\negrita{Pre condici'on}: - \newline
\indent\negrita{Post condici'on}: La/s mesa/s seleccionadas por el jefe de contabilidad se comportar'an de la forma se'nalada por 'este.

\negrita{Actor primario}: Jefe de contabilidad \newline
\indent\negrita{Actores secundarios}: -

\negrita{Desarrollo normal}
\begin{enumerate}
\item El sistema le muestra al jefe de contabilidad las opciones de
	\begin{itemize}
	\item Activar/desactivar modo dirigido de craps.
	\item Activar/desactivar modo dirigido de tragamonedas.
	\item Activar ''todos ponen''.
	\item Activar ''jugada feliz''.
	\end{itemize}
\item El jefe de contabilidad elige la opci'on deseada e ingresa los parametros necesarios.
\item El sistema efect'ua los cambios correspondientes sobre las mesas correspondientes.
\item Fin CU.
\end{enumerate}


