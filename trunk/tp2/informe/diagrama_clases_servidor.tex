\subsection{Gr'aficos del Diagrama}
Para facilitar la visualizaci'on del diagrama de clases, en lugar de imprimirlo completamente (de ser as'i se necesitar'ia una enorme mesa con agujeros en el medio para poder recorrerlo) preferimos particionarlo generando un gr'afico por cada m'odulo.

\subsubsection{Comunicaci'on}
\imagenvertical{DC_Servidor/DC_Comunicacion.png}{Diagrama de clases de la capa de comunicacion}{0.4}
\clearpage

\subsubsection{MensajeroDeEntrada}
%\imagenvertical{DC_Servidor/DC_Comunicacion.png}{Diagrama de clases de la capa de comunicacion}{0.3}
\clearpage

\subsubsection{MensajeroDeSalida}
%\imagenvertical{DC_Servidor/DC_Comunicacion.png}{Diagrama de clases de la capa de comunicacion}{0.3}
\clearpage

\subsubsection{Modelo}
%\imagenvertical{DC_Servidor/DC_Comunicacion.png}{Diagrama de clases de la capa de comunicacion}{0.3}
\clearpage


%\todo{Agregar el diagrama}
% \imagenvertical{DC_DiagramaCompleto_modelo.png}{Diagrama de Clases}{0.18}

\clearpage



\subsection{Explicaci'on}
Para facilitar el dise'no, mantener un nivel bajo de acoplamiento e incluso fomentar la reutilizaci'on de los elementos del diagrama, todas las clases fueron agrupadas en m'odulos. Cada m'odulo define una 'unica responsabilidad y las clases que lo contienen deben respetarla y llevarla a cabo. A continuaci'on tenemos la explicaci'on detallada de cada uno de ellos con sus clases relevantes.


\subsubsection{Comunicaci'on}
Se encarga de mantener la comunicaci'on de bajo nivel contra los clientes. 'Esto incluye la recepci'on de pedidos y el envio de las respuestas y mensajes de estado. Adem'as es el encargado de abstraer el medio de comunicaci'on (por ejemplo, por archivo que es el utilizado en el presente trabajo). Su l'ogica es simple y genera, consistiendo de rutinas de escucha de mensajes

\subsubsubsection{Clases relevantes}

\begin{description}
\item[ReceptorPedidos] Es el encargado de \italica{poolear} el medio que le fue especificado en b'usqueda de nuevos pedidos. Posee un ReceptorPedidosConcreto qui'en es el encargado de especificar el medio por el cual debe escuchar pedidos (en nuestro caso, por archivo) y es establecido en el arranque de la aplicaci'on. Una vez que se invoca a ComenzarRecepcion, el servidor ya est'a listo para comenzar a recibir pedidos de clientes. Notar que no es un singleton porque no se requiere limitar la cantidad de receptores (pueden haber uno por thread si el sistema fuese multithreading) ni requiere ser referenciado. 
\item[DespachadorRespuestas] Similar al receptor, 'este se encarga de enviar la respuesta ya generada de vuelta a un cliente. Es un singleton para que se permita su uso a quien as'i lo requiera, pero no para limitar la cantidad de instancias (en principio, podrian ser m'as si se necesitan atender despachos de varios threads). Al igual que el receptor, posee un DespachadorRespuestasEspecifico que se encarga de hacer el env'io real porque es quien realmente conoce el medio por donde se debe enviar la respuesta (en nuestro caso, por archivo).
\end{description}


\subsubsection{MensajeroDeEntrada}
Tiene como tarea la de manejar el flujo de informaci'on dentro del servidor. 'Esto incluye distribuir los pedidos recibidos desde la capa de comunicaci'on a los respectivos encargados de atenderlos y de avisarle a los encargados de generar las respuestas que las generen (adem'as de informar cu'al de ellas deben generar). En su l'ogica, principalmente sabe recorrer los XML para obtener los valores que necesitan los responsables de atender al pedido. Adem'as, sabe comprender dada una respuesta del manejo del pedido a quien le debe informar que respuesta devolver al cliente.

\subsubsubsection{Clases relevantes}

\begin{description}
\item[DespachadorPedidos] Es el responsable de despachar cada pedido recien entrado a su manejador correspondiente.
\item[\italica{Manejadores}] Cada manejador posee la responsabilidad de atender a pedidos de un conector del lado del cliente y posee un m'etodo por cada pedido que pueda llegar. Dentro de cada uno de 'estos m'etodos se encuentra la l'ogica de invocaci'on al Modelo y MensajeroDeSalida.
\end{description}


\subsubsection{MensajeroDeSalida}
Se encarga de generar las respuestas a los clientes, sabiendo de donde obtener la informaci'on necesaria para lograrlo. Adem'as es quien atiende eventos generados por la l'ogica del casino y sabe como actual ante cada uno de ellos. Tiene l'ogica para conseguir los datos necesarios para generar las respuestas, al igual que sabe como 'estas deben ser armadas.

\subsubsubsection{Clases relevantes}

\begin{description}
\item[\italica{Manejadores}] De forma similar a los manejadores del DespachadorDeEntrada, existe uno por cada conector del lado del cliente y posee un m'etodo por cada respuesta posible que se le deba enviar. Dentro de cada uno de 'estos se encuentra la l'ogica para obtener del Modelo la informaci'on necesaria en la construcci'on de las respuestas (al igual que la forma en la cual 'estas deben ser armadas).
\item[MesaCrapsObserver] Es un observador de todas las mesas de craps abiertas en el casino. Se encarga de realizar las notificaciones necesarias al realizarse alguna modificaci'on en una de ellas; para ello es que contiene informaci'on relevante a los destinatarios de 'estas notificaciones.
\end{description}


\subsubsection{Modelo}
Contiene la l'ogica de negocio junto con las estructuras asociadas a ella. Expone interfaces (llamadas \italica{fachadas}) mediantes las cuales los demas m'odulos pueden comunicarse con 'el. Su l'ogica esta compuesta por las validaciones ante cada mensaje y las consultas y modificaciones que 'estos generan. 

\subsubsubsection{Clases relevantes}

\begin{description}
\item[\italica{Fachadas}] Conforman la interfaz del Modelo. Mediante ellas los clientes pueden comunicarse con la l'ogica del casino. Contienen la l'ogia de validaci'on de las operaciones invocadas y conocen como manipular las dem'as clases del modelo para cumplir con los pedios que les solicitan. Para agregar una mayor cohesi'on cada una representa una parte del casino. 'Estas son:
	\begin{description}
	\item[LobbyCasino] representa al punto de entrada a un casino, algo asi como el \italica{lobby} o recepci'on de un casino. Aqu'i tambi'en es donde se quedan los observadores (dado que no pueden acceder a las mesas de juego).
	\item[JuegoCraps] engloba un juego de craps (el cual no necesariamente debe estar dentro de un casino). 'Esto incluye la administraci'on de las mesas, sus jugadores, apuestas y resultados.
	\item[JuegoTragamonedas] al igual que JuegoCraps, representa la creaci'on de mesas tragamonedas y el manejo de apuestas y jugadas sobre ellas.
	\item[AdministradorDeCasino] es el punto de administraci'on del casino y brinda servicios requeridos por los miembros del cuerpo administrativo de un casino: la configuraci'on del manejo dirigido y el pedido de reportes.
	\end{description}

\item[\italica{Capa de soporte}] Las clases aqu'i encotradas son los puntos de acceso a la informaci'on del casino. Son todas singleton porque al ser solo soporte de las dem'as clases, con una instancia alcanza. Adem'as, es necesario que sean accedidas desde cualquier punto (principalmente, las fachadas). A continuaci'on, el detalle de ellas:
	\begin{description}
	\item[JugadoresRegistrados] Administra y brinda acceso a los jugadores registrados, es decir, aquellos que marketing estableci'o en la lista que le configura al servidor.
	\item[UsuariosEnCasino] Administra y brinda acceso a los usuarios (los cuales pueden ser jugadores u observadores) que ya ingresaron al casino.
	\item[MesasAbiertas] Administra y brinda acceso a las mesas del casino que fueron creadas y por ende (debido a que una mesa sin jugadores es autom'aticamente destruida) que tienen jugadores jugando.
	\item[HistorialJugadas] Administra y brinda acceso a las jugadas que ya fueron realizadas en el transcurso del d'ia, ya sean de craps o tragamonedas.
	\item[ServidorJugadas] Es el encargado de brindar los resultados y tipos para las jugadas efectuadas en cada mesa del casino. De igual manera, permite la configuraci'on del modo dirigido.
	\item[ConfiguracionCasino] Brinda informaci'on sobre la configuraci'on global del casino y sus juegos.
	\item[Pozos] Administra y brinda acceso a los pozos del casino.
	\end{description}

\item[JugadorRegistrado] Representa a un jugador inscripto en la lista de marketing. Contiene el nombre y saldo del jugador y una \italica{abstract factory} (JugadorFactory) que permite abstraer el tipo de jugador a construir (Vip o Normal) cuando 'este entre en el casino.

\item[Usuario] Representa a un observador o un jugador. Contiene todos los datos de 'este (en el caso del observador solo el nombre, para el jugador existe tambi'en el saldo y la mesa en la cual est'a jugando (a menos que no lo est'e). Un jugador a su vez puede ser normal o vip, donde la 'unica diferencia es distinguir qui'en puede quedarse con saldo negativo (vip) y quien no (normal).

\item[Mesa] Representa a una mesa abierta en el casino. Contiene informaci'on de sus jugadores, las apuestas que se realizaron pero todavia no se resolvieron, ultima jugada realizada, observadores que la estan observando (para craps), pr'oximo y 'ultimo tirador, etc. Se especializa en un tipo de mesa para cada juego (en este caso, craps y tragamonedas). Notar que la mesa es la responsable de recibir y administrar las apuestas, los tiros y generar las jugadas correspondientes (al igual que registrarlas en el historial).

\item[Jugada] Una jugada es un tiro en alg'un juego; JugadaCraps es un tiro de dados y JugadaTragamonedas es un giro de rodillos. Cada una tiene la posibilidad de resolverse pero para esto primero necesita que le brinden toda la informaci'on relevante (el resultado de la jugada, el tipo de jugada y la/s apuestas) aunque guarde a'un m'as informaci'on de utilidad para el historial (como por ejemplo el tirador y el valor de los pozos relevantes). En su resoluci'on, le pide al tipo de jugada (Feliz, TodosPonen, Normal) que se resuelva sin importar cual sea (patr'on \italica{strategy}) y 'este hace lo mismo con cada apuesta en funci'on de lo que su l'ogica le indique. Una jugada adem'as guarda los premios pagados (cuanto gan'o cada jugador, separado por concepto).

\item[\italica{Selectores}] Los selectores son los portadores de la configuraci'on del modo dirigido. Cada uno le abstrae su funcionalidad espec'ifica al ServidorJugadas (patr'on \italica{strategy}) estableci'endo solamente que informaci'on es capaz de brindarle. Espec'ificamente, las clases selectoras son
	\begin{description}
	\item[SelectorResultadoCraps] Brinda el resultado de craps (los dados) que saldr'an en la mesa especificada. Notar que 'este puede ser tanto un selector que elige azarosamente o bien que utiliza una configuraci'on asignada por el modo dirigido.
	\item[SelectorResultadoTragamonedas] S'imil al de craps, con la diferencia que el resultado ahora es de tragamonedas por lo que se compone de tres rodillos en lugar de dos dados.
	\item[SelectorTipoJugada] Devuelve un tipo de jugada (Normal, Feliz, TodosPonen) para la mesa requerida. La elecci'on pudo haber sido basada en varios criterios, configurables por el administrador:
		\begin{description}
			\item[Feliz] La mesa elegida debe sacar siempre jugada feliz.
			\item[TodosPonen] La mesa elegida debe sacar siempre jugada todos ponen.
			\item[Normal] La mesa elegida debe sacar siempre jugada normal.
			\item[Azar] La mesa elegida calcular'a su tipo de jugada azarosamente, calculando primero si debe ser feliz en funci'on de la probabilidad establecida y de no ser as'i realiza la misma operaci'on para todos ponen. Si tambi'en falla, la jugada ser'a normal.
			\item[NoFeliz] Similar a Azar, pero no se realiza la verificaci'on de una jugada feliz.
			\item[NoTodosPonen] Similar a Azar, pero no se realiza la verificaci'on de una jugada todos ponen. 
		\end{description} 
	\end{description}

\item[\italica{Configuraciones}] 'Estas clases simplemente agrupan las configuraciones del casino bajo distintos criterios, para facilitar su trato. Solamente son utilitarias del soporte ConfiguracionCasino. 

\item[Pozo] Cada pozo (tanto el Feliz como el Progresivo) est'a compuesto por un monto (su valor actual) y posee la popiedad de resetearse a su monto m'inimo pre establecido en la configuraci'on del casino (utilizado cuando alguien lo gana). 

\end{description}
