\documentclass[spanish, a4paper, 10pt, titlepage]{article}
\author{Echevarr'ia - Farjat - Freijo - Giusto}

% Incluyo los paquetes y configuraciones
% Archivo de configuracion del informe
% -------------------------------------------

\usepackage[spanish,activeacute]{babel}			% Idioma castellano


\usepackage{caratula}														% Caratula de Algo2
%\usepackage[a4paper=true,pagebackref=true]{hyperref}				% Agrega la TOC al PDF e hipervinculos
\usepackage{graphicx} 											% Permite insertar graficos
\usepackage{fancyhdr}														% Permite manejo de cabeceras de pagina
\usepackage{eufrak}															% Usado en el enunciado del trabajo
\usepackage{latexsym}

%\usepackage{algorithmic}													% Para escribir los algos
%\usepackage{dsfont}															% Para el simbolo de naturales


\usepackage[font=small,labelfont=bf]{caption}						% Para editar las captions

\usepackage[utf8]{inputenc}
\usepackage{amsmath}
\usepackage[x11names, rgb]{xcolor}
% \usepackage{tikz}
% 	\usetikzlibrary{snakes,arrows,shapes}
\usepackage{listings}
\usepackage{lastpage}
\usepackage{geometry}
  \geometry{left=1cm, right=1cm, top=2cm, bottom=2cm}

% Estilo de pagina para tener las cabeceras y pieseras
\pagestyle{fancy}
  \fancyhead[LO]{Ingenier\'ia de Software I}
  \fancyhead[C]{Trabajo Práctico}
  \fancyhead[RO]{Primer Cuatrimestre 2008}
  \renewcommand{\headrulewidth}{0.4pt}

  \fancyfoot[LO]{Echevarr\'ia - Farjat - Freijo - Giusto}
  \fancyfoot[C]{}
  \fancyfoot[RO]{P\'agina \thepage\ de \pageref{LastPage}}
  \renewcommand{\footrulewidth}{0.4pt}

\parindent = 1.5 em 
\parskip = 8 pt


%%%%%%%%%%%%%%%% COMANDOS %%%%%%%%%%%%%%%%%%%%
\newcommand{\todo}{{\large\textbf{TODO: }}}
\newcommand{\paso}{\textsc{Paso }}
\newcommand{\func}[1]{\verb"#1"}
\newcommand{\imagen}[3]
{
	\begin{figure}[p!hbt]
	  \centering
	    \includegraphics[scale=#3]{../img/#1}
	  \caption{#2}
	\end{figure}
}
\newcommand{\imagenvertical}[3]
{
	\begin{figure}[p!hbt]
	  \centering
	    \includegraphics[angle=90,scale=#3]{../img/#1}
	  \caption{#2}
	\end{figure}
}

\newcommand{\nat}{\mathds{N}}
\newcommand{\algoritmo}[3]{\noindent {\bf\underline{#1}:} #2 $\longrightarrow$ #3}
\newcommand{\superindice}[1]{$^\textrm{{\tiny #1}}$}
\newcommand{\subsubsubsection}[1]{\noindent\negrita{#1}

}
\newcommand{\negrita}[1]{{\bf #1}}
\newcommand{\italica}[1]{{\it #1}}

% -----------------------------------------------------------------------------------------------------
% Codigo para generar los casos de uso, basado en caratula.sty del DC-FCEN-UBA 
% autor Nicolas Rosner
% Modificado por Pablo Echevarria 25-5-08
% ToDo: ver como declarar primero titulo y despues operaciones
% ver como numerar automaticamente las operaciones
% -----------------------------------------------------------------------------------------------------
% Token list para las instrucciones ----
\newtoks\oplist\oplist={}
% Comando para que el usuario agregue operaciones del CU
\newcommand{\op}[2]{\oplist=\expandafter{\the\oplist
\hline#1&#2\\ }}
% Comando para generar el CU con las operaciones ya pasadas y dandole la info que falta
\newcommand{\cu}[5]{
\begin{center}
\begin{tabular}{|p{12cm}|p{5cm}|}
\hline
\multicolumn{2}{|l|}{\begin{large}\italica{\negrita{CASO DE USO:} #1}\end{large} } \\ [0.2em]
\hline
\multicolumn{2}{|l|}{\negrita{Actor Primario:} #2}\\[0.2em]
\hline
\multicolumn{2}{|l|}{\negrita{Actor Secundario:} #3 }\\[0.2em]
\hline
\multicolumn{2}{|l|}{\negrita{Precondición:} #4 }\\[0.2em]
\hline
\multicolumn{2}{|l|}{\negrita{Poscondición:} #5} \\[0.2em]
\hline
\multicolumn{2}{|l|}{} \\[0.2em]
\hline
\negrita{Curso Normal} & \negrita{CursoAlternativo} \\[0.2em]
\the\oplist
\hline
\end{tabular}
\end{center}
\oplist={}
}



%%%%%%%%%%%%%%%% FIN COMANDOS %%%%%%%%%%%%%%%%%%%%


%%%%%%%%%%%%%%%%%%%%%%%%%%%%%%%%%%%%%%%%%%%%%%%%%%%%%%%%%%%%%%
%%%%%%%%%%%%%%%%%%%%%%%%%%%%%%%%%%%%%%%%%%%%%%%%%%%%%%%%%%%%%%%%%%%%%%%%%%%%%%%%%%%%
%%%%%   Inicio del documento
%%%%%%%%%%%%%%%%%%%%%%%%%%%%%%%%%%%%%%%%%%%%%%%%%%%%%%%%%%%%%%%%%%%%%%%%%%%%%%%%%%%%
%%%%%%%%%%%%%%%%%%%%%%%%%%%%%%%%%%%%%%%%%%%%%%%%%%%%%%%%%%%%%%
\begin{document}

% Caratula y tabla de contenidos

\materia{Ingenieria de Software 1}
\submateria{Primer Cuatrimestre 2008}
\titulo{Trabajo Pr'actico}
\subtitulo{Casino Online} 
\grupo{}

   \integrante{Diego Freijo }{ 4/05  }{  giga.freijo@gmail.com}
   \integrante{Lucas Farjat  }{ 468/05 }{   lacacks@gmail.com}
   \integrante{Maximiliano Giusto }{ 486/05  }{  maxi.giusto@gmail.com}
   \integrante{Pablo Echevarr'ia }{ 133/00 }{   pablohe@gmail.com}
 
\maketitle


% 
% \newpage



\tableofcontents
\clearpage

% ------------------------------------------------------
% Secciones
% ------------------------------------------------------


% Introduccion
\section{Introducci'on}
% Introduccion. Cambios con respecto al informe 1.

En esta secci'on describiremos los cambios con respecto al informe 1.


\clearpage
 
% Escenarios
\section{Escenarios y Diagramas de Secuencia}
Para la construcci'on de los escenarios y diagramas de secuencia fue basada en el echo de que contabamos con los mensajes de protocolo los cuales inducen interaci'on y modificaci'on de nuestro modelo.

Tambi'en decidimos hacer que nuestros escenarios sean lo m'as gen'ericos posible siempre y cuando esto no complique la lectura. 

En cuanto a la profundidad para algunos DS mostramos la interacci'on de punta 
a punta estos se encuentran en la secci'on del mismo nombre.
Factorizandolos en 2 o 3 DS's cada uno. El resto de los diagramas, salvo excepciones, comienzan con la llamada al m'etodo de la fachada del modelo.

Dado que la secci'on de recepci'on de pedidos y el despacho es muy parecida en estos diagramas 
decidimos obviarlos (salvo, claro, en los de punta a punta).


Por simpleza no usamos los \textit{ObtenerInstancia} de los Singletons. Estos estar'an implicitos.


Para tratar con los XML asumiremos que contamos con cierta funcionalidad, provista por los objetos, por ejemplo:
\begin{itemize}
\item  Para obtener el atributo \textbf{idMesa} de un XML usamos \textit{XML.idMesa}, donde idMesa es el tag que se encuentra en el XML, y  \textit{XML.idMesa}, nos devoveria un entero almacenado en ese tag.

\item As'i tambi'en para la lista de jugadores, \textit{XML.jugadoresEnMesa} devuelve una \textbf{lista$<$jugadores$>$}.
\end{itemize}

% 
% \escenario{ \textbf{Startup del Casino}
% \begin{itemize}
% \item se inicia el servidor
% \item se leen los achivos de configuraci'on
% \end{itemize}
% 
% }
% 
% \escenario{ \textbf{Startup Cliente}
% \begin{itemize}
% \item se incia el cliente
% \item ???????????????????????????????????
% \end{itemize}
% 
% }

\subsection{Diagramas de punta a punta}

En esta secci'on mostraremos DS desde que se leen los arhivos XML hasta que el mensajero de salida despacha los archivos XML de respuesta.

\subsubsection{Craps}

\escenario{Tirar Dados de punta a punta}{

El usuario puede o no estar en la mesa. El usuario puede o no ser el tirador. En caso de que sea el tirador y est'e en la mesa. Se tiran los dados. Si sali'o alg'un valor de punto este se setea. Hay alguna cantidad de apuestas de alg'un tipo que se resuelven o no con su l'ogica particular, dependiendo del contexto. Estamos en un ``est'an saliendo''. Es una jugada feliz 

}
% 
% $^1$ Resoluci'on de distintas apuestas se ve en otros DS's (en sitio a perder y a venir)
% $^2$ En otro DS se ve la situacion con el caso de que el punto est'e establecido
% $^3$ En otro DS se ve la jugada Todos ponen

\textbf{Aclaraciones: }El pozo feliz se reparte si o si. El servidor de Jugadas no devuelve una feliz mientras el pozo no lleque al m'inimo.
La notificaci'on de que una mesa cambió se pude ver en el DS \textit{apostarCraps}, sin bien en este DS est'a instanciado en una mesa particular, aplica si la mesa fuera gen'erica.


Este DS se dividió en 3 secciones:
\begin{enumerate}
\item Recepcion de pedido: es la recepcion de pedido,  y la respuesta hacia el modulo de comunciación, no se muestra lo que sucede en la llamada TirarCraps (usuario, unXML)
\item  TirarCraps: Hace todo lo concerniente a la validaci'on, no se hace zoom en TirarDados.
\item TirarDados: aqui puede verse lo que pasa cuando se hace un tirar dados de una mesa (Va impreso aparte o digital por el tama~no)
\end{enumerate}

%aca iria una imagen------------------------------------------------------------
\imagenvertical{DS_Craps/TirarDados/RecepcionpPedido.png}{Tirar Dados Recepcion de pedido}{0.5}
\imagenvertical{DS_Craps/TirarDados/tirarCraps.png}{Tirar Dados: TirarCraps}{0.5}
% \imagen{}{TirarDados}{0.5}
Tirar dados \tam

\escenario{Apostar Craps de punta a punta}{

El jugador Cosme Fulanito desea apostar en el juego de craps. La apuesta realizada es a ganar sobre el n'umero cinco. La cantidad apostada es una ficha de valor 20 y dos de valor 10. Cosme est'a en la mesa 25.
}
%Se crea el XML correspondiente con el nombre del archivo "apuestaCraps059999" y se deposita en la carpeta del servidor.

%En el servidor se levanta dicho XML, se lo parsea y se efect'uan las valiaciones correspondientes. A saber:

Para que su apuesta sea efectiva deber'a pasar por las siguientes validaciones:

\begin{enumerate}
 \item Cosme Fulanito es un jugador de la mesa en la que acusa estar;
 \item Los valores de fichas utilizados son v'alidas para 'este d'ia;
 \item Debe tener los fondos suficientes para poder pagar lo que apost'o o que sea un cliente vip;
\end{enumerate}

Si alguna de las validaciones anteriormente mencionadas no se cumple entonces no se le dejar'a realizar la apuesta. En caso contrario si.

Si se cumplen todas las validaciones se agregar'a la apuesta a la mesa y adem'as se notificar'a a todos los dem'as jugadores de la mesa 25 del cambio sucedido.



 \escenario{ PedirEstado de punta a punta }{
Un usuario desea informarse sobre el estado del casino. Se le informar'a s'olo si ha ingresado en el casino, m'as haya si es en modo jugador o en modo observador.

El estado del casino est'a formado por:
\begin{itemize}
 \item La lista de jugadores y observadores ingresados en el casino;
 \item El valor del pozo feliz y del pozo progresivo;
 \item El estado de las mesas del juegos de craps:
	\begin{itemize}
	 \item Los jugadores;
	 \item El 'ultimo tirador y el pr'oximo;
	 \item Si el siguiente es tiro de salida o ya est'a el punto establecido;
	 \item El valor de los dados en el 'ultimo tiro;
	\end{itemize}
 \item El estado de las mesas del juego tragamonedas:
	\begin{itemize}
 	 \item Los jugadores;
 	 \item El valor de los rodillos en el 'ultimo tiro;
 	 \item El 'ultimo tirador y el pr'oximo;
	\end{itemize}
\end{itemize}
 }



\subsubsection{Tragamonedas}
\input{escenarios_tragamonedas_pap.tex}

\subsection{Casino}
 \escenario{Entrar Casino}{
Un usuario desea entrar al casino, puede hacerlo en modo jugador o en modo observador.
Si ya ha ingresado en modo jugador no se lo dejar'a entrar nuevamente. Si est'a en modo observador y desea ingresar en el mismo modo tampoco podr'a hacerlo.

En cambio si quiere entrar como jugador (independientemente de si ingres'o como observador o si no ingres'o) se deber'a validar que sea un usuario autorizado por marketing:

\begin{itemize}
 \item En caso afirmativo quedar'a ingresado en modo jugador.
 \item En caso negativo quedar'a en modo observador o fuera del casino seg'un cual fuese su estado anterior.
\end{itemize}
 }
\imagen{DS_Casino/EntarCasino/DS_EntrarCasinoFueraDelModelo.png}{Entrar casino fuera del Modelo}{0.6}
%aca iria una imagen------------------------------------------------------------
Entrar casino dentro del Modelo \tam

\clearpage



%aca iria una imagen------------------------------------------------------------

\clearpage

\textbf{Pedir Estado Casino}

Un usuario desea informarse sobre el estado del casino. Se le informar'a s'olo si ha ingresado en el casino, m'as haya si es en modo jugador o en modo observador.

El estado del casino est'a formado por:

\begin{itemize}
 \item La lista de jugadores y observadores ingresados en el casino;
 \item El valor del pozo feliz y del pozo progresivo;
 \item El estado de las mesas del juegos de craps:
	\begin{itemize}
	 \item Los jugadores;
	 \item El 'ultimo tirador y el pr'oximo;
	 \item Si el siguiente es tiro de salida o ya est'a el punto establecido;
	 \item El valor de los dados en el 'ultimo tiro;
	\end{itemize}
 \item El estado de las mesas del juego tragamonedas:
	\begin{itemize}
 	 \item Los jugadores;
 	 \item El valor de los rodillos en el 'ultimo tiro;
 	 \item El 'ultimo tirador y el pr'oximo;
	\end{itemize}
\end{itemize}
\tam

%aca iria una imagen------------------------------------------------------------

\clearpage

\textbf{Salir Casino}

Usuario desea salir del casino.

Si ha ingresado como observador no tendr'a ning'un tipo de validaci'on, por consiguiente tampoco problemas.

Si ha ingresado como jugador, para poder salir deber'a estar fuera de toda mesa. Es decir, no puede pretender salir del casino si es que est'a dentro de una mesa jugando.

\imagen{DS_Casino/SalirCasino/DS_SalirCasino.png}{Salir del Casino}{0.6}

%aca iria una imagen------------------------------------------------------------





\subsection{Diagr'amas de que no son de punta a punta}
Dado que la secci'on de recepci'on de pedidos y el despacho es muy parecida en estos diagramas decidimos obviarlo.

\subsubsection{Funcionalidades de inicializaci'on}
Los escenarios aqui presentados son muy gen'ericos.


\escenario{ la Configuraci'on general del Casino}
{
Se setea el valor de las fichas, el saldo del casino y la pasword del aministrador
}
% imagen

\imagenvertical{DS_InicioServidor/DS_InicializarConfiguracion.png}{Inicializar Configuraci'on}{0.5}

\escenario{Jugadores Registrados}{
En este DS se ve como se setea la lista de jugadores registrados del casino
}
\imagenvertical{DS_InicioServidor/DS_InicializarJugadoresRegistrados.png}{Inicializar Jugadores Registrados}{0.4}

% imagen


\escenario{Inicializar Mesas}{
En este DS puede verse como se inicalizan las mesas abiertas, se asigna el observador de cambios.
}
\imagen{DS_InicioServidor/DS_InicializarMesas.png}{Inicializar Mesas}{0.5}
% imagen


\escenario{Inicio del Servidor}{
En este DS se puede ver como se ``enciende'' el servidor,
como se crea el el Obtenerdor de pedidos, el receptor de pedidos de archivos,  

}
\imagenvertical{DS_InicioServidor/DS_InicioServidor.png}{Inicio Servidor}{0.4}


\subsubsection{Funcionalidades generales de los administradores}

En el DS: puede verse la secuencia de un seteo de el modo dirigido de dados.


En el DS: puede verse el pedido de el reporte PedidoReporteDetalleMovimientosPorJugador()





\subsubsection{Tragamonedas}

 \escenario{Tirar Tragamonedas}
 { El usuario puede estar en modo jugador o no. Una vez validado el usuario, ahora jugador, puede ingresar al Tragamonedas o no. Puede o no haber hecho una apuesta. Es una jugada normal. Gira los rodillos. La apuesta se resuelve y se le acreditan creditos si corresponde.}
%imagen
\imagenvertical{DS_Tragamonedas/TirarTragamonedas/Tirar Tragamonedas.png}{Tirar Tragamonedas}{0.5}


 \escenario{Entrar Tragamonedas}{
El jugador puede estar en el casino en modo jugador o no.
Podr'ia estar en alguna mesa. En este escenario mostramos en particular como se hace para enviar los mensajes donde se Acepta o Deniega, en este caso la Entrada al juego.
Se muesta la iteracci'on con los mesajeros. 
}

%imagen


\subsubsection{Craps}
\escenario{ Entrar Craps}{
Este escenario es bastante gen'erico. Se muestra como se valida cada cosa, como actua el sistema en cada caso
y que mensaje de error da.

El usuario puede o no estar en el casino en modo jugador.(incluye modo observador o no haber ingresado)
Puede estar en otra mesa o puede desear crearla.
}

% imagen


\escenario{Resolverse Apuesta de Sitio a Ganar}{
La ronda esta en ``Est'an Saliendo'' sali'o un 4. La apuesta se resuelve, pasa a estar cerrada
}

% imagen



\escenario{Resolverse Apuesta Venir}{
Se estableci'o el punto. Se le paga. La apuesta se cierra
}

% imagen


\escenario{ \textbf{Jugador de Craps haciendo una apuesta}
  \begin{itemize}
   \item El usuario esta en una mesa de craps 
  \item elige un valor de ficha de \$20
  \item elige un valor de ficha de \$15
  \item elige el tipo de apuesta
  \item por cada eleccion se ve un mensaje en el log
  \item si elige una apuesta antes de una ficha  da un error
  \end{itemize}
}






\clearpage

% - Pseudocodigo de operaciones mas complicadas en lo algoritmico.
\section{Pseudoc'odigo de operaciones m'as complicadas en lo algoritmo.}
\subsubsubsection{ObtenerJugadoresMasGanadores(): Lista $<$Nombre$>$}

'Este m'etodo pertenece a la clase \italica{AdministradorDeCasino}. Devuelve la lista ordenada en forma descendente de los jugadores que m'as dinero han ganado en el casino durante el presente d'ia.

\begin{verbatim}
AdministradorDeCasino::ObtenerJugadoresMasGanadores(): Lista<Nombre>
	
   jugadas = elHistorialDeJugadas.GetJugadas()
   jugadasTragamonedas = elHistorialDeJugadas.GetJugadasTragamonedas()
   premios = jugadas.GetPremios()
   premiosTragamonedas = jugadasTragamonedas.GetPremio()

   Para cada elemento de premios hacer
      gano = monto_normal + monto_feliz - monto_todosponen - monto_apostado
      Si al jugador asociado es nuevo (no aparecio en una iteracion anterior)
         se le guarda este valor
      si no
         se le suma al valor que ya tenia
      Fin si
   Fin para

   Para cada elemento de premiosTragamonedas hacer
      gano = monto_normal + monto_feliz - monto_todosponen - monto_apostado +
             monto_progresivo
      Si al jugador asociado es nuevo (no aparecio en una iteracion anterior)
         se le guarda este valor
      si no
         se le suma al valor que ya tenia
      Fin si
   Fin para

   De los jugadores que menor valor tienen en gano se toma a los 3 primeros y se
   los retorna
\end{verbatim}


\subsubsubsection{ObtenerJugadoresMasPerdedores(): Lista$<$Nombre$>$}

'Este m'etodo pertenece a la clase \italica{AdministradorDeCasino}. Devuelve la lista ordenada en forma descendente de los jugadores que m'as dinero han perdido en el casino durante el presente d'ia.

\begin{verbatim}
AdministradorDeCasino::ObtenerJugadoresMasPerdedores(): Lista<Nombre>
	
   jugadas = elHistorialDeJugadas.GetJugadas()
   jugadasTragamonedas = elHistorialDeJugadas.GetJugadasTragamonedas()
   premios = jugadas.GetPremios()
   premiosTragamonedas = jugadasTragamonedas.GetPremio()

   Para cada elemento de premios hacer
      perdio = monto_normal + monto_feliz - monto_todosponen - monto_apostado
      Si al jugador asociado es nuevo (no aparecio en una iteracion anterior)
         se le guarda este valor
      si no
         se le suma al valor que ya tenia
      Fin si
   Fin para

   Para cada elemento de premiosTragamonedas hacer
      perdio = monto_normal + monto_feliz - monto_todosponen - monto_apostado +
               monto_progresivo
      Si al jugador asociado es nuevo (no aparecio en una iteracion anterior)
         se le guarda este valor
      si no
         se le suma al valor que ya tenia
      Fin si
   Fin para

   De los jugadores que menor valor tienen en perdio se toma a los 3 primeros y
   se los retorna
\end{verbatim}


\subsubsubsection{DetalleMovimientoJugadores(): Coleccion$<$Tupla$<$Nombre, Texto, Creditos, Creditos, Creditos, Creditos$>$ $>$}

'Este m'etodo pertenece a la clase \italica{AdministradorDeCasino}. Devuelve la una coleccion con los jugadores sus apuestas y los premios ganados y perdidos durante el presente d'ia.

\begin{verbatim}
AdministradorDeCasino::DetalleMovimientoJugadores(): Coleccion<Tupla<Nombre,
Texto, Creditos, Creditos, Creditos, Creditos>>

   jugadas = elHistorialDeJugadas.GetJugadas()
   jugadasTragamonedas = elHistorialDeJugadas.GetJugadasTragamonedas()
   premios = jugadas.GetPremios()
   premiosTragamonedas = jugadasTragamonedas.GetPremio()

   Para cada elemento p de premios hacer
      unaTupla = Tupla(p.apostador, p.nombre_tipo_apuesta, p.monto_normal,
                       p.monto_feliz, p.monto_todosponen, Null)
      laListaDeRetorno.Agregar(unaTupla)
   Fin para

   Para cada elemento p de premiosTragamonedas hacer
      unaTupla = Tupla(p.apostador, p.nombre_tipo_apuesta, p.monto_normal,
                       p.monto_feliz, p.monto_todosponen, p.montoProgresivo)
      laListaDeRetorno.Agregar(unaTupla)
   Fin para

   Retornar laListaDeRetorno
\end{verbatim}



% Diagrama de clases
\section{Diagrama de clases}
\subsection{Introducci'on}
El m'odulo del casino se puede separar en dos partes: 

\begin{itemize}
\item componentes \italica{Cliente}
\item componente \italica{Servidor}
\end{itemize}  

La comunicaci'on entre el componente {\it Servidor} y los componentes {\it Cliente} ya fue resuelta por el documento de Arquitectura Conceptual y Protocolo brindado por los clientes (ver documento adjunto) y por las extensiones realizadas al mismo (ver secci'on \ref{Seccion::ModificacionesAlProtocolo}.).

No existir'a comunicaci'on entre los componentes {\it Cliente} desde el punto de vista arquitectural, aunque s'i se relacionar'an indirectamente a trav'es de la l'ogica de negocio del servidor.

\subsection{Servidor}
\subsection{Gr'afico del Diagrama}

\todo{Agregar el diagrama}
% \imagenvertical{DC_DiagramaCompleto_modelo.png}{Diagrama de Clases}{0.18}

\clearpage

\subsection{Explicaci'on}
Para facilitar el dise'no, mantener un nivel bajo de acoplamiento e incluso fomentar la reutilizaci'on de los elementos del diagrama, todas las clases fueron agrupadas en m'odulos. Cada m'odulo define una 'unica responsabilidad y las clases que lo contienen deben respetarla y llevarla a cabo. A continuaci'on tenemos la explicaci'on detallada de cada uno de ellos con sus clases relevantes.


\subsubsection{Comunicaci'on}
Se encarga de mantener la comunicaci'on de bajo nivel contra los clientes. 'Esto incluye la recepci'on de pedidos y el envio de las respuestas y mensajes de estado. Adem'as es el encargado de abstraer el medio de comunicaci'on (por ejemplo, por archivo que es el utilizado en el presente trabajo). Su l'ogica es simple y genera, consistiendo de rutinas de escucha de mensajes

\subsubsubsection{Clases relevantes}

\begin{description}
\item[ReceptorPedidos] Es el encargado de \italica{poolear} el medio que le fue especificado en b'usqueda de nuevos pedidos. Posee un ReceptorPedidosConcreto qui'en es el encargado de especificar el medio por el cual debe escuchar pedidos (en nuestro caso, por archivo) y es establecido en el arranque de la aplicaci'on. Una vez que se invoca a ComenzarRecepcion, el servidor ya est'a listo para comenzar a recibir pedidos de clientes. Notar que no es un singleton porque no se requiere limitar la cantidad de receptores (pueden haber uno por thread si el sistema fuese multithreading) ni requiere ser referenciado. 
\item[DespachadorRespuestas] Similar al receptor, 'este se encarga de enviar la respuesta ya generada de vuelta a un cliente. Es un singleton para que se permita su uso a quien as'i lo requiera, pero no para limitar la cantidad de instancias (en principio, podrian ser m'as si se necesitan atender despachos de varios threads). Al igual que el receptor, posee un DespachadorRespuestasEspecifico que se encarga de hacer el env'io real porque es quien realmente conoce el medio por donde se debe enviar la respuesta (en nuestro caso, por archivo).
\end{description}


\subsubsection{MensajeroDeEntrada}
Tiene como tarea la de manejar el flujo de informaci'on dentro del servidor. 'Esto incluye distribuir los pedidos recibidos desde la capa de comunicaci'on a los respectivos encargados de atenderlos y de avisarle a los encargados de generar las respuestas que las generen (adem'as de informar cu'al de ellas deben generar). En su l'ogica, principalmente sabe recorrer los XML para obtener los valores que necesitan los responsables de atender al pedido. Adem'as, sabe comprender dada una respuesta del manejo del pedido a quien le debe informar que respuesta devolver al cliente.

\subsubsubsection{Clases relevantes}

\begin{description}
\item[DespachadorPedidos] Es el responsable de despachar cada pedido recien entrado a su manejador correspondiente.
\item[\italica{Manejadores}] Cada manejador posee la responsabilidad de atender a pedidos de un conector del lado del cliente y posee un m'etodo por cada pedido que pueda llegar. Dentro de cada uno de 'estos m'etodos se encuentra la l'ogica de invocaci'on al Modelo y MensajeroDeSalida.
\end{description}


\subsubsection{MensajeroDeSalida}
Se encarga de generar las respuestas a los clientes, sabiendo de donde obtener la informaci'on necesaria para lograrlo. Adem'as es quien atiende eventos generados por la l'ogica del casino y sabe como actual ante cada uno de ellos. Tiene l'ogica para conseguir los datos necesarios para generar las respuestas, al igual que sabe como 'estas deben ser armadas.

\subsubsubsection{Clases relevantes}

\begin{description}
\item[\italica{Manejadores}] De forma similar a los manejadores del DespachadorDeEntrada, existe uno por cada conector del lado del cliente y posee un m'etodo por cada respuesta posible que se le deba enviar. Dentro de cada uno de 'estos se encuentra la l'ogica para obtener del Modelo la informaci'on necesaria en la construcci'on de las respuestas (al igual que la forma en la cual 'estas deben ser armadas).
\item[MesaCrapsObserver] Es un observador de todas las mesas de craps abiertas en el casino. Se encarga de realizar las notificaciones necesarias al realizarse alguna modificaci'on en una de ellas; para ello es que contiene informaci'on relevante a los destinatarios de 'estas notificaciones.
\end{description}


\subsubsection{Modelo}
Contiene la l'ogica de negocio junto con las estructuras asociadas a ella. Expone interfaces (llamadas \italica{fachadas}) mediantes las cuales los demas m'odulos pueden comunicarse con 'el. Su l'ogica esta compuesta por las validaciones ante cada mensaje y las consultas y modificaciones que 'estos generan. 

\subsubsubsection{Clases relevantes}

\begin{description}
\item[\italica{Fachadas}] Conforman la interfaz del Modelo. Mediante ellas los clientes pueden comunicarse con la l'ogica del casino. Contienen la l'ogia de validaci'on de las operaciones invocadas y conocen como manipular las dem'as clases del modelo para cumplir con los pedios que les solicitan. Para agregar una mayor cohesi'on cada una representa una parte del casino. 'Estas son:
	\begin{description}
	\item[LobbyCasino] representa al punto de entrada a un casino, algo asi como el \italica{lobby} o recepci'on de un casino. Aqu'i tambi'en es donde se quedan los observadores (dado que no pueden acceder a las mesas de juego).
	\item[JuegoCraps] engloba un juego de craps (el cual no necesariamente debe estar dentro de un casino). 'Esto incluye la administraci'on de las mesas, sus jugadores, apuestas y resultados.
	\item[JuegoTragamonedas] al igual que JuegoCraps, representa la creaci'on de mesas tragamonedas y el manejo de apuestas y jugadas sobre ellas.
	\item[AdministradorDeCasino] es el punto de administraci'on del casino y brinda servicios requeridos por los miembros del cuerpo administrativo de un casino: la configuraci'on del manejo dirigido y el pedido de reportes.
	\end{description}

\item[\italica{Capa de soporte}] Las clases aqu'i encotradas son los puntos de acceso a la informaci'on del casino. Son todas singleton porque al ser solo soporte de las dem'as clases, con una instancia alcanza. Adem'as, es necesario que sean accedidas desde cualquier punto (principalmente, las fachadas). A continuaci'on, el detalle de ellas:
	\begin{description}
	\item[JugadoresRegistrados] Administra y brinda acceso a los jugadores registrados, es decir, aquellos que marketing estableci'o en la lista que le configura al servidor.
	\item[UsuariosEnCasino] Administra y brinda acceso a los usuarios (los cuales pueden ser jugadores u observadores) que ya ingresaron al casino.
	\item[MesasAbiertas] Administra y brinda acceso a las mesas del casino que fueron creadas y por ende (debido a que una mesa sin jugadores es autom'aticamente destruida) que tienen jugadores jugando.
	\item[HistorialJugadas] Administra y brinda acceso a las jugadas que ya fueron realizadas en el transcurso del d'ia, ya sean de craps o tragamonedas.
	\item[ServidorJugadas] Es el encargado de brindar los resultados y tipos para las jugadas efectuadas en cada mesa del casino. De igual manera, permite la configuraci'on del modo dirigido.
	\item[ConfiguracionCasino] Brinda informaci'on sobre la configuraci'on global del casino y sus juegos.
	\item[Pozos] Administra y brinda acceso a los pozos del casino.
	\end{description}

\item[JugadorRegistrado] Representa a un jugador inscripto en la lista de marketing. Contiene el nombre y saldo del jugador y una \italica{abstract factory} (JugadorFactory) que permite abstraer el tipo de jugador a construir (Vip o Normal) cuando 'este entre en el casino.

\item[Usuario] Representa a un observador o un jugador. Contiene todos los datos de 'este (en el caso del observador solo el nombre, para el jugador existe tambi'en el saldo y la mesa en la cual est'a jugando (a menos que no lo est'e). Un jugador a su vez puede ser normal o vip, donde la 'unica diferencia es distinguir qui'en puede quedarse con saldo negativo (vip) y quien no (normal).

\item[Mesa] Representa a una mesa abierta en el casino. Contiene informaci'on de sus jugadores, las apuestas que se realizaron pero todavia no se resolvieron, ultima jugada realizada, observadores que la estan observando (para craps), pr'oximo y 'ultimo tirador, etc. Se especializa en un tipo de mesa para cada juego (en este caso, craps y tragamonedas). Notar que la mesa es la responsable de recibir y administrar las apuestas, los tiros y generar las jugadas correspondientes (al igual que registrarlas en el historial).

\item[Jugada] Una jugada es un tiro en alg'un juego; JugadaCraps es un tiro de dados y JugadaTragamonedas es un giro de rodillos. Cada una tiene la posibilidad de resolverse pero para esto primero necesita que le brinden toda la informaci'on relevante (el resultado de la jugada, el tipo de jugada y la/s apuestas) aunque guarde a'un m'as informaci'on de utilidad para el historial (como por ejemplo el tirador y el valor de los pozos relevantes). En su resoluci'on, le pide al tipo de jugada (Feliz, TodosPonen, Normal) que se resuelva sin importar cual sea (patr'on \italica{strategy}) y 'este hace lo mismo con cada apuesta en funci'on de lo que su l'ogica le indique. Una jugada adem'as guarda los premios pagados (cuanto gan'o cada jugador, separado por concepto).

\item[\italica{Selectores}] Los selectores son los portadores de la configuraci'on del modo dirigido. Cada uno le abstrae su funcionalidad espec'ifica al ServidorJugadas (patr'on \italica{strategy}) estableci'endo solamente que informaci'on es capaz de brindarle. Espec'ificamente, las clases selectoras son
	\begin{description}
	\item[SelectorResultadoCraps] Brinda el resultado de craps (los dados) que saldr'an en la mesa especificada. Notar que 'este puede ser tanto un selector que elige azarosamente o bien que utiliza una configuraci'on asignada por el modo dirigido.
	\item[SelectorResultadoTragamonedas] S'imil al de craps, con la diferencia que el resultado ahora es de tragamonedas por lo que se compone de tres rodillos en lugar de dos dados.
	\item[SelectorTipoJugada] Devuelve un tipo de jugada (Normal, Feliz, TodosPonen) para la mesa requerida. La elecci'on pudo haber sido basada en varios criterios, configurables por el administrador:
		\begin{description}
			\item[Feliz] La mesa elegida debe sacar siempre jugada feliz.
			\item[TodosPonen] La mesa elegida debe sacar siempre jugada todos ponen.
			\item[Normal] La mesa elegida debe sacar siempre jugada normal.
			\item[Azar] La mesa elegida calcular'a su tipo de jugada azarosamente, calculando primero si debe ser feliz en funci'on de la probabilidad establecida y de no ser as'i realiza la misma operaci'on para todos ponen. Si tambi'en falla, la jugada ser'a normal.
			\item[NoFeliz] Similar a Azar, pero no se realiza la verificaci'on de una jugada feliz.
			\item[NoTodosPonen] Similar a Azar, pero no se realiza la verificaci'on de una jugada todos ponen. 
		\end{description} 
	\end{description}

\item[\italica{Configuraciones}] 'Estas clases simplemente agrupan las configuraciones del casino bajo distintos criterios, para facilitar su trato. Solamente son utilitarias del soporte ConfiguracionCasino. 

\item[Pozo] Cada pozo (tanto el Feliz como el Progresivo) est'a compuesto por un monto (su valor actual) y posee la popiedad de resetearse a su monto m'inimo pre establecido en la configuraci'on del casino (utilizado cuando alguien lo gana). 

\end{description}

\subsection{Cliente}
\input{diagrama_clases_cliente.tex}
\clearpage

% - Justificacion, analisis y explicacion de su dise�o, utilizando desde principios de dise�o, hasta patterns, pasando por depedencias, acoplamiento y cohesion, hasta mas secuencias de ejemplo para explicar porque su dise�o es bueno y elegante al resolver los problemas que se les presentaron.
\section{Justificaci'on, an'alisis y explicaci'on del dise'no}
\subsection{Servidor}
Se model'o basandonos en el patr'on de diseño MVC. Con lo cual tenemos tres grandes paquetes:

\begin{itemize}
 \item Modelo
 \item Controlador
 \item Vista
\end{itemize}
 
\todo REVISAR ESTO
La recepci'on de los pedidos se hace con la Clase ReceptorPedidos, que est'a dentro del paquete Vista. Esta clase ser'a especializada de la manera que sea necesaria. Esto brinda una mayor flexibilidad a la hora de tomar los pedidos y nos liga menos al tipo en el que llega el pedido.

La Clase DespachadorPedidos (paquete controlador) toma los par'ametros de entrada y multiplexa seg'un corresponda. Esta clase es singleton debido a que solo es necesaria una sola instancia para no tener problemas de concurrencia.

Las clases JuegoCraps, AccesoYVistaCraps, AccesoYVistaTragamonedas y AccesoYVistaCasino procesan los datos seg'un corresponda. Los datos son obtenidos de los administradores que est'an en el paquete Modelo. Esto crea acoplamiento sobre dichas clases, pero no las crea sobre el resto del paquete modelo. Si bien el acoplamiento est'a, 'este es entre paquetes que es menor que entre clases de distintos paquetes. 'Estas clases son singleton.

Tenemos las Clases Singleton para la emisi'on de la respuesta: JuegoCraps, AccesoYVistaCraps, AccesoYVistaTragamonedas y AccesoYVistaCasino.

Estas clases se corresponden uno a uno con las del controlador, brindando las respuestas a los pedidos. En el caso de necesitar datos, que no fueron brindados por el controlador, tambi'en le hacen el pedido a las clases administrador del paquete Modelo.

Como son para emitir la respuesta ent'an dentro del paquete Vista.

Para emitir la respuesta utilizar'an un tipo llamado Respuesta. Esto permite desacoplar del tipo de respuesta y dejar a la clase especializada del Despachador de respuestas, la responsabilidad del formato de respuesta. Como es especializaci'on en caso en que se modifique el modo en que se envia la respuesta se hereda una clase nueva. Favoreciendo el open-close.

\todo HASTA ACA

\subsubsection{Modelo}

\subsubsection{Controlador}

\subsubsection{Vista}

Tenemos una clase \textit{Pedido} que tiene adentro un xml con los par'ametros del pedido. Un tipo de pedido.
Entendemos que podriamos prescindir de esta clase y llamar a Controlador correspondiente pasandole el xml,
creemos que no es una buena decisi'on ya que si no contamos con ella, violariamos el principio Open-Close, dado 
que si en un futuro el pedido cambia y por ejemplo decicimos agregar un timestamp para que caduque tendriamos que 
modificar funcionalidades ya testeadas.

\subsection{Cliente}

El cliente se basa en un diseño en capas.

\imagenvertical{DC_Cliente.png}{Diagrama de Clases Cliente}{0.4}

\begin{enumerate}
	\item \textbf{Presentaci'on: }Estar'an las clases qeu se encargaran de dibujar las distintas pantallas de juego y de manejar todo lo relacionado a los enventos que los usuarios activar'an.
	\item \textbf{Controlador: } estara la lógica que construye los mensajes al servidor y las clases necesarias para ello.
	\item \textbf{Comunicaci'on:} estar'an las clases y metodos que construyen el archivo xml y la acci'on de escribirlos en el puerto correspondiente del servidor (en este caso una carpeta en el mismo).
	\item  \textbf{Modelo:} este paquete tendrá una clase, \textit{configuraci'on}, la cual contendr'a toda la informaci'on que los mensajes de comunicaci'on no contengan y que son necesarios para poder mostrar las pantallas y construir los mensajes de comunciaci'on
 \end{enumerate}

\subsubsection{Justificaci'on del dise~no}
Creemos que este dise~no es adecuado para este problema ya que es sumamente flexible tanto si se cambia la capa de presentaci'on como si se cambia la capa de comunciaci'on. Por ejemplo, si el d'ia de mañana queremos mostrarlo v'ia una p'agina web o si decidimos que la comunicaci'on sea hecha v'ia sockets nuestro dide~no se puede adaptar facilmente, ya que solo hay que modificar la capa correspondiente.

\subsubsection{Ejemplos / Referencias a DS's}
Ejemplos de la interac'on de las distintas capas del cliente pueden verse en los DS's:




\clearpage

\section{\label{ModificacionesAlProtocolo}Modificaciones al protocolo}
A continuaci'on se detallan las extensiones al protocolo realizadas para soportar las funcionalidades referidas al manejo de pedido de reportes y modo dirigido para administradores desde una terminal de administrador (ver \ref{Clientes::Administradores}).

\subsection{Componente TVirt Admin$_i$} 
\subsubsection{Port AccesoYManejoAdministrador$_i$}

\subsubsection{Descripci'on}

Este protocolo corresponde a un port de un componente cliente de un Cliente/Servidor.

Se usa para que un usuario administrador pueda:
\begin{itemize}
    \item{pedir reportes}
    \item{configurar opciones del modo dirigido}
\end{itemize}

El servidor no deber'ia aceptar los diferentes pedidos admitidos por este protocolo si el usuario administrador no se autentica como tal proporcionando el password correcto en cada operaci'on.

Hemos factorizado en un mismo mensaje las dos posibles respuestas de parte del servidor en cuanto a la configuraci'on del casino en modo dirigido. Es importante destacar que estas respuestas son simplemente a modo informativo de que las configuraciones de modo dirigido fueron modificadas con 'exito (o no han podido ser modificadas). La aplicaci'on cliente no tendr'ia por qu'e modificar su estado en respuesta a estas notificaciones, salvo a lo sumo dar acuse de recibo al usuario por alg'un mecanismo (un messagebox por ejemplo). El 'unico caso en que el servidor deber'ia contestar en forma negativa estos pedidos es cuando el usuario que hace el pedido no fue autenticado satisfactoriamente por el servidor.

\subsubsection{Diagrama}
El siguiente diagrama describe los estados y transiciones que ocurren cuando se hace un uso esperado del protocolo.
Por la forma del protocolo dise~nado cualquiera podr'ia ser el estado inicial.

Se realizaron cambios con respecto a la notaci'on empleada en el documento del protocolo original para poder reducir la cantidad de flechas con el fin de que el diagrama sea m'as legible. A saber, se admiten flechas bidireccionales en el grafo dirigido, esto significa que puede recorrerse ese eje en ambas direcciones. En el documento original se utilizaban 2 flechas con direcci'on opuesta para representar esto.

\imagenvertical{FSM_portAccesoYManejoAdministrador.png}{Diagrama de estados}{0.4}

\clearpage
\subsubsection{Archivos de comunicaci'on}

\subsubsubsection{pedirReporteRankingDeJugadores}
\bf{Nombre del archivo:} \it{pedidoReporteRankingDeJugadoresXXYYYY.xml}
\begin{itemize}
    \item{XX: n'umero de grupo}
    \item{YYYY: n'umero de terminal virtual}
\end{itemize}

\bf{Descripci'on del xml:}
\begin{verbatim}
    <?xml version="1.0" encoding="UTF-8" ?>
    <pedidoReporteRankingDeJugadores vTerm="id terminal virtual" password="password de administrador" />
\end{verbatim}


\subsubsubsection{respuestaReporteRankingDeJugadores}
\bf{Nombre del archivo:} \it{respuestaReporteRankingDeJugadoresXXYYYY.xml}
\begin{itemize}
    \item{XX: n'umero de grupo}
    \item{YYYY: n'umero de terminal virtual}
\end{itemize}

\bf{Descripci'on del xml:}
\begin{verbatim}
<?xml version="1.0" encoding="UTF-8" ?>
    <respuestaReporteRankingDeJugadores vTerm="id terminal virtual">
        <aceptado fueAceptado="si | no" />
       [<jugadoresMasGanadores>
            <jugador ranking="posicion" nombre="nombre del jugador" />
            ...
        </jugadoresMasGanadores>
        <jugadoresMasPerdedores>
            <jugador ranking="posicion" nombre="nombre del jugador" />
            ...
        </jugadoresMasPerdedores>]
    </respuestaReporteRankingDeJugadores>
\end{verbatim}


\subsubsubsection{pedirReporteEstadoActual}
\bf{Nombre del archivo:} \it{pedidoReporteEstadoActualXXYYYY.xml}
\begin{itemize}
    \item{XX: n'umero de grupo}
    \item{YYYY: n'umero de terminal virtual}
\end{itemize}

\bf{Descripci'on del xml:}
\begin{verbatim}
    <?xml version="1.0" encoding="UTF-8" ?>
    <pedidoReporteEstadoActual vTerm="id terminal virtual" password="password de administrador" />
\end{verbatim}


\subsubsubsection{respuestaReporteEstadoActual}
\bf{Nombre del archivo:} \it{respuestaReporteEstadoActualXXYYYY.xml}
\begin{itemize}
    \item{XX: n'umero de grupo}
    \item{YYYY: n'umero de terminal virtual}
\end{itemize}

\bf{Descripci'on del xml:}
\begin{verbatim}
    <?xml version="1.0" encoding="UTF-8" ?>
    <respuestaReporteEstadoActual vTerm="id terminal virtual">
        <aceptado fueAceptado="si | no" />
       [<jugadores>
            <jugador nombre="nombre del usuario jugador" saldo="saldo del jugador" />
            ... 
        </jugadores>
        <observadores>
            <observador nombre="nombre del usuario observador" />
            ...
        </observadores>
        <juegos>
            <tragamonedas>
               <pozoProgresivo>"monto del pozo progresivo de las tragamonedas"</pozoProgresivo>
               <mesasTragamonedas>
                  <mesaTragamonedas id="id del tragamonedas">
                    <jugador>"nombre del jugador en la mesa"</jugador>
                    <ultimoTiro>
                      <tirador>"nombre del jugador que efectu'o el ultimo tiro"</tirador>
                      <resultado>
                       <rueda1>"cereza" | "barSimple" | "barDoble" | "barTriple" | "dinosaurio"</rueda1>
                       <rueda2>"cereza" | "barSimple" | "barDoble" | "barTriple" | "dinosaurio"</rueda2>
                       <rueda3>"cereza" | "barSimple" | "barDoble" | "barTriple" | "dinosaurio"</rueda3>
                      </resultado>
                    </ultimoTiro>
                  </mesaTragamonedas>
                   ...
               </mesasTragamonedas>
            </tragamonedas>
            <craps>
                <mesasCraps>
                    <mesaCraps id="id de mesa craps">
                        <jugadores>
                            <jugador>"nombre de un jugador en la mesa"</jugador>
                            ...
                        </jugadores>
                        <proximoTiro>
                            <tirador>"nombre del jugador con el turno" (si no ning'un pr'oximo tirador,
                            el tag estar'a vac'io) </tirador>
                            <tiroSalida>"Si" o "No", seg'un sea o no tiro de salida (si no hay pr'oximo
                            tirador, el tag estar'a vac'io) </tiroSalida>
                            <punto>"valor del punto establecido en el tiro de salida" (si es punto de
                            salida o no hay pr'oximo tirador, el tag estar'a vac'io) </punto>
                        </proximoTiro>
                        <ultimoTiro>
                            <tirador>"nombre del jugador que efectu'o el ultimo tiro"</tirador>
                            <resultado></resultado>
                        </ultimoTiro>
                    </mesaCraps>
                    ...
                </mesasCraps>
            </craps>
        </juegos>
        <pozosCasino>
            <pozoFeliz>"monto del pozo feliz"</pozoFeliz>
        </pozosCasino>
        <saldoCasino>"saldo del casino"</saldoCasino>]
    </respuestaReporteEstadoActual>
\end{verbatim}


\subsubsubsection{pedirReporteMovimientos}
\bf{Nombre del archivo:} \it{pedidoReporteMovimientosXXYYYY.xml}
\begin{itemize}
    \item{XX: n'umero de grupo}
    \item{YYYY: n'umero de terminal virtual}
\end{itemize}

\bf{Descripci'on del xml:}
\begin{verbatim}
    <?xml version="1.0" encoding="UTF-8" ?>
    <pedidoReporteMovimientos vTerm="id terminal virtual" password="password de administrador" />
\end{verbatim}


\subsubsubsection{respuestaReporteMovimientos}
\bf{Nombre del archivo:} \it{respuestaReporteMovimientosXXYYYY.xml}
\begin{itemize}
    \item{XX: n'umero de grupo}
    \item{YYYY: n'umero de terminal virtual}
\end{itemize}

\bf{Descripci'on del xml:}
\begin{verbatim}
    <?xml version="1.0" encoding="UTF-8" ?>
    <respuestaReporteMovimientos vTerm="id terminal virtual">
        <aceptado fueAceptado="si | no" />
       [<jugador nombre="nombre del jugador">
            <jugadasTragamonedas>
                <jugadaTragamonedas tipo="normal | todos ponen | feliz | gordo progresivo |
                feliz y gordo progresivo">
                    <mesa id="id de mesa" valorFicha="valor de ficha en la mesa" />
                    <apuesta fichas="cantidad de fichas apostadas" />
                    <resultado>
                        <rodillo figura="figura del rodillo" />
                        ...
                    </resultado>
                    <pozoFeliz monto="monto del pozo">
                    <pozoProgesivo monto="monto del pozo">
                </jugadaTragamonedas>
                ...
            </jugadasTragamonedas>
            <jugadasCraps tiroSalida="si | no" punto="valor de punto" tipo="normal | todos ponen |
            feliz | gordo progresivo | feliz y gordo progresivo">
                <jugadaCraps>
                    <mesa id="id de mesa" />
                    <apuesta monto="creditos apostados" tipo="en sitio a perder | en sitio a ganar |
                    de campo | venir | no venir | barra no pase | linea de pase" />
                    <resultado>
                        <dado numero="numero del dado">
                        ...
                    </resultado>
                    <pozoFeliz monto="monto del pozo">
                </jugadaCraps>
                ...
            </jugadasCraps>
            ...
        </jugador>
        ...]
    </respuestaReporteMovimientos>
\end{verbatim}


\subsubsubsection{configurarModoDirigidoTragamonedas}
\bf{Nombre del archivo:} \it{configurarModoDirigidoTragamonedasXXYYYY.xml}
\begin{itemize}
    \item{XX: n'umero de grupo}
    \item{YYYY: n'umero de terminal virtual}
\end{itemize}

\bf{Descripci'on del xml:}
\begin{verbatim}
    <?xml version="1.0" encoding="UTF-8" ?>
    <configurarModoDirigidoTragamonedas vTerm="id terminal virtual" password="password de administrador">
        <controlResultados activo="si | no">
           [<rueda1>"cereza" | "barSimple" | "barDoble" | "barTriple" | "dinosaurio"</rueda1>
            <rueda2>"cereza" | "barSimple" | "barDoble" | "barTriple" | "dinosaurio"</rueda2>
            <rueda3>"cereza" | "barSimple" | "barDoble" | "barTriple" | "dinosaurio"</rueda3>]
        </controlResultados>
        <controlTipoJugadas activo="si | no">
           [<tipo>"Normal" | "TodosPonen" | "NoNormal" | "NoTodosPonen" | "NoFeliz"</tipo>]
        </controlTipoJugadas>
    </configurarModoDirigidoTragamonedas>
\end{verbatim}


\subsubsubsection{respuestaConfigurarModoDirigidoTragamonedas}
\bf{Nombre del archivo:} \it{respuestaConfigurarModoDirigidoTragamonedasXXYYYY.xml}
\begin{itemize}
    \item{XX: n'umero de grupo}
    \item{YYYY: n'umero de terminal virtual}
\end{itemize}

\bf{Descripci'on del xml:}
\begin{verbatim}
    <?xml version="1.0" encoding="UTF-8" ?>
    <respuestaConfigurarModoDirigidoTragamonedas vTerm="id terminal virtual">
        <aceptado fueAceptado="si | no" />
        <descripcion>"informaci'on adicional, de por qu'e no fue aceptado por ejemplo u otra
        informacion relevante"</descripcion>
    </respuestaConfigurarModoDirigidoTragamonedas>
\end{verbatim}


\subsubsubsection{configurarModoDirigidoCraps}
\bf{Nombre del archivo:} \it{configurarModoDirigidoCrapsXXYYYY.xml}
\begin{itemize}
    \item{XX: n'umero de grupo}
    \item{YYYY: n'umero de terminal virtual}
\end{itemize}

\bf{Descripci'on del xml:}
\begin{verbatim}
    <?xml version="1.0" encoding="UTF-8" ?>
    <configurarModoDirigidoCraps vTerm="id terminal virtual" password="password de administrador">
        <controlResultados activo="si | no">
           [<dado1>"1" | "2" | "3" | "4" | "5" | "6"</dado1>
            <dado2>"1" | "2" | "3" | "4" | "5" | "6"</dado2>]
        </controlResultados>
        <controlTipoJugadas activo="si | no">
           [<tipo>"Normal" | "TodosPonen" | "NoNormal" | "NoTodosPonen" | "NoFeliz"</tipo>]
        </controlTipoJugadas>
    </configurarModoDirigidoCraps>
\end{verbatim}


\subsubsubsection{respuestaConfigurarModoDirigidoCraps}
\bf{Nombre del archivo:} \it{respuestaConfigurarModoDirigidoCrapsXXYYYY.xml}
\begin{itemize}
    \item{XX: n'umero de grupo}
    \item{YYYY: n'umero de terminal virtual}
\end{itemize}

\bf{Descripci'on del xml:}
\begin{verbatim}
    <?xml version="1.0" encoding="UTF-8" ?>
    <respuestaConfigurarModoDirigidoCraps vTerm="id terminal virtual">
        <aceptado fueAceptado="si | no" />
        <descripcion>"informaci'on adicional, de por qu'e no fue aceptado por ejemplo u otra
        informacion relevante"</descripcion>
    </respuestaConfigurarModoDirigidoCraps>
\end{verbatim}


\subsubsubsection{configurarModoDirigidoJugadaFeliz}
\bf{Nombre del archivo:} \it{configurarModoDirigidoJugadaFelizXXYYYY.xml}
\begin{itemize}
    \item{XX: n'umero de grupo}
    \item{YYYY: n'umero de terminal virtual}
\end{itemize}


\bf{Descripci'on del xml:}
\begin{verbatim}
    <?xml version="1.0" encoding="UTF-8" ?>
    <configurarModoDirigidoJugadaFeliz vTerm="id terminal virtual" password="password de administrador">
        <control activo="si | no">
           [<mesa>"id de la mesa"</mesa>]
        </control>
    </configurarModoDirigidoJugadaFeliz>
\end{verbatim}


\subsubsubsection{respuestaConfigurarModoDirigidoJugadaFeliz}
\bf{Nombre del archivo:} \it{respuestaConfigurarModoDirigidoJugadaFelizXXYYYY.xml}
\begin{itemize}
    \item{XX: n'umero de grupo}
    \item{YYYY: n'umero de terminal virtual}
\end{itemize}

\bf{Descripci'on del xml:}
\begin{verbatim}
    <?xml version="1.0" encoding="UTF-8" ?>
    <respuestaConfigurarModoDirigidoJugadaFeliz vTerm="id terminal virtual">
        <aceptado fueAceptado="si | no" />
        <descripcion>"informaci'on adicional, de por qu'e no fue aceptado por ejemplo u otra
        informacion relevante"</descripcion>
    </respuestaConfigurarModoDirigidoJugadaFeliz>
\end{verbatim}

\clearpage

% - Conclusiones.
\section{Conclusiones}
'Esta secci'on est'a dirigida al corrector del presente trabajo y ya no a los Timbalistas. Lo que deseamos es plasmar todos los problemas que tuvimos en su realizaci'on, experiencias aprendidas y, capaz, algunos de los momentos de p'anico. .

\subsection{Primeros conceptos: cualquiera}
Nadie en el grupo entr'o a la materia con s'olidos conceptos de dise'no. Los 'unicos que ten'iamos eran los brindados por las clases pr'acticas. Pero no es el mismo tipo de dise'no el que se realiza para un ejercicio que para un trabajo pr'actico. 'Esto nos llev'o a un primer dise'no (del servidor) en donde se agarr'o al modelo conceptual del TP1 y se trat'o de agregar operaciones y dependencias (a'un sabiendo que eso no estaba bien, pero no se nos ocurr'ia otra forma). De ah'i quedaron cosas como que la clase Jugador representaba a un jugador dentro del casino y a la vez representaba al jugador real (persona usuario del sistema) porque era el encargado de realizar las apuestas en una mesa. Adem'as el jugador era el que jugaba (o sea, invocaba la operaci'on de jugar sobre una mesa) porque, \italica{obviamente}, un jugador juega!

Luego nos dimos cuenta que en realidad existen otras entidades adicionales que no se ven en el modelo conceptual, clases que permiten la recepci'on de mensajes y ejecutan las acciones correspondientes para modificar el estado del (aunque todavia no lo llam'abamos as'i) Modelo. M'as adelante nos dimos cuenta que el modelo conceptual representa solamente a las estructuras y l'ogica de negocio del casino.


\subsection{Problemas con MVC}
Despu'es de darnos cuenta que deb'iamos establecer, como dise'no a gran escala, clases (o un conjunto de ellas) que faciliten la comunicaci'on con los clientes, otras que sepan leer los mensajes y actuar en funci'on de ello, otras que representen la funcionalidad innata del problema (l'ogica de negocio) y otras que generen las respuestas a los clientes. Algunos miembros del grupo ten'ian unas nocione del patr'on Model-View-Controller el cual es muy utilizado en aplicaciones web. Y, en principio, el servidor aparentaba un servidor web (que se maneja con XMLs en lugar de HTMLs) y los clientes, los navegadores web.

Para ello nos bas'amos en el MVC propuesto para Java (\cite{MVC-Java}). Pero lo comprendimos mal. El mayor error fue condiderar que en el Modelo solamente est'an las estructuras referentes a la l'ogica de negocio del problema pero nada de l'ogica (una especie de repositorio de datos espec'ificos del problema que queremos resolver). Y en el Controlador supusimos que deb'ia estar la l'ogica encargada de leer los pedidos y ejecutar ellos mismos la l'ogica de negocio, utilizando los datos que encuentra en el Modelo. Al principio parec'ia que iba a funcionar. Pero comezaron a surgir cosas que no nos cerraban:

\begin{itemize}
\item la cohesi'on entre la l'ogica y los datos era m'inima y no deber'ia ser as'i porque uno esta muy relacionado al otro.
\item el controlador realizaba las tareas de l'ogica del casino y a la vez el parseo de los XML.
\item ten'iamos que si o si utilizar condicionales hasta para la menor decisi'on posible (eso es porque no pod'iamos utilizar el \italica{strategy}, ya que estar'iamos dandole l'ogica de negocio al Modelo, y eso nos parecia mal)
\end{itemize}

Despu'es de estar semanas bajo 'este paradigma, lleg'o un punto que todos 'estos problemas eran intratables. All'i nos dimos cuenta de nuestro error. Entramos en p'anico. No pudimos continuar hasta no consultar c'omo realmente utilizar MVC. El corrector nos dijo que el verdadero MVC no era soportado por el protocolo por no tener mensajes que le permitan al Modelo actualizar a la Vista. Y nos dijo y repiti'o que no usemos MVC. Entonces adoptamos el diagrama actual, en donde el MensajeroDeSalida (ex Controlador) s'olo sabe parsear los XML y toda la l'ogica la realiza el Modelo. 

Realmente todo lo que tuvimos que pensar y debatir en grupo bas'andonos en un paradigma err'oneo nos permiti'o darnos cuenta porque es que el enfoque no serv'ia, y porque se debe pegar la logica a la representaci'on si queremos tomar un enfoque orientado a objetos (capaz que con un repositorio de datos basado en, por ejemplo, una base de datos relacional el enfoque cambiar'ia porque las ventajas y desventajas comparadas con la orientaci'on a objetos son diferentes).


\subsection{?`C'omo y d'onde hacer el ``dynamic cast''?}
Algo que nos carcomi'o mucho la cabeza fue c'omo darle sentido a los valores de un pedido (un XML, el cual puede ser considerado un simple texto) y as'i saber de que mensaje se trata para saber como tratarlo, al igual que qui'en es el encargado de hacer el \italica{parseo} de texto al valor que ya sabe debe ser. Estabamos hablando, si se quiere usar un nombre m'as lindo, sobre donde y como hacer el \italica{dynamic cast}. A medida que fuimos d'andole forma al dise'no, nos tuvimos que ir tomando la decisi'on varias veces, siempre tratando que sea lo m'as transparente posible y que sea en un s'olo lugar. Terminamos convergiendo a que sea toda tarea del MensajeroDeEntrada: el DespachadorPedidos realiza el primer paso eligiendo quien es el responsable de manejar el pedido que acaba de llegar, y luego cada manejador hace el segundo paso al saber como recorrer el XML y comprendiendo que valor representa el texto que le llega (en realidad s'olo hablando de tipos, por ejemplo puede saber que un IdMesa es un Entero pero no comprende que es un valor que representa a una mesa porque es l'ogica s'olo del Modelo). 

El enfoque de dejar todo 'esto en un s'olo lugar trae como consecuencia (una consecuencia justamente buscada) que si el protocolo cambia, ya sea la estructura (se pasa de XML a otra estructura) o la sintaxis de los mensajes, 'estos cambios s'olo afectar'an al MensajeroDeEntrada. Lo cual va de la mano al Single Responsability Principle.


\subsection{Poco grado de paralelismo en el grupo}
Ya en la realizaci'on del TP1 nos dimos cuenta que el enfoque de 'estos trabajos pr'acticos no era igual a los de otras materias. Aqu'i no s'olo hay que tomar muchas m'as decisiones si no que adem'as las posibilidades que se nos ocurr'ian eran muchas. Y todas parec'ian igual de v'alidas y \italica{buenas}. Eso generaba que todo el grupo tenga que estar reunido haciendo de a una cosa a la vez, para poder debatir entre nosotros. Y la consecuencia fue que los avances eran de a peque'nos pasos, por no poder hacer mucho por separado (y encima no es f'acil tener a todo el grupo reunido en todo momento). Pero igualmente, se lleg'o a un punto cuando ya todos comprend'iamos bastante m'as del tema, en donde pudimos separarnos los diagramas y redacci'on del informe. 

Pero para el TP2 'este cuello de botella se increment'o. Aqu'i la 'unica divisi'on posible, a priori, fue o hacer el DC o hacer los DS. Los segundos no se pod'ian hacer si no se ten'ia el primero ya bien definido. Y el DC es at'omico: ten'iamos que estar de nuevo todos juntos deliberando sobre el mejor criterio de dise'no a utilizar y que es mas \italica{lindo} y que es m'as \italica{feo}. 'Esto gener'o que ahora los avances fuesen a paso de hormiga. Y encima, la utilizaci'on err'onea del MVC junto al cambio de paradigma nos consumi'o mucho tiempo. Pero gracias a las sucesivas extensiones que nos brindaron, llegamos a terminar todo lo que quer'iamos hacer.


\subsection{Conclusiones finales}
Entre todos creemos que 'este fue el trabajo pr'actico m'as largo en lo que llevamos de la carrera. Creemos que las cosas las fuimos haciendo en forma lenta, pero la verdad que de tenerlo que hacer nuevamente utilizar'iamos las mismas metodolog'ias. Capaz que nustra forma de trabajo no es eficiente y de ah'i la tardanza. Capaz le dedicamos mucho tiempo a pensar en la \italica{mejor} soluci'on cuando el objetivo del trabajo no era para tanto. No lo sabemos. Sea como sea, logramos terminarlo :)

\clearpage


\end{document}
