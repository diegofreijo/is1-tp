\documentclass[spanish, a4paper, 10pt, titlepage]{article}
\author{Echevarr'ia - Farjat - Freijo - Giusto}

% Incluyo los paquetes y configuraciones
% Archivo de configuracion del informe
% -------------------------------------------

\usepackage[spanish,activeacute]{babel}								% Idioma castellano
\usepackage{caratula}														% Caratula de Algo2
%\usepackage[a4paper=true,pagebackref=true]{hyperref}				% Agrega la TOC al PDF e hipervinculos
\usepackage{graphicx} 														% Permite insertar graficos
\usepackage{fancyhdr}														% Permite manejo de cabeceras de pagina
\usepackage{eufrak}															% Usado en el enunciado del trabajo
\usepackage{latexsym}
%\usepackage{algorithmic}													% Para escribir los algos
%\usepackage{dsfont}															% Para el simbolo de naturales
\usepackage[font=small,labelfont=bf]{caption}						% Para editar las captions
%\usepackage{array}

\usepackage[utf8]{inputenc}
\usepackage{amsmath}
\usepackage[x11names, rgb]{xcolor}
% \usepackage{tikz}
% 	\usetikzlibrary{snakes,arrows,shapes}
\usepackage{listings}
\usepackage{lastpage}
\usepackage{geometry}
  \geometry{left=1cm, right=1cm, top=2cm, bottom=2cm}

% Estilo de pagina para tener las cabeceras y pieseras
\pagestyle{fancy}
  \fancyhead[LO]{Ingenier\'ia de Software I}
  \fancyhead[C]{Trabajo Práctico}
  \fancyhead[RO]{Primer Cuatrimestre 2008}
  \renewcommand{\headrulewidth}{0.4pt}

  \fancyfoot[LO]{Echevarr\'ia - Farjat - Freijo - Giusto}
  \fancyfoot[C]{}
  \fancyfoot[RO]{P\'agina \thepage\ de \pageref{LastPage}}
  \renewcommand{\footrulewidth}{0.4pt}

\parindent = 1.5 em 
\parskip = 8 pt


%%%%%%%%%%%%%%%% COMANDOS %%%%%%%%%%%%%%%%%%%%
\newcommand{\todo}{{\large\textbf{TODO: }}}
\newcommand{\paso}{\textsc{Paso }}
\newcommand{\func}[1]{\verb"#1"}
\newcommand{\imagen}[3]
{
	\begin{figure}[p!hbt]
	  \centering
	    \includegraphics[scale=#3]{../img/#1}
	  \caption{#2}
	\end{figure}
}
\newcommand{\imagenvertical}[3]
{
	\begin{figure}[p!hbt]
	  \centering
	    \includegraphics[angle=90,scale=#3]{../img/#1}
	  \caption{#2}
	\end{figure}
}

\newcommand{\nat}{\mathds{N}}
\newcommand{\algoritmo}[3]{\noindent {\bf\underline{#1}:} #2 $\longrightarrow$ #3}
\newcommand{\superindice}[1]{$^\textrm{{\tiny #1}}$}
\newcommand{\subsubsubsection}[1]{\noindent\negrita{#1}

}
\newcommand{\negrita}[1]{{\bf #1}}
\newcommand{\italica}[1]{{\it #1}}

% -----------------------------------------------------------------------------------------------------
% Codigo para generar los casos de uso, basado en caratula.sty del DC-FCEN-UBA 
% Autor: Nicolas Rosner
% Modificado por Pablo Echevarria 25-5-08
% ToDo: ver como declarar primero titulo y despues operaciones, ver como numerar automaticamente las operaciones
% -----------------------------------------------------------------------------------------------------
% Token list para las instrucciones ----
\newtoks\oplist\oplist={}

% Comando para que el usuario agregue operaciones del CU
% Uso: \op{Caso normal}{Caso alternativo}
\newcommand{\op}[2]{\oplist=\expandafter{\the\oplist
\hline#1&#2\\ }}

% Comando para generar el CU con las operaciones ya pasadas y dandole la info que falta
% Uso: \cu{Nombre CU}{Actor primario}{Actores secundarios}{Precondicion}{Postcondicion}{Aclaraciones generales a todos los pasos}
\newcommand{\cu}[6]{
\begin{center}	
\begin{tabular}{|p{11cm}|p{5cm}|}
\hline
\multicolumn{2}{|l|}{\begin{large}\italica{\negrita{CASO DE USO:} #1}\end{large}}\\[0.2em]
\hline
\multicolumn{2}{|l|}{\negrita{Actor Primario:} #2}\\[0.2em]
\hline
\multicolumn{2}{|l|}{\negrita{Actor Secundario:} #3 }\\[0.2em]
\hline
\multicolumn{2}{|l|}{\negrita{Precondici'on:} #4 }\\[0.2em]
\hline
\multicolumn{2}{|l|}{\negrita{Postcondici'on:} #5}\\[0.2em]
\hline
\multicolumn{2}{|l|}{}\\[0.2em]
\hline
\negrita{Curso Normal} & \negrita{Curso Alternativo}\\[0.2em]
\the\oplist
\hline
\multicolumn{2}{|l|}{}\\[0.2em]
\hline
%\multicolumn{2}{|1|}{#6}\\[0.2em]
%\hline
\end{tabular}
\end{center}
\oplist={}
}




%%%%%%%%%%%%%%%% FIN COMANDOS %%%%%%%%%%%%%%%%%%%%


%%%%%%%%%%%%%%%%%%%%%%%%%%%%%%%%%%%%%%%%%%%%%%%%%%%%%%%%%%%%%%
%%%%%%%%%%%%%%%%%%%%%%%%%%%%%%%%%%%%%%%%%%%%%%%%%%%%%%%%%%%%%%%%%%%%%%%%%%%%%%%%%%%%
%%%%%   Inicio del documento
%%%%%%%%%%%%%%%%%%%%%%%%%%%%%%%%%%%%%%%%%%%%%%%%%%%%%%%%%%%%%%%%%%%%%%%%%%%%%%%%%%%%
%%%%%%%%%%%%%%%%%%%%%%%%%%%%%%%%%%%%%%%%%%%%%%%%%%%%%%%%%%%%%%
\begin{document}

% Caratula y tabla de contenidos
\materia{Ingenier'ia de Software I}
\submateria{Primer Cuatrimestre 2008}
\titulo{Trabajo Pr'actico - Parte III}
\subtitulo{Implementaci'on}
\grupo{Grupo 5}

	\integrante{Echevarria, Pablo}{133/00}{pablohe@gmail.com}
	\integrante{Farjat, Lucas}{468/05}{lacacks@gmail.com}
	\integrante{Freijo, Diego}{4/05}{giga.freijo@gmail.com}
	\integrante{Giusto, Maximiliano}{486/05}{maxi.giusto@gmail.com}

\maketitle
\clearpage



\tableofcontents
\clearpage

% ------------------------------------------------------
% Secciones
% ------------------------------------------------------


% Introduccion
\section{Introducci'on}
% Introduccion. Cambios con respecto al informe 1.

En esta secci'on describiremos los cambios con respecto al informe 1.


\clearpage
 
% Escenarios
\section{Escenarios y Diagramas de Secuencia}
Para la construcci'on de los escenarios y diagramas de secuencia fue basada en el echo de que contabamos con los mensajes de protocolo los cuales inducen interaci'on y modificaci'on de nuestro modelo.

Tambi'en decidimos hacer que nuestros escenarios sean lo m'as gen'ericos posible siempre y cuando esto no complique la lectura. 

En cuanto a la profundidad para algunos DS mostramos la interacci'on de punta 
a punta estos se encuentran en la secci'on del mismo nombre.
Factorizandolos en 2 o 3 DS's cada uno. El resto de los diagramas, salvo excepciones, comienzan con la llamada al m'etodo de la fachada del modelo.

Dado que la secci'on de recepci'on de pedidos y el despacho es muy parecida en estos diagramas 
decidimos obviarlos (salvo, claro, en los de punta a punta).


Por simpleza no usamos los \textit{ObtenerInstancia} de los Singletons. Estos estar'an implicitos.


Para tratar con los XML asumiremos que contamos con cierta funcionalidad, provista por los objetos, por ejemplo:
\begin{itemize}
\item  Para obtener el atributo \textbf{idMesa} de un XML usamos \textit{XML.idMesa}, donde idMesa es el tag que se encuentra en el XML, y  \textit{XML.idMesa}, nos devoveria un entero almacenado en ese tag.

\item As'i tambi'en para la lista de jugadores, \textit{XML.jugadoresEnMesa} devuelve una \textbf{lista$<$jugadores$>$}.
\end{itemize}

% 
% \escenario{ \textbf{Startup del Casino}
% \begin{itemize}
% \item se inicia el servidor
% \item se leen los achivos de configuraci'on
% \end{itemize}
% 
% }
% 
% \escenario{ \textbf{Startup Cliente}
% \begin{itemize}
% \item se incia el cliente
% \item ???????????????????????????????????
% \end{itemize}
% 
% }

\subsection{Diagramas de punta a punta}

En esta secci'on mostraremos DS desde que se leen los arhivos XML hasta que el mensajero de salida despacha los archivos XML de respuesta.

\subsubsection{Craps}




\textbf{Tirar Dados}


Este escenario es bastante gen'erico,

\escenario{
      \begin{itemize}
      \item El usuario puede o no estar en la mesa.
      \item El usuario puede o no ser el tirador.
      \item En caso de que sea el tirador y est'e en la mesa.
      \item Se tiran los dados.
      \item Si sali'o alg'un valor propio de punto este se setea.
      \item Hay alguna cantidad de apuestas de alg'un tipo que se resuelven o no con su l'ogica particular, dependiendo del contexto %$^1$
      \item Estamos en un ``est'an saliendo''% $^3$
      \item Es una jugada feliz %$^2$
    \end{itemize}
}

% 
% $^1$ Resoluci'on de distintas apuestas se ve en otros DS's (en sitio a perder y a venir)
% $^2$ En otro DS se ve la situacion con el caso de que el punto esté establecido
% $^3$ En otro DS se ve la jugada Todos ponen

El pozo feliz se reparte si o si el servidor de Jugadas no devuelve una feliz mientras el pozo no lleque al m'inimo.
La notificaci'on a de que una mesa cambió se pude ver en apostar, sin bien est'a instanciado en una mesa, aplica si la mesa fuera gen'erica.


Este DS se dividió en 3 secciones:
\begin{enumerate}
 \item Recepcion de pedido: es la recepcion de pedido,  y la respuesta hacia el modulo de comunciación, no se muestra lo que sucede en la llamada TirarCraps (usuario, unXML)
\item  TirarCraps: Hace todo lo concerniente a la validaci'on, no se hace zoom en TirarDados.
\item TirarDados: aqui puede verse lo que pasa cuando se hace un tirar dados de una mesa
\end{enumerate}

% \subsubsection{Tragamonedas}
% 


% 
% \subsection{Casino}
% %  \escenario{Entrar Casino}{
Un usuario desea entrar al casino, puede hacerlo en modo jugador o en modo observador.
Si ya ha ingresado en modo jugador no se lo dejar'a entrar nuevamente. Si est'a en modo observador y desea ingresar en el mismo modo tampoco podr'a hacerlo.

En cambio si quiere entrar como jugador (independientemente de si ingres'o como observador o si no ingres'o) se deber'a validar que sea un usuario autorizado por marketing:

\begin{itemize}
 \item En caso afirmativo quedar'a ingresado en modo jugador.
 \item En caso negativo quedar'a en modo observador o fuera del casino seg'un cual fuese su estado anterior.
\end{itemize}
 }
\imagen{DS_Casino/EntarCasino/DS_EntrarCasinoFueraDelModelo.png}{Entrar casino fuera del Modelo}{0.6}
%aca iria una imagen------------------------------------------------------------
Entrar casino dentro del Modelo \tam

\clearpage



%aca iria una imagen------------------------------------------------------------

\clearpage

\textbf{Pedir Estado Casino}

Un usuario desea informarse sobre el estado del casino. Se le informar'a s'olo si ha ingresado en el casino, m'as haya si es en modo jugador o en modo observador.

El estado del casino est'a formado por:

\begin{itemize}
 \item La lista de jugadores y observadores ingresados en el casino;
 \item El valor del pozo feliz y del pozo progresivo;
 \item El estado de las mesas del juegos de craps:
	\begin{itemize}
	 \item Los jugadores;
	 \item El 'ultimo tirador y el pr'oximo;
	 \item Si el siguiente es tiro de salida o ya est'a el punto establecido;
	 \item El valor de los dados en el 'ultimo tiro;
	\end{itemize}
 \item El estado de las mesas del juego tragamonedas:
	\begin{itemize}
 	 \item Los jugadores;
 	 \item El valor de los rodillos en el 'ultimo tiro;
 	 \item El 'ultimo tirador y el pr'oximo;
	\end{itemize}
\end{itemize}
\tam

%aca iria una imagen------------------------------------------------------------

\clearpage

\textbf{Salir Casino}

Usuario desea salir del casino.

Si ha ingresado como observador no tendr'a ning'un tipo de validaci'on, por consiguiente tampoco problemas.

Si ha ingresado como jugador, para poder salir deber'a estar fuera de toda mesa. Es decir, no puede pretender salir del casino si es que est'a dentro de una mesa jugando.

\imagen{DS_Casino/SalirCasino/DS_SalirCasino.png}{Salir del Casino}{0.6}

%aca iria una imagen------------------------------------------------------------




% 
% \subsection{Diagr'amas de que no son de punta a punta}
% Dado que la secci'on de recepci'on de pedidos y el despacho es muy parecida en estos diagramas decidimos obviarlo.
% 
% \subsubsection{Funcionalidades de inicializaci'on}
% Los escenarios aqui presentados son muy gen'ericos.


\escenario{ la Configuraci'on general del Casino}
{
Se setea el valor de las fichas, el saldo del casino y la pasword del aministrador
}
% imagen

\imagen{DS_InicioServidor/DS_InicializarConfiguracion.png}{Inicializar Configuraci'on}{0.5}

\escenario{Jugadores Registrados}{
En este DS se ve como se setea la lista de jugadores registrados del casino
}
\imagenvertical{DS_InicioServidor/DS_InicializarJugadoresRegistrados.png}{Inicializar Jugadores Registrados}{0.4}

% imagen


\escenario{Inicializar Mesas}{
En este DS puede verse como se inicalizan las mesas abiertas, se asigna el observador de cambios.
}
\imagen{DS_InicioServidor/DS_InicializarMesas.png}{Inicializar Mesas}{0.5}
% imagen


\escenario{Inicio del Servidor}{
En este DS se puede ver como se ``enciende'' el servidor,
como se crea el el Obtenerdor de pedidos, el receptor de pedidos de archivos,  

}
\imagen{DS_InicioServidor/DS_InicioServidor.png}{Inicio Servidor}{0.5}

% 
% \subsubsection{Funcionalidades generales de los administradores}
% \escenario{Modo Dirigido Craps} { 
Puede verse la secuencia de un seteo de el modo dirigido de craps, donde se setea alg'un valor de cada dado
y que es una jugada \textit{todosPonen}.}
\tam
% \imagen{DS_Admin/ConfigurarModoDirigidoCraps.png}{Modo Dirigido Craps}{0.4}


\escenario{ Modo Dirigido setear JugadaFeliz }{
Puede verse la secuencia de un seteo de el modo dirigido para la jugada Feliz.
}
\tam


\escenario{Pedir Reporte Ranking de Jugadores}{
Un administrador pide el reporte de ranking de jugadores el cual le informa cu'ales son los jugadores m'as ganadores y perdedores en lo que va del d'ia.

Para poder recibirlo debe introducir la password correcta.
}

\imagen{DS_Admin/PedirReporte/DS_PedidoReporteRankingDeJugadores.png}{Validaci'on de la password}{0.5}

\imagen{DS_Admin/Pedir Reporte/DS_RespuestaReporteRankingDeJugadores.png}{Armado del reporte}{0.5}


\escenario{Pedir Reporte de Movimientos}{
Un administrador pide el reporte de todos los movimientos por jugador (apuestas,
premios ganados, monto ganado) desde que ingresaron al casino.

Para poder recibirlo debe introducir la password correcta.
}

\imagen{DS_Admin/PedirReporte/DS_RespuestaReporteMovimientos.png}{Armado del reporte}{0.5}


\escenario{Pedir Estado Actual}{
Un administrador pide el reporte de movimientos el cual le informa cu'ales es el saldo de los jugadores y el saldo del casino.

Para poder recibirlo debe introducir la password correcta.
}

\imagen{DS_Admin/PedirReporte/DS_RespuestaReporteEstadoActual.png}{Armado del reporte}{0.5}



% 
% \subsubsection{Funcionalidades generales del casino}
%  \escenario{Entrar Casino}{
Un usuario desea entrar al casino, puede hacerlo en modo jugador o en modo observador.
Si ya ha ingresado en modo jugador no se lo dejar'a entrar nuevamente. Si est'a en modo observador y desea ingresar en el mismo modo tampoco podr'a hacerlo.

En cambio si quiere entrar como jugador (independientemente de si ingres'o como observador o si no ingres'o) se deber'a validar que sea un usuario autorizado por marketing:

\begin{itemize}
 \item En caso afirmativo quedar'a ingresado en modo jugador.
 \item En caso negativo quedar'a en modo observador o fuera del casino seg'un cual fuese su estado anterior.
\end{itemize}
 }
\imagen{DS_Casino/EntarCasino/DS_EntrarCasinoFueraDelModelo.png}{Entrar casino fuera del Modelo}{0.6}
%aca iria una imagen------------------------------------------------------------
Entrar casino dentro del Modelo \tam

\clearpage



%aca iria una imagen------------------------------------------------------------

\clearpage

\textbf{Pedir Estado Casino}

Un usuario desea informarse sobre el estado del casino. Se le informar'a s'olo si ha ingresado en el casino, m'as haya si es en modo jugador o en modo observador.

El estado del casino est'a formado por:

\begin{itemize}
 \item La lista de jugadores y observadores ingresados en el casino;
 \item El valor del pozo feliz y del pozo progresivo;
 \item El estado de las mesas del juegos de craps:
	\begin{itemize}
	 \item Los jugadores;
	 \item El 'ultimo tirador y el pr'oximo;
	 \item Si el siguiente es tiro de salida o ya est'a el punto establecido;
	 \item El valor de los dados en el 'ultimo tiro;
	\end{itemize}
 \item El estado de las mesas del juego tragamonedas:
	\begin{itemize}
 	 \item Los jugadores;
 	 \item El valor de los rodillos en el 'ultimo tiro;
 	 \item El 'ultimo tirador y el pr'oximo;
	\end{itemize}
\end{itemize}
\tam

%aca iria una imagen------------------------------------------------------------

\clearpage

\textbf{Salir Casino}

Usuario desea salir del casino.

Si ha ingresado como observador no tendr'a ning'un tipo de validaci'on, por consiguiente tampoco problemas.

Si ha ingresado como jugador, para poder salir deber'a estar fuera de toda mesa. Es decir, no puede pretender salir del casino si es que est'a dentro de una mesa jugando.

\imagen{DS_Casino/SalirCasino/DS_SalirCasino.png}{Salir del Casino}{0.6}

%aca iria una imagen------------------------------------------------------------




% 
% \subsubsection{Tragamonedas}
% \escenario
{
\begin{itemize}
 \item Un jugador con una mesa elegida para jugar
\item  inserta una cantidad v'alida de fichas en una m'aquina tragamonedas
\item  gira los rodillos. 
\item 'Esta, luego de debitar el monto establecido con aterioridad correspondiente al pozo progresivo, consulta el tipo de jugada al casino 
\item le responde que es una jugada normal
\item la m'aquina genera el resultado de la jugada en base a las probabilidades que le establecieron. 
\item 'Esta termina arrojando un resultado ganador 
\item el jugador observa el resultado y tipo de jugada 
\item recibe el cobro.

\end{itemize}

}

\escenario
{
\begin{itemize}
  \item Un jugador con una mesa elegida para jugar inserta una cantidad v'alida de fichas en una m'aquina tragamonedas 
  \item gira los rodillos. 
  \item debita el monto establecido con aterioridad correspondiente al pozo progresivo
  \item consulta el tipo de jugada al casino
  \item le responde que es una jugada normal
  \item la m'aquina genera el resultado de la jugada en base a las probabilidades que le establecieron
  \item termina arrojando un resultado perdedor
  \item el jugador observa el resultado y tipo de jugada.
\end{itemize}
}

\escenario
{
Un jugador con una mesa elegida para jugar inserta una cantidad v'alida de fichas en una m'aquina tragamonedas y gira los rodillos. 'Esta, luego de debitar el monto establecido con aterioridad correspondiente al pozo progresivo, consulta el tipo de jugada al casino y le responde que es una jugada todos ponen. Luego la m'aquina genera el resultado de la jugada en base a las probabilidades que le establecieron. 'Esta termina arrojando un resultado ganador para el jugador qui'en luego observa el resultado y tipo de jugada, recibe el cobro y la m'aquina debita en el pozo feliz el importe correspondiente.
}

\escenario
{
Un jugador con una mesa elegida para jugar inserta una cantidad v'alida de fichas en una m'aquina tragamonedas y gira los rodillos. 'Esta, luego de debitar el monto establecido con aterioridad correspondiente al pozo progresivo, consulta el tipo de jugada al casino y le responde que es una jugada feliz. Luego la m'aquina genera el resultado de la jugada en base a las probabilidades que le establecieron. 'Esta termina arrojando un resultado ganador para el jugador. 'Este observa el resultado, el tipo de jugada y luego cobra la suma entre el pago por la apuesta ganada y el pozo feliz hasta ese momento. Luego el pozo feliz es reiniciado a su valor inicial. 
}

\escenario
{
Un jugador con una mesa elegida para jugar inserta la m'axima cantidad de fichas en una m'aquina tragamonedas luego de jugar en ella el mismo importe tantas veces como las necesarias para que pueda ganar el pozo progresivo y gira los rodillos. 'Esta, luego de debitar el monto establecido con aterioridad correspondiente al pozo progresivo, consulta el tipo de jugada al casino y le responde que es una jugada de pozo progresivo. Luego la m'aquina genera el resultado de la jugada en base a las probabilidades que le establecieron. 'Esta termina arrojando el resultado con pago m'aximo. El jugador observa el resultado y tipo de jugada. Luego cobra la suma entre el pago por la apuesta ganada y el monto total del pozo progresivo. Luego el pozo progresivo es reiniciado a su valor inicial.
}

% 
% 
% \subsubsection{Craps}
% \escenario{ Entrar Craps}{
Este escenario es bastante gen'erico. Se muestra como se valida cada cosa, como actua el sistema en cada caso
y que mensaje de error da.

El usuario puede o no estar en el casino en modo jugador.(incluye modo observador o no haber ingresado)
Puede estar en otra mesa o puede desear crearla.
}
\tam

\clearpage
\escenario{Resolverse Apuesta de Sitio a Ganar}{
La ronda esta en ``Est'an Saliendo'' sali'o un 4. La apuesta se resuelve, pasa a estar cerrada
}
% imagen
\imagenvertical{DS_Craps/CrapsResolverse_ApuestaDeSitioaGanar.png}{Resolverse Apuesta de Sitio a Ganar}{0.5}

\clearpage
\escenario{Resolverse Apuesta Venir}{
Se estableci'o el punto. Se le paga. La apuesta se cierra
}
% imagen

\imagenvertical{DS_Craps/CrapsResolverse_AuestaVenir.png}{Resolverse Apuesta Venir}{0.5}

\escenario{ \textbf{Jugador de Craps haciendo una apuesta} }{
     El usuario esta en una mesa de craps. Elige un valor de ficha de \$20
  Elige un valor de ficha de \$15. Elige el tipo de apuestatem por cada eleccion se ve un mensaje en el log. Si elige una apuesta antes de una ficha  da un error.

}
% imagen


% 
% 

\clearpage

% - Pseudocodigo de operaciones mas complicadas en lo algoritmico.
% \section{Pseudoc'odigo de operaciones m'as complicadas en lo algoritmo.}
% \subsubsubsection{ObtenerJugadoresMasGanadores(): Lista<Nombre>}

'Este m'etodo pertenece a la clase \italica{AdministradorDeCasino}. Devuelve la lista ordenada en forma descendente de los jugadores que m'as dinero han ganado en el casino durante el presente d'ia.

\begin{verbatim}
AdministradorDeCasino::ObtenerJugadoresMasGanadores(): Lista<Nombre>
	
	jugadas = elHistorialDeJugadas.GetJugadas()
	jugadasTragamonedas = elHistorialDeJugadas.GetJugadasTragamonedas()
	premios = jugadas.GetPremios()
	premiosTragamonedas = jugadasTragamonedas.GetPremio()

	Para cada elemento de premios hacer
		gano = monto_normal + monto_feliz - monto_todosponen - monto_apostado
		Si al jugador asociado es nuevo (no aparecio en una iteracion anterior)
			se le guarda este valor
		si no
			se le suma al valor que ya tenia
		Fin si
	Fin para

Para cada elemento de premiosTragamonedas hacer
		gano = monto_normal + monto_feliz - monto_todosponen - monto_apostado + monto_progresivo
		Si al jugador asociado es nuevo (no aparecio en una iteracion anterior)
			se le guarda este valor
		si no
			se le suma al valor que ya tenia
		Fin si
	Fin para

	De los jugadores que menor valor tienen en gano se toma a los 3 primeros y se los retorna
\end{verbatim}


\subsubsubsection{ObtenerJugadoresMasPerdedores(): Lista<Nombre>}

'Este m'etodo pertenece a la clase \italica{AdministradorDeCasino}. Devuelve la lista ordenada en forma descendente de los jugadores que m'as dinero han perdido en el casino durante el presente d'ia.

\begin{verbatim}
AdministradorDeCasino::ObtenerJugadoresMasPerdedores(): Lista<Nombre>
	
	jugadas = elHistorialDeJugadas.GetJugadas()
	jugadasTragamonedas = elHistorialDeJugadas.GetJugadasTragamonedas()
	premios = jugadas.GetPremios()
	premiosTragamonedas = jugadasTragamonedas.GetPremio()

	Para cada elemento de premios hacer
		perdio = monto_normal + monto_feliz - monto_todosponen - monto_apostado
		Si al jugador asociado es nuevo (no aparecio en una iteracion anterior)
			se le guarda este valor
		si no
			se le suma al valor que ya tenia
		Fin si
	Fin para

	Para cada elemento de premiosTragamonedas hacer
		perdio = monto_normal + monto_feliz - monto_todosponen - monto_apostado + monto_progresivo
		Si al jugador asociado es nuevo (no aparecio en una iteracion anterior)
			se le guarda este valor
		si no
			se le suma al valor que ya tenia
		Fin si
	Fin para

	De los jugadores que menor valor tienen en perdio se toma a los 3 primeros y se los retorna
\end{verbatim}


\subsubsubsection{DetalleMovimientoJugadores(): Coleccion<Tupla<Nombre, Texto, Creditos, Creditos, Creditos, Creditos>>}

'Este m'etodo pertenece a la clase \italica{AdministradorDeCasino}. Devuelve la una coleccion con los jugadores sus apuestas y los premios ganados y perdidos durante el presente d'ia.

\begin{verbatim}
AdministradorDeCasino::DetalleMovimientoJugadores(): Coleccion<Tupla<Nombre, Texto, Creditos, Creditos, Creditos, Creditos>>

	jugadas = elHistorialDeJugadas.GetJugadas()
	jugadasTragamonedas = elHistorialDeJugadas.GetJugadasTragamonedas()
	premios = jugadas.GetPremios()
	premiosTragamonedas = jugadasTragamonedas.GetPremio()

	Para cada elemento p de premios hacer
		unaTupla = Tupla(p.apostador, p.nombre_tipo_apuesta, p.monto_normal, p.monto_feliz, p.monto_todosponen, Null)
		laListaDeRetorno.Agregar(unaTupla)
	Fin para

	Para cada elemento p de premiosTragamonedas hacer
		unaTupla = Tupla(p.apostador, p.nombre_tipo_apuesta, p.monto_normal, p.monto_feliz, p.monto_todosponen, p.montoProgresivo)
		laListaDeRetorno.Agregar(unaTupla)
	Fin para

	Retornar laListaDeRetorno
\end{verbatim}

% \clearpage

% Diagrama de clases
\section{Diagrama de clases}
\subsection{Introducci'on}
El m'odulo del casino se puede separar en dos partes: 

\begin{itemize}
\item componentes \italica{Cliente}
\item componente \italica{Servidor}
\end{itemize}  

La comunicaci'on entre el componente {\it Servidor} y los componentes {\it Cliente} ya fue resuelta por el documento de Arquitectura Conceptual y Protocolo brindado por los clientes (ver documento adjunto) y por las extensiones realizadas al mismo (ver \ref{ModificacionesAlProtocolo}).

No existir'a comunicaci'on entre los componentes {\it Cliente} desde el punto de vista arquitectural, aunque s'i se relacionar'an indirectamente a trav'es de la l'ogica de negocio del servidor.


\subsection{Servidor}
\subsection{Gr'afico Diagrama de Clases}

El enfoque en este diagrama es sobre el \italica{Servidor}.

El grafico del modelo va en formato digital

% A continuaci'on tenemos el gr'afico del diagrama de clases para la resoluci'on del problema sobre el casino.
% 

% y
% \imagenvertical{DC_DiagramaCompleto_vista_control.png}{Diagrama de Clases}{0.18}
% 
% \imagenvertical{DC_DiagramaCompleto_modelo.png}{Diagrama de Clases}{0.18}

\clearpage

\subsection{Explicaci'on}
Para facilitar el dise'no, mantener un nivel bajo de acoplamiento e incluso fomentar la reutilizaci'on de los elementos utilizados en el diagrama, se agruparon ciertas clases en m'odulos. Cada m'odulo define una 'unica responsabilidad y las clases que lo contienen deben respetarla y ejecutarla.

A continuaci'on tenemos la explicaci'on detallada de los m'odulos junto con sus clases relevantes.


\subsubsection{Comunicaci'on}
Se encarga de mantener la comunicaci'on de bajo nivel contra los clientes. 'Esto incluye la recepci'on de pedidos y el envio de las respuestas y mensajes de estado. Adem'as es el encargado de abstraer el medio de comunicaci'on (por ejemplo, por archivo que es el utilizado en el presente trabajo). Su l'ogica es simple y genera, consistiendo de rutinas de escucha de mensajes

\subsubsubsection{Clases relevantes}

\begin{description}
\item[ReceptorPedidos] Es el encargado de \italica{poolear} el medio que le fue especificado en b'usqueda de nuevos pedidos. Posee un ReceptorPedidosConcreto qui'en es el encargado de especificar el medio por el cual debe escuchar pedidos (en nuestro caso, por archivo) y es establecido en el arranque de la aplicaci'on. Una vez que se invoca a ComenzarRecepcion, el servidor ya est'a listo para comenzar a recibir pedidos de clientes. Notar que no es un singleton porque no se requiere limitar la cantidad de receptores (pueden haber uno por thread si el sistema fuese multithreading) ni requiere ser referenciado. 
\item[DespachadorRespuestas] Similar al receptor, 'este se encarga de enviar la respuesta ya generada de vuelta a un cliente. Es un singleton para que se permita su uso a quien as'i lo requiera, pero no para limitar la cantidad de instancias (en principio, podrian ser m'as si se necesitan atender despachos de varios threads). Al igual que el receptor, posee un DespachadorRespuestasEspecifico que se encarga de hacer el env'io real porque es quien realmente conoce el medio por donde se debe enviar la respuesta (en nuestro caso, por archivo).
\end{description}


\subsubsection{MensajeroDeEntrada}
Tiene como tarea la de manejar el flujo de informaci'on dentro del servidor. 'Esto incluye distribuir los pedidos recibidos desde la capa de comunicaci'on a los respectivos encargados de atenderlos y de avisarle a los encargados de generar las respuestas que las generen (adem'as de informar cu'al de ellas deben generar). En su l'ogica, principalmente sabe recorrer los XML para obtener los valores que necesitan los responsables de atender al pedido. Adem'as, sabe comprender dada una respuesta del manejo del pedido a quien le debe informar que respuesta devolver al cliente.

\subsubsubsection{Clases relevantes}

\begin{description}
\item[DespachadorPedidos] Es el responsable de despachar cada pedido recien entrado a su manejador correspondiente.
\item[\italica{Manejadores}] Cada manejador posee la responsabilidad de atender a pedidos de un conector del lado del cliente y posee un m'etodo por cada pedido que pueda llegar. Dentro de cada uno de 'estos m'etodos se encuentra la l'ogica de invocaci'on al Modelo y MensajeroDeSalida.
\end{description}


\subsubsection{MensajeroDeSalida}
Se encarga de generar las respuestas a los clientes, sabiendo de donde obtener la informaci'on necesaria para lograrlo. Adem'as es quien atiende eventos generados por la l'ogica del casino y sabe como actual ante cada uno de ellos. Tiene l'ogica para conseguir los datos necesarios para generar las respuestas, al igual que sabe como 'estas deben ser armadas.

\subsubsubsection{Clases relevantes}

\begin{description}
\item[\italica{Manejadores}] 
\end{description}


\subsubsection{Modelo}
Contiene la l'ogica de negocio junto con las estructuras asociadas a ella. Expone interfaces (llamadas \italica{fachadas}) mediantes las cuales los demas m'odulos pueden comunicarse con 'el. Su l'ogica esta compuesta por las validaciones ante cada mensaje y las consultas y modificaciones que 'estos generan. 

\subsubsubsection{Clases relevantes}

\begin{description}
\item[AdministradorJugador], \italica{AdministradorObservador} y \italica{FachadaUsuario}. 'Estas clases hacen el manejo del ingreso y egreso de jugadores y observadores. Adem'as de las consultas sobre el saldo, la existencia (entre otras) de jugadores y observadores

\item[Usuarios] contiene la lista de jugadores (normales y vip) y la lista de observadores.
Sus operaciones est'an relacionadas con el manejo de las antedichas listas y obtenci'on de datos sobre el jugador, mediante \italica{ObtenerSaldoJugador}

\item[AdministradorMesaCraps] es la responsable del ingreso y egreso de jugadores en las mesas, decir qu'e mesas est'an abiertas, ultimos resultados, entre otros. B'asicamente se ocupa de administrar todo lo referente a las mesas y los tiros del juego craps.

\item[AdministradorMesaTragamonedas] al igual que la anterior administra lo referente a las mesas y los tiros, pero en este caso es sobre el juego de Tragamonedas.

\item[Mesas] la lista de mesas de craps y la lista de mesas tragamonedas. La \italica{Clase MesaTragamoneda} y la \italica{Clase MesaCraps} son las representaciones de las mesas del juego Tragamonedas y del juego Craps respectivamente. 

\item[JugadaCraps] contiene el resultado de los dados, a traves de la \italica{Clase Dado}. Tambi'en contiene las apuestas efectudas y por qu'e importe.

\item[JugadaTragamonedas] contiene el resultado de los rodillos, a traves de la \italica{Clase ResultadoTragamonedas}. Pues esta contiene la \italica{Clase RodilloTragamonedas} que tiene como atributo el valor del rodillo.
Tambi'en se relaciona con la \italica{Clase PozoProgresivo}, de la cual se obtendr'a el monto para realizar el pago y el consiguiente reseteo del mismo.

\item[AdministradorPozos] muest]a los saldos del pozo progresivo y del pozo feliz.

\item[ReceptorPedidos] media]te su especializaci'on levantar'a los distintos pedidos de la manera que sea necesaria.

\item[DespachadorRespuestas] es an]aloga a la anterior y tiene el mismo comportamiento diferenci'andose en que es para las respuestas. Hay una por puerto.

\item[DespachadorPedidos] inter]retar'a el pedido y seg'un de qui'en provenga se lo dar'a para procesar a alguna de las Clases \italica{JuegoCraps}, \italica{AccesoYVistaCraps}, \italica{AccesoYVistaTragamonedas} o \italica{AccesoYVistaCasino}. 

\end{description}



\subsection{Cliente Jugador}
Debido a la necesidad de una interfaz gr'afica de usuario para la utilizaci'on del producto, decidimos adoptar como estrategia para el modelado una aproximaci'on a un sistema de eventos y ventanas tal como los masivamente usados en casi todas las plataformas. En este tipo de sistemas el desarrollador de aplicaciones consta con bibliotecas y APIs que le permiten interactuar con el sistema de ventanas y eventos. Hemos realizado la simplificaci'on y abstracci'on que creemos conveniente para que nuestros diagramas sean simples y al mismo tiempo lo suficientemente expresivos para no dejar de lado cuestiones que hacen al mundo de la programaci'on orientada a eventos.
Presentamos a continuaci'on la descripci'on de los m'odulos componentes.

\subsubsection{GUI}
Es el conjunto de clases que modelan la interfaz gr'afica presentada al usuario.

\subsubsubsection{Clases relevantes}

\begin{description}
\item[\italica{Ventana}] Modela la clase base a toda representaci'on subyacente de un gr'afico de ventana. Generalmente esta clase es provista por una biblioteca. Es de esta clase que, mediante herencia, podemos definir las ventanas personalizadas que utilizan nuestros clientes.
\end{description}



\subsubsection{Controladores GUI}
Consisten en el conjunto de clases que controlan los eventos desencadenados por acciones del usuario en su interacci'on con el sistema de ventanas o generados exclusivamente por el sistema de ventanas. Existe un controlador por cada ventana a excepci'on de ventanas particularmente sencillas de las cuales no resulta imprescindible manejar de una forma interesante los eventos que podr'ia generar un usuario al interaccionar con las mismas.

Nuestro diagrama puede dar una noci'on parcial referente a la apariencia visual de las ventanas gr'aficas definidas si inferimos y caracterizamos de alguna forma intuitiva los nombres de los m'etodos de las clases de ventanas y controladores. Cre'imos apropiado incluir una descripci'on aproximada pero mucho m'as precisa de c'omo lucir'an nuestras interfases gr'aficas en el producto final. Adem'as, es en estas interfases en las que pensamos al momento de dise'nar el cliente, con lo cual hay una correspondencia 1 a 1 (salvando algunos casos) con las clases presentadas en el diagrama.

% \subsubsubsection{LogIn}
\imagen{PrototiposPantalla/PlayerClient_SignIn.png}{Prototipo de pantalla de ingreso al casino. Clase \italica{VentanaLogin}}{0.5}
% 
% 
% % \subsubsubsection{Lobby}
\imagen{PrototiposPantalla/PlayerClient_Lobby.png}{Prototipo de pantalla del lobby del casino. Clase \italica{VentanaLobby}}{0.5}
% 
% 
% % \subsubsubsection{Selecci'on de ficha}
\imagen{PrototiposPantalla/PlayerClient_SelectCoinValue.png}{Prototipo de pantalla de selecci'on de ficha. Clase \italica{VentanaSeleccionarValorFicha}}{0.5}
% 
% 
% % \subsubsubsection{Selecci'on de mesa}
\imagen{PrototiposPantalla/PlayerClient_SelectTable.png}{Prototipo de pantalla de selecci'on de mesa. Clase \italica{VentanaSeleccionarMesa}}{0.5}
% 
% % \subsubsubsection{Tragamonedas}
\imagen{PrototiposPantalla/PlayerClient_Tragamonedas.png}{Prototipo de pantalla del juego Tragamonedas. Clase \italica{VentanaTragamonedas}}{0.5}
% \clearpage
% 
% % \subsubsubsection{Craps}
\imagenvertical{PrototiposPantalla/PlayerClient_Craps.png}{Prototipo de pantalla del juego Craps. Clase \italica{VentanaCraps}}{0.65}
% \clearpage

\subsubsection{Comunicaci'on}
Esquema semejante al que utilizamos para el dise'no del servidor aunque con algunos cambios que vale la pena mencionar.


\subsection{Cliente Administrador}
\todo{Hacer esto}
\clearpage

\subsubsubsection{Panel de administraci'on}
\imagen{PrototiposPantalla/PlayerAdmin_AdminPanel.png}{Prototipo de pantalla del panel de administradores}{1}
\clearpage

\subsubsubsection{Verificaci'on de identidad}
\imagen{PrototiposPantalla/PlayerAdmin_VerifyPassword.png}{Prototipo de pantalla de verificaci'on de password de administrador}{1}
\clearpage

\subsubsubsection{Modo Dirigido :: Tragamonedas}
\imagen{PrototiposPantalla/PlayerAdmin_Tragamonedas.png}{Prototipo de pantalla de configuraci'on del modo dirigido para el juego Tragamonedas}{1}
\clearpage

\subsubsubsection{Modo Dirigido :: Craps}
\imagen{PrototiposPantalla/PlayerAdmin_Craps.png}{Prototipo de pantalla de configuraci'on del modo dirigido para el juego Craps}{1}
\clearpage

\subsubsubsection{Modo Dirigido :: Jugada Feliz}
\imagen{PrototiposPantalla/PlayerAdmin_HappyMove.png}{Prototipo de pantalla de configuraci'on de la jugada feliz en modo dirigido}{1}
\clearpage


\clearpage

% - Detalles de la operatoria de cada componente
\section{Detalles de operatoria}
\subsection{Operatoria del Servidor}
\subsubsection{Medio de comunicaci'on}
El servidor dise'nado utiliza como medio de comunicaci'on (aunque permite ser f'acilmente reemplazado) con los clientes el sistema de archivos, es decir que los XML de pedidos y respuestas son depositados como archivos en directorios. En particular existen dos directorios: un \italica{buffer de entrada} en donde los clientes depositan sus pedidos y un \italica{buffer de salida} en donde el servidor deposita las respuestas listas para ser tomadas por los clientes. 'Esto no respeta en su totalidad la arquitectura propuesta por los Timbalistas porque no mantiene un medio aislado por conector si no que para todos es el mismo. Ademas, no es el mismo medio de entrada que de salida ya que exite un buffer para el primero y uno para el segundo. La elecci'on se realiz'o por cuestiones de simpleza en el dise'no e implementaci'on de la capa de Comunicacion, y porque los Timbalistas mismos dijeron que el documento era s'olo una gu'ia que pod'ia ser respetada en el grado que se desee.

\subsubsection{B'usqueda de nuevos pedidos}
La forma en la cual el servidor buscar'a nuevos pedidos es por \italica{pooling}. Iterar'a indefinidamente (hasta que se apague el servidor) buscando nuevos archivos en el buffer de entrada. De no encontrar archivo para procesar, esperar'a cierto lapso de tiempo para luego volver a consultar por nuevos pedidos. As'i fue hecho por ser la forma m'as simple y flexible de relizarlo (una norificaci'on por interrupciones por parte del sistema operativo capaz que no se puede realizar en todo lenguaje que se vaya a implementar).

\subsubsection{Threading y concurrencia}
La operatoria del servidor ser'a, en todo momento, mono-thread. Uno s'olo ser'a el encargado de verificar nuevos pedidos y, al encontrar uno, 'el mismo ser'a el encargado de hacer todo el procesamiento hasta depositar la respuesta correspondiente. Luego, continua con la recepci'on del pr'oximo pedido. 'Esto simplifica enormemente el funcionamiento del servidor, debido que no se requieren controles por problemas de concurrencia. Adem'as evita la necesidad de estructuras de dato extra que permitan sincronizar todos los threads a utilizar. Es por eso que se opt'o por hacerlo de esa forma.



\subsection{Operatoria del Cliente}
% \todo{Hacer esto}


\clearpage

% - Justificacion, analisis y explicacion de su dise�o, utilizando desde principios de dise�o, hasta patterns, pasando por depedencias, acoplamiento y cohesion, hasta mas secuencias de ejemplo para explicar porque su dise�o es bueno y elegante al resolver los problemas que se les presentaron.
\section{An'alisis del dise'no}
\subsection{An'alisis del Servidor}
\subsubsection{Digamos NO al pegote (bajo acoplamiento)}
El objetivo primordial del dise'no del servidor fue generar un m'inimo acoplamiento entre los elementos de software. Las estrategias utilizadas para conseguirlo fueron la modularizaci'on y dependencias de entidades abstractas.

La modularizaci'on consiste en agrupar todas las clases \italica{m'odulos}. Cada m'odulo posee una interfaz, compuesta por una serie de clases principalmente con m'etodos de acceso p'ublico que permiten la utilizaci'on de la funcionalidad del mismo. Toda clase de otro m'odulo s'olo podr'a depender bien del resto que habita en su mismo m'odulo o de aquellas definidas en la interfaz de otro m'odulo. As'i se logra un m'inimo acoplamiento entre las clases, ya que la estructura interna de un m'odulo puede cambiar completamente y sin la necesidad de revisar dependencias con otros m'odulos (obviamente, se asume que la interfaz no debe variar y que se debe seguir cumpliendo la misma funcionalidad que antes).

Tambi'en se foment'o que en lo posible toda clase dependiese de entidades abstractas (ya sean clases abstractas o interfaces). De 'esta forma, las clases concretas que realizan o heredan de 'estas pueden ser modificadas o reemplazadas (e incluso agregar m'as) por otras sin alterar aquellas que poseen dependencias. Y aquello es una consecuencia altamente favorable que provino del bajo acoplamiento generado. Algunos ejemplos de 'esta pr'actica son: 

\begin{itemize}
\item MesasAbiertas contiene una colecci'on de Mesa (clase abstracta)
\item ServidorJugada siempre devuelve Resultado y TipoJugada (que son clases abstractas)
\item UsuariosEnCasino depende de Usuario (otra abstracta)
\item MesaCraps posee como obsevador a un elemento que realiza la interfaz MesaCrapsObserver
\item DespachadorRespuestas posee una referencia a la interfaz DespachadorRespuestasEspecifico (y de forma similar ocurre en el ReceptorPedidos)
\end{itemize}



\subsubsection{Hagamos cosas con sentido (alta cohesi'on)}
Por lo general, lograr una ca'ida en el acoplamiento del dise'no trae como efectos secundarios una ca'ida de la cohesi'on. Y por cohesi'on, en el presente trabajo, se considerar'a a la sem'antica que cada entidad de software recibi'o. En el dise'no planteado hay algunos puntos a destacar que ayudaron a aumentar tal aspecto:

\begin{itemize}
\item La interfaz del Modelo est'a compuesta por fachadas (el patr'on de dise'no \italica{facade}). Cada fachada simbolliza un concepto f'acilmente manejable por los clientes que la utilizan (los dem'as m'odulos, que en nuestro caso son los mensajeros). Un ejemplo es la fachada JuegoCraps que representa al juego de craps como una ente separado de, entre otros, JuegoTragamonedas. Notar que la representaci'on real interna de ambos juegos es muy similar (ambos tipos de mesas heredan de la clase abstracta Mesa, ambos poseen referencias a Jugador, ambas poseen apuestas que terminan heredando de la clase abstracta Apuesta, etc). Pero 'esto no le es mostrado a los clientes, ellos cuando quieren usar al juego de craps solo utilizan al juego de craps y se desligan de la existencia del juego de tragamonedas.
\item Al favorecer el bajo acoplamiento se utilizaron dependencias contra entidades abstractas. Pero para que 'estas dependencias no pierdan el sentido, se cuid'o que
	\begin{itemize}
	\item el nombre de las entidades abstractas sea realmente declarativo. Un ejemplo: el ServidorJugadas posee SelectorResultadoCraps (clase abstracta) y f'acilmente se comprende que se trata de una clase que elige (selecciona) que resultado de craps debe ocurrir, sin importarnos el criterio utilizado. De haberla bautizado ResultadoModoDirigidoCraps el ServidorJugadas podr'ia pensar que solamente devuelve el resultado establecido por ModoDirigido para craps, con lo que podr'ia entrar en duda que hacer cuando se debe elegir un resultado al azar (adem'as de no saber como hacerlo por no poseer tal l'ogica).
	\item la elecci'on si una clase es abstracta bo interfaz no fuese tomada a la liguera. Sabemos que no es lo mismo decir ''A hereda de B'' que ''A realiza B''. En el primer caso estamos asumiendo que A \negrita{es} B por lo que A solamente es una extensi'on de B (pero todo lo que hac'ia B lo sigue haciendo A de la misma manera). En el segundo, estamos diciendo que A \negrita{se comporta como} B por lo que solamente se dice que ofrece el misma funcionalidad, peor no se establece de que forma la realiza ni estamos diciendo que sem'anticamente sean lo mismo. Por ejemplo, un JugadorVip \negrita{es} un Jugador y por ende puede hacer todo lo que hace un jugador (por ejemplo, elegir mesas para jugar). En cambio el InformanteDeCambiosACientes \negrita{se comporta como} un observador de mesas de craps porque permite ser notificado de cambios en una mesa de craps, pero sem'anticamente no se quiso decir m'as que eso. 
	\end{itemize}
\item se definieron tipos de dato b'asicos para ayudar a comprender su significado. Por ejemplo, se defini'o el tipo Nombre para referenciar al nombre de un usuario, siendo en realidad una redefinici'on del tipo Texto. As'i es como en EntrarCasino(Nombre, Texto) de LobbyCasino es f'acil identificar que el primer par'ametro debe corresponder al nombre de un jugador (por supuesto que casos como 'este se ven en todo el dise'no). Lo mismo sucede con Creditos, IdMesa e IdTerminalVirtual.
\end{itemize}



\subsubsection{Reutilizando recetas \italica{exitosas} (patrones de dise'no)}
El dise'no del servidor plante'o varios problemas que fueron resueltos utilizando algunos patrones de dise'no. A continuaci'on se muestra cada uno junto a los lugares donde se usaron:

\begin{description}
\item[Singleton] Asegura que solo una instancia de la clase ser'a creada y permite el acceso a 'esta desde cualquier punto de la aplicaci'on sin necesidad de guardar la referencia. Se utiliz'o en
	\begin{itemize}
	\item DespachadorPedidos de MensajeroDeEntrada
	\item Manejadores del MensajeroDeEntrada y del MensajeroDeSalida
	\item Fachadas del Modelo
	\item clases Soporte del Modelo
	\item NotificadorDeCambiosAClientes del MensajeroDeSalida
	\item DespachadorRespuestas de Comunicacion
	\end{itemize}
\item[Abstract Factory] Permite desacoplarle la responsabilidad de elegir que objeto esec'ifico crear al encargado de crearlo. Se utiliz'o en:
	\begin{itemize}
	\item JugadorFactory del Modelo, permitiendo que quien quiera crear al objeto jugador cuando ingrese al casino no deba elegir si crear un JugadorVip o JugadorNormal.
	\item SelectorTipoJugada del Modelo, haciendo que ServidorJugada genere un nuevo TipoJugada a travez del selector sin saber espec'ificamente cu'al est'a generando.
	\end{itemize}
\item[Strategy] Permite cambiar din'amicamente el algoritmo a utilizar para realizar cierta funcionalidad. Se utiliz'o en:
	\begin{itemize}
	\item TipoJugada del Modelo, para que la Jugada simplemente le pida que se resuelva y ella, seg'un la instancia concreta, se resuelva como ya sabe.
	\item Apuesta del Modelo, para que TipoJugada simplemente le pida que se resuelva y ella, seg'un la instancia concreta, se resuelva como ya sabe.
	\item Selectores del Resultado en el Modelo, para que ServidorJugadas no deba elegir el Resultado que debe revolver en funci'on de como fue condigurado el modo dirigido.
	\end{itemize}
\item[Facade] Permite ver a un subsistema como una sola entidad. Se utiliz'o en:
	\begin{itemize}
	\item LobbyCasino del Modelo, para que quien solamente quiera ver al Modelo como que contiene a la recepci'on de un casino as'i pueda.
	\item JuegoCraps y JuegoTragamonedas del Modelo, los clientes ven solamente los respectivos juegos sin importar la implementaci'on de los subsistemas.
	\item AdministradorDeCasino del Modelo, muestra el centro de administraci'on de un casino.
	\end{itemize}
\item[Observer] Permite que un objeto sea notificado cuando otro objeto cambia su estado. Se utiliz'o en:
	\begin{itemize}
	\item IMesaObserver del MensajeroDeSalida, donde el 'unico que en este caso realiza la interfaz (NotificadorDeCambiosAClientes) es notificado por la mesa que observa (en este caso se sabe que solamente ser'a notificado de mesas de craps) que alg'un estado suyo cambi'o.
	\end{itemize}
%\item[Template Method] Permite definir un algoritmo en forma gen'erica delegando ciertos pasos a las subclases. Se utiliz'o en:
%	\begin{itemize}
%	\item Jugador del Modelo, donde el PuedePagar
%	\end{itemize}
\end{description}



\subsubsection{Cuesti'on de criterio (criterios de dise'no)}
Para finalizar el an'alisis, aqu'i se muestran los criterios respetados (en mayor o menor medida) en el dise'no del servidor.

\begin{description}
\item[Open-Closed] El dise'no es extensible (sin necesidad de modificaci'on) en los siguientes puntos:
	\begin{description}
	\item nuevo tipo de jugador: el nuevo tipo debe heredar de Jugador como lo hacen JugadorNormal y JugadorVip, y hay que crear una nueva JugadorFactory que genere al nuevo tipo.
	\item nuevo juego: se debe generar una nueva fachada para la inerfaz del Modelo, nuevo tipo de resultado que herede de Resultado, nueva/s apuesta/s que herden de Apuesta, nuevo tipo de mesa que herde de Mesa y nuevos Selectores para el manejo del modo dirigido y la elecci'on del resultado y tipo de jugada que debe obtener cada mesa.
	\end{description}
\item[Liskov Substitution] Toda clase derivada se comporta igual que su clase padre, y se puede ver en toda herencia del dise'no: Jugador, Mesa, Pozo, Resultado, Apuesta, etc.
\item[Dependency Inversion] En donde no era necesario depender de clases espec'ificas se utiliz'o una clase abstracta o una interfaz. Por ejemplo, los observadores de mesas, obtenedores de pedidos, etc. Pero en varios lados era in'util agregar una abstracci'on cuando ya se sab'ia la subclase se iba a representar (una MesaCraps no tiene referencia a Apuesta sino a ApuestaCraps, de no haber sido as'i habr'ia tenido que hacer \italica{downcast} siempre para utilizar la l'ogica espec'ifica).
\item[Single Responsability] Toda clase (e incluso m'odulo) posee una sola responsabilidad y si debe ser modificada o reemplazada es por un s'olo motivo. Por ejemplo, si se establece que los jugadores vip tengan otra forma de validar si puede pagar cierta apuesta (por ej, tienen una cota m'inima de saldo que pueden tener) entonces habr'a que modificar JugadorVip, pero no hay otro motivo que as'i lo requiera.
\item[Interface Segregation] Se intent'o mantener a las interfaces de cada clase lo minimal posible, es decir que atiendan a una sola necesidad. Las fachadas del Modelo son las que terminaron obteniendo una interfaz muy grande, pero es de entender porque en una sola clase se necesit'o centralizar toda la funcionalida que ofrece una pieza del casino. Y de haber dividido esa interfaz, hubiese perdido sentido el objetivo de la fachada: ver esa parte del casino como una sola entidad (clase).
\end{description}


% \subsection{An'alisis del Cliente}
% \todo{Borrar lo que no sirva de aca que es de la version anterior y hacerlo bien}
Bas'andonos en las especificaciones observamos que ser'ia coherente que un usuario jugador/observador no conozca las posibles funcionalidades que puede utilizar un administrador y viceversa. En particular nos parece que los aspectos relacionados con el modo dirigido deber'ian incluso quedar expl'icitamente ocultos a observadores y jugadores. Por estas razones hemos decidido construir dos aplicaciones cliente separadas para modularizar el acceso a las distintas funcionalidades que un usuario puede efectuar. De esta forma dise~namos un cliente para jugadores/observadores y otro cliente exclusivo para administradores.

\subsubsection{Cliente Jugadores/Observadores}\label{Clientes::Jugadores/Observadores}

\subsubsection{Cliente Administradores}\label{Clientes::Administradores}

El cliente se basa en un dise'no en capas.

%\imagenvertical{DC_Cliente.png}{Diagrama de Clases Cliente}{0.4}

\begin{enumerate}
	\item \textbf{Presentaci'on: }Estar'an las clases qeu se encargaran de dibujar las distintas pantallas de juego y de manejar todo lo relacionado a los enventos que los usuarios activar'an.
	\item \textbf{Controlador: } estara la l'ogica que construye los mensajes al servidor y las clases necesarias para ello.
	\item \textbf{Comunicaci'on:} estar'an las clases y metodos que construyen el archivo xml y la acci'on de escribirlos en el puerto correspondiente del servidor (en este caso una carpeta en el mismo).
	\item  \textbf{Modelo:} este paquete tendr'a una clase, \textit{configuraci'on}, la cual contendr'a toda la informaci'on que los mensajes de comunicaci'on no contengan y que son necesarios para poder mostrar las pantallas y construir los mensajes de comunciaci'on
 \end{enumerate}

\subsubsection{Justificaci'on del dise~no}
Creemos que este dise~no es adecuado para este problema ya que es sumamente flexible tanto si se cambia la capa de presentaci'on como si se cambia la capa de comunciaci'on. Por ejemplo, si el d'ia de ma'nana queremos mostrarlo v'ia una p'agina web o si decidimos que la comunicaci'on sea hecha v'ia sockets nuestro dide~no se puede adaptar facilmente, ya que solo hay que modificar la capa correspondiente.

\subsubsection{Ejemplos / Referencias a DS's}
Ejemplos de la interac'on de las distintas capas del cliente pueden verse en los DS's:





\clearpage

% - Modificaciones al protocolo.
\section{Modificaciones al protocolo}\label{Seccion::ModificacionesAlProtocolo}

% % \input{../doc/Extensiones\ al\ protocolo.txt}
\clearpage

% - Comentarios, aclaraciones al corrector
\section{Comentarios}
En la presente secci'on agregaremos aclaraciones de las decisiones que el grupo fue tomando en el transcurso del dise'no. Adem'as expondremos otras alternativas que se nos ocurrieron ante cada problema. 

\subsection{}


\begin{itemize}
\item Las fachadas necesitan responder informaci'on del estilo \italica{pude hacer la acci'on} / \italica{no pude hacer la acci'on}, y es por eso que casi todas las acciones devuelven un Boolean. Pero hay otras acciones que adem'as necesitan devolver informaci'on espec'ifica de la acci'on. Para resolver la situaci'on se nos ocurri'o:
	\begin{itemize}
	\item Que las acciones devuelvan un objeto RespuestaDelModelo. 'Este deber'ia ser parecido a un \verb|Diccionario<nombre_parametro:Texto, valor_parametro:Objeto>| para que sea abarcativo a todos los pedidos de todas las clases (incluso los que no existen pero pueden ser agregados en un futuro, lo mismo vale para nuevas fachadas). Pero har'ia falta \italica{downcastear} para usar cada valor devuelto y no nos parece \italica{prolijo}.
	\item Idem al punto anterior pero que cada acci'on devuelva una estructura espec'ifica para 'esa acci'on la cual solamente es una lista de los par'ametros de respuesta. La ventaja con la opci'on anterior es que es ahora se tienen los valores fuertemente tipados. La gran desventaja es que se incrementan en gran n'umero la cantidad de clases del diagrama, adem'as que se dificulta agregar nuevas acciones porque hace falta agregar una clase que represente la respuesta.
	\item Que se guarde toda esa informaci'on y sea accedida a travez de metodos de consulta. 'Esto disminuye un poco la cohesi'on de la fachada porque ahora se deben agregar m'etodos de consulta del estilo DetalleUltimaAccion (devuelve el mensaje de respuesta de porque fue rechazada la 'ultima acci'on realizada, o los mensajes de bienvenida si fue aceptada). Dada la simpleza de 'esta opci'on, porque los par'ametros a devolver no son numerosos y porque nuestro servidor funciona mono-thread (de no ser asi, la 'ultima acci'on no se sabe si es la realizada en el procesamiento del pedido actual o el de otro thread) , fue la elegida.
	\end{itemize}

\item En el protocolo, entrarCraps tiene como sem'antica que cuando IdMesa posee un valor nulo (vac'io) es porque hace falta crear una mesa. Las opciones para 'esto son:
	\begin{itemize}
	\item Hacer que el MensajeroDeEntrada verifique con un condicional si primero llama a CrearMesa o no seg'un si 'este texto es nulo o no. 'Esto har'ia que haya logica de negocio en el mensajero.
	\item Hacer que haya una sobrecarga de EntrarJugadorMesaCraps que una no reciba el id de mesa y as'i se asuma que hay que crear la mesa. Pero el mensajero deber'ia seguir teniendo l'ogica de negocio dentro porque tiene que decidir a cual llamar.
	\item Que la fachada haga la verificaci'on. Esto genera que no se entienda del todo bien la l'ogica porque no es intuitivo que si el id de mesa es nulo hay que crear una mesa nueva, pero es la mejor opcion en comparaci'on con las otras 3 porque mantiene la l'ogica de negocio en el Modelo. Adem'as, es una fachada, por lo que tener l'ogica de control de flujo no est'a mal.
	\end{itemize}

\item Las fachadas muestran un comportamiento \italica{plug and play} desde afuera, algo as'i como que desde afuera se pueden sacar y poner juegos de la nada utilizando formas de representaci'on completamente diferente desde afuera, por ejemplo. Pero en el fondo no es tan as'i. Por ejemplo, las fachadas van a ejecutar logica de validacion, y esta l'ogica muchas veces va a necesitar cosas que no son tan dependientes del juego en si (por ejemplo, del LobbyCasino). No estamos diciendo que nuestro dise'no no sea extensible (en efecto, lo es y se explica en la secci'on de an'alisis), solamete es un dato curioso que vale la pena notar.

\item Hacemos algunas verificaciones de si existe o no algo comparando con null para simplificar (por ejemplo, ObtenerMesaCraps de MesasAbiertas devuelve null cuando no existe la mesa con el id dado). Podriamos tambien agregar mas m'etodos que respondan con un Booleano si existe o no lo que busco, pero deber'iamos agregar varios metodos m'as. Por eso, y para mantener la simpleza del dise'no, se opt'o por utilizar comparaci'on por null (disminuyendo en peque'na medida el significado a simple vista del m'etodo).

\item Estamos asumiendo que el nombre de un jugador lo identifica un'ivocamente y no pueden existir dos jugadores con el mismo nombre. Lo mismo aplica para las mesas y su IdMesa.

\item En la resolucion de las apuestas, para poder reutilizar la l'ogica de TipoJugada, se le necesitan pasar las jugadas, apuestas y resultados en forma generica, es decir, los objetos abstractos (Jugada en lugar de JugadaCraps, por ejemplo). 'Esto genera que haya que terminar haciendo un casteo cuando se necesite informaci'on especifica (por ejemplo, el valor de los dados en un resultado craps o el valor del punto en una jugada craps). Y aunque no nos guste mucho usar downcast, vali'o la pena hacerlo con el objeto de hacer la l'ogica de TipoJugada bien gen'erica como para usarla en cualquier apuesta de cualquier juego.

\item No implementamos el cancelamiento de apuestas en craps porque el protocolo no lo soportaba.

\item En TirarDados de MesaCraps tuvimos que agregar un condicional para actuar en funci'on del estado de la ronda. Esto se podria haber solucionado haciendo que el estado de la ronda pase a ser una clase abstracta en donde cada estado herede de 'este y que la l'ogica necesaria sea implementada por cada concreta (se poodr'ia hablar de patr'on State). Pero por simpleza y porque es l'ogica muy ligada al juego, lo preferimos dejar como est'a.

\item Si se desea cambiar el medio de transmisi'on de los mensajes entre el cliente y el servidor, solamente hay que agregar nuevas clases que realicen las interfaces IObtenedorPedidos e IDespachadorRespuestasEspecifico y, en el inicio del servidor, asignarle a ReceptorPedidos y DespachadorRespuestas las instancias de 'estas nuevas clases (de forma similar a como se est'a haciendo ahora con ObtenedorPedidosArchivo y DespachadorPedidosArchivo).

\item Notar que hubo cierto acoplamiento en el dise'no al XML como estructura de pedidos. Y nos pareci'o razonable que as'i sea. Hacer que el medio por el cual viajan los mensajes s'i que deber'ia ser extensible porque en un futuro se pueden optar por formas mas eficientes, seguras, baratas, etc. Pero no le encontramos sentido (ni se especific'o un requerimiento) hacer que la aplicaci'on \italica{hable distintos idiomas}. De 'ultima, si es muy necesario realizar un cambio en la estructura 'este se sentir'ia fuertemente en los manejadores de los mensajeros (hay que cambiarlos casi completamente), en menor medida en la capa de comunicaci'on pero absolutamente nada en el Modelo (esta abstra'ido de quien o como lo usan).

\end{itemize}


\subsection{Sobre el observador para el EventHandler}
Para el E/H necesario en el juego de craps implementamos el patron observer entre las mesas de craps (observado) y NotificadorDeCambiosAClientes, una clase del MensajeroDeSalida (observador). La suscripci'on es s'olo una: cuando se notifica un cambio de estado el mensajero va a recuperar del modelo la info que necesita de la mesa que acaba de cambiar de estado (en el mensaje de notificaci'on viaja el numero de mesa para identificarla)  para armar el XML. Despu'es, el observador revisa que jugadores hay en 'esa mesa y para cada uno busca en un diccionario suyo destinado a tal fin (terminales) en que terminal virtual est'a. En 'esta taba es donde los jugadores realizan la suscripcion, porque contra el modelo no lo deben hacer (el Modelo no deber'ia conocer que existen terminales virtuales). Notar que de 'esta forma, la l'ogica de negocio que dice \italica{todos los jugadores en una mesa deben enterarse de los cambios en esa mesa} se la est'a definiendo en el MensajeroDeSalida y no el Modelo (cuando se defini'o en un principio que la l'ogica de negocio estar'a s'olo en el Modelo), pero de no ser asi hace falta que el modelo haga la suscripci'on y conozca las terminales virtuales (es decir, tambi'en se defini'o que lo que est'a en el Modelo es l'ogica de negocio). Pero como es una l'ogica m'inima, y encima en ning'un momento el MensajeroDeSalida utiliza estructuras internas del Modelo (todos los datos necesarios los toma de fachada correspondiente) es que decidimos hacerlo de 'esta forma.

Otra opci'on ser'ia que los observadores sean directamente cada terminal virtual. Pero 'esto no se puede hacer porque el protocolo no soporta los mensajes que necesitar'ia env'iar el cliente para obtener la informaci'on necesaria de la mesa que cambi'o el estado.

Cada mesa obtiene cuando es constru'ida el observador al cual le deber'ia notificar los cambios. Para ello, MesasAbiertas ya tiene la instancia del observador y es ella quien se la pasa a las mesas que crea (notar que solo ella crea las mesas).

Notar que cuando alguien entra, el MensajeroDeEntrada debe hacer un NotificarEstadoCraps con el que recien entr'o para que le llegue el estado de la mesa en el momento en que entr'o (notar que el observer no lo va a incluir en la actualizacion generadoa por \italica{entr'o alguien a la mesa} porque en ese entonces todav'ia no estaba suscripto). 'Esto puede ser considerado l'ogica de negocio que est'a realizando el MensajeroDeEntrada, pero es la mejor opci'on que hay. Tambi'en se puede poner un nuevo m'etodo que me diga si se puede o no hacer la entrada pero no la realice y otro metodo que haga la entrada real, donde realmente se hace la notificacio del cambio, pero igual deber'iamos poner l'ogica en el MensajeroDeEntrada que haga o no el llamado correspondiente a hacer la notificaci'on real cuando la validaci'on se pudo hacer. As'i que optamos por la opci'on sencilla y dejarlo como la primer alternativa. Con la salida 'esto no pasa porque el cambio en la mesa es justamente que el jugador que sale no est'a m'as, entonces cuando el observer quiera notificar no va a encontrar al que esta saliendo de la mesa.


\subsection{Sobre el modo dirigido}
La configuraci'on de modo dirigido (o incluso la falta de 'este en el estado Azar) se realiza a nivel de mesa para cualquier estado y a travez del selector de turno asignado a esa mesa. Si llega un mensaje de configuracion para hacer el tipo de jugada de una mesa siempre feliz, se actualiza el estado solamente para la mesa que se especif'ica en el mensaje. En cambio si llega otro (es decir, uno que se especific'o que apunta a nivel de juego) se hace que cada mesa de 'ese juego actualice su estado. 

Notar que 'esto genera que cualquier cantidad de mesas puedan estar configuradas para que siempre saquen jugada feliz. Por consiguiente, se podr'ia considerar que dos o m'as mesas sacan jugada feliz \italica{simult'aneamente}. Y en principio parecer'ia que no se cumple el requerimiento planteado por los Timbalistas que plantea que no pueden suceder simult'aneamente en dos o m'as mesas del casino una jugada feliz. Pero en la minuta de relevamiento se establece de forma m'as espec'ifica que no puede suceder una jugada feliz hasta que no se pague una jugada feliz ocurrida anteriormente. Y 'esto no sucede, dado que la ejecuci'on de la jugada, la decisi'on de su tipo y el pago de los premios se hacen en el mismo mensaje (en el de tirarCraps y girarRodillos, dependiendo del juego). Adem'as, nuestro servidor es mono-thread, por lo que efectivamente ning'un mensaje de tiro nuevo se levantar'a para procesar hasta no haber procesado el actual. Y en procesar se incluyen los pagos (no solo la asignaci'on de mas saldo al objeto jugador, si no adem'as las notificaciones de respuesta de lo que sucedio e incluso las notificaciones por E/H en craps). Es por eso que los requerimientos exigidos se siguen cumpliendo.

Algo con lo cual uno no podr'ia estar muy de acuerdo con lo dicho anteriormente es que as'i y todo puede pasar que a dos jugadores en diferentes mesas les aparezca a la vez que su jugada fue feliz. 'Esto puede pasar si el mensaje de respuesta del primer jugador se tard'o en leer por el lado del cliente, y lo levant'o junto al cliente del segundo jugador cuand se genero la respuesta de 'este (y en ambas respuestas se decia que la jugada fue feliz). Para reparrar 'esta situaci'on lo que hay que hacer no es ''no hacer que jugadas sucesivas sean felices'' si no hacer que hasta que todos los clientes no hayan levantado sus mensajes no seguir procesando otro pedido de entrada. Pero tanto detalle nos termin'o pareciendo innecesario y esa situaci'on dejamos que pueda ocurrir.



\subsection{Notas varias}

\clearpage

% - Conclusiones, cosas que nos dimos cuenta.
\section{Conclusiones}
\subsection{Comentarios al corrector}
\subsubsection{La ficci'on supera a la realidad}
Docente de IS1, en la seccion de requerimientos del sistema nos comprometimos, entre otros, a que nuestro sistema sea escalable y seguro. Lo hicimos porque son requerimientos importantes y como tales deberemos cumplir para que los stakeholders est'en satisfechos con la soluci'on que les ofreceremos. Pero en el contexto del TP, no obtuvimos ninguna m'etrica acerca de que tan escalable y qu'e tan seguro deber'ia ser (a'un as'i, no son variables que deber'ian obviarse por las caracter'isticas del producto). Es por eso que en realidad, en el TP, no nos comprometemos a cumplir estos requerimientos. 

De forma similar ocurre con el requerimiento de ser r'apido. En realidad nos comprometemos a qu'e el sistema sea ``utilizable'' pero no ofrecemos ninguna garant'ia real, principalmente por que no se nos ofrecieron m'etricas a cumplir.

El dise'no no ser'a inclu'ido no porque tengamos dise'nadores en huelga si no porque ning'un integrante de nuestro grupo se destaca por poseer cualidades art'isticas...



\clearpage

% - Referencias
\section{Referencias}
\begin{thebibliography}{99}
\bibitem[MVC-Java]{MVC-Java} Patr'on MVC: http://java.sun.com/blueprints/patterns/MVC-detailed.html
\bibitem[Reactor]{Reactor} Patr'on Reactor: http://www.cse.wustl.edu/~schmidt/PDF/reactor-siemens.pdf
\end{thebibliography}



\clearpage

% ---

% Anexo A - Ejemplo de configuraciones
\appendix
\section{Ejemplos de los archivos configurac'ion}
\subsection{ConfiguracionCasino.xml}
Es el archivo de configuraci'on general del casino. Ser'a consumido por la clase ConfiguracionCasino al iniciar el servidor.

\begin{verbatim}
<configuracionCasino>
    <configuracionDelCasino>
        <password_admin valor="MentaGranizada" />
        <saldoCasino monto="10000" />
        <fichasValidas>
            <fichaValida valor="20" />
            <fichaValida valor="10" />
            <fichaValida valor="5" />
            <fichaValida valor="1" />
        </fichasValidas>
    </configuracionDelCasino>
    <configuracionTragamonedas>
        <probabilidadOcurrenciaFiguras>
            <figura tipo="Blanco" ocurrencia="0.40" />
            <figura tipo="Cereza" ocurrencia="0.29" />
            <figura tipo="BarSimple" ocurrencia="0.15" />
            <figura tipo="BarDoble" ocurrencia="0.10" />
            <figura tipo="BarTriple" ocurrencia="0.05" />
            <figura tipo="Dinosaurio" ocurrencia="0.01" />
        </probabilidadOcurrenciaFiguras>
    </configuracionTragamonedas>
    <configuracionCraps>
        <probabilidadOcurrenciaNumeros>
            <numero valor="1" ocurrencia="0.16" />
            <numero valor="2" ocurrencia="0.16" />
            <numero valor="3" ocurrencia="0.16" />
            <numero valor="4" ocurrencia="0.16" />
            <numero valor="5" ocurrencia="0.16" />
            <numero valor="6" ocurrencia="0.16" />
        </probabilidadOcurrenciaNumeros>
    </configuracionCraps>
    <configuracionPozoFeliz>
        <porcentajeDescuentoTodosPonen valor="0.2" />
        <probabilidadOcurrenciaFeliz valor="0.05" />
        <probabilidadOcurrenciaTodosPonen valor="0.1" />
        <montoMinimo valor="1000" />
    </configuracionPozoFeliz>
    <configuracionPozoProgresivo>
        <cantApuestasJugagaMaximaGordoProgresivo valor="3" />
        <porcentajeDescuentoPorApuesta valor="0.1" />
        <montoMinimo valor="500" />
    </configuracionPozoProgresivo>
</configuracionCasino>

\end{verbatim}

\subsection{JugadoresRegistrados.xml}
Es la lista que genera Marketing a partir de la informaci'on que le va dando Rosa Espinosa. Contiene la informaci'on de los jugadores registrados para poder jugar en el casino junto con su saldo y tipo de jugador. Ser'a consumido por la clase JugadoresRegistrados al iniciar el servidor.

\begin{verbatim}
<jugadores>
    <jugador nombre="Cosme Fulanito" saldo="2000" tipo="normal" />
    <jugador nombre="Pedro el de la Esquina" saldo="100.50" tipo="normal" />
    <jugador nombre="Matute" saldo="0.25" tipo="vip" />
</jugadores>
\end{verbatim}

\clearpage



% ------------------------
% Fin del documento
% ------------------------
\end{document}
