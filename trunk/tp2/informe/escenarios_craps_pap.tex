



\textbf{Tirar Dados}


Este escenario es bastante gen'erico,

\escenario{
      \begin{itemize}
      \item El usuario puede o no estar en la mesa.
      \item El usuario puede o no ser el tirador.
      \item En caso de que sea el tirador y est'e en la mesa.
      \item Se tiran los dados.
      \item Si sali'o alg'un valor propio de punto este se setea.
      \item Hay alguna cantidad de apuestas de alg'un tipo que se resuelven o no con su l'ogica particular, dependiendo del contexto %$^1$
      \item Estamos en un ``est'an saliendo''% $^3$
      \item Es una jugada feliz %$^2$
    \end{itemize}
}

% 
% $^1$ Resoluci'on de distintas apuestas se ve en otros DS's (en sitio a perder y a venir)
% $^2$ En otro DS se ve la situacion con el caso de que el punto esté establecido
% $^3$ En otro DS se ve la jugada Todos ponen

El pozo feliz se reparte si o si el servidor de Jugadas no devuelve una feliz mientras el pozo no lleque al m'inimo.
La notificaci'on a de que una mesa cambió se pude ver en apostar, sin bien est'a instanciado en una mesa, aplica si la mesa fuera gen'erica.


Este DS se dividió en 3 secciones:
\begin{enumerate}
 \item Recepcion de pedido: es la recepcion de pedido,  y la respuesta hacia el modulo de comunciación, no se muestra lo que sucede en la llamada TirarCraps (usuario, unXML)
\item  TirarCraps: Hace todo lo concerniente a la validaci'on, no se hace zoom en TirarDados.
\item TirarDados: aqui puede verse lo que pasa cuando se hace un tirar dados de una mesa
\end{enumerate}

\clearpage
\textbf{Apostar Craps}

El escenario para 'esta funcionalidad es el siguiente:

\escenario{
El jugador Cosme Fulanito desea apostar en el juego de craps. La apuesta realizada es a ganar sobre el n'umero cinco. La cantidad apostada es una ficha de valor 20 y dos de valor 10. Cosme est'a en la mesa 25.

%Se crea el XML correspondiente con el nombre del archivo "apuestaCraps059999" y se deposita en la carpeta del servidor.

%En el servidor se levanta dicho XML, se lo parsea y se efect'uan las valiaciones correspondientes. A saber:

Para que su apuesta sea efectiva deber'a pasar por las siguientes validaciones:

\begin{itemize}
 \item Cosme Fulanito es un jugador de la mesa en la que acusa estar;
 \item Los valores de fichas utilizados son v'alidas para 'este d'ia;
 \item Debe tener los fondos suficientes para poder pagar lo que apost'o o que sea un cliente vip;
\end{itemize}

Si alguna de las validaciones anteriormente mencionadas no se cumple entonces no se le dejar'a realizar la apuesta. En caso contrario si.

Si se cumplen todas las validaciones se agregar'a la apuesta a la mesa y adem'as se notificar'a a todos los dem'as jugadores de la mesa 25 del cambio sucedido.
}

%aca iria una imagen------------------------------------------------------------

\clearpage

\textbf{Entrar Casino}

\escenario{
Un usuario desea entrar al casino, puede hacerlo en modo jugador o en modo observador.
Si ya ha ingresado en modo jugador no se lo dejar'a entrar nuevamente. Si est'a en modo observador y desea ingresar en el mismo modo tampoco podr'a hacerlo.

En cambio si quiere entrar como jugador (independientemente de si ingres'o como observador o si no ingres'o) se deber'a validar que sea un usuario autorizado por marketing:

\begin{itemize}
 \item En caso afirmativo quedar'a ingresado en modo jugador.
 \item En caso negativo quedar'a en modo observador o fuera del casino seg'un cual fuese su estado anterior.
\end{itemize}
}

%aca iria una imagen------------------------------------------------------------

\clearpage

\textbf{Pedir Estado Casino}

\escenario{
Un usuario desea informarse sobre el estado del casino. Se le informar'a s'olo si ha ingresado en el casino, m'as haya si es en modo jugador o en modo observador.

El estado del casino est'a formado por:

\begin{itemize}
 \item La lista de jugadores y observadores ingresados en el casino;
 \item El valor del pozo feliz y del pozo progresivo;
 \item El estado de las mesas del juegos de craps:
	\begin{itemize}
	 \item Los jugadores;
	 \item El 'ultimo tirador y el pr'oximo;
	 \item Si el siguiente es tiro de salida o ya est'a el punto establecido;
	 \item El valor de los dados en el 'ultimo tiro;
	\end{itemize}
 \item El estado de las mesas del juego tragamonedas:
	\begin{itemize}
 	 \item Los jugadores;
 	 \item El valor de los rodillos en el 'ultimo tiro;
 	 \item El 'ultimo tirador y el pr'oximo;
	\end{itemize}
\end{itemize}
}

