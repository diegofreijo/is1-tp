
\escenario{Tirar Dados de punta a punta}{

El usuario puede o no estar en la mesa. El usuario puede o no ser el tirador. En caso de que sea el tirador y est'e en la mesa. Se tiran los dados. Si sali'o alg'un valor de punto este se setea. Hay alguna cantidad de apuestas de alg'un tipo que se resuelven o no con su l'ogica particular, dependiendo del contexto. Estamos en un ``est'an saliendo''. Es una jugada feliz 

}
% 
% $^1$ Resoluci'on de distintas apuestas se ve en otros DS's (en sitio a perder y a venir)
% $^2$ En otro DS se ve la situacion con el caso de que el punto est'e establecido
% $^3$ En otro DS se ve la jugada Todos ponen

\textbf{Aclaraciones: }El pozo feliz se reparte si o si. El servidor de Jugadas no devuelve una feliz mientras el pozo no lleque al m'inimo.
La notificaci'on de que una mesa cambió se pude ver en el DS \textit{apostarCraps}, sin bien en este DS est'a instanciado en una mesa particular, aplica si la mesa fuera gen'erica.


Este DS se dividió en 3 secciones:
\begin{enumerate}
\item Recepcion de pedido: es la recepcion de pedido,  y la respuesta hacia el modulo de comunciación, no se muestra lo que sucede en la llamada TirarCraps (usuario, unXML)
\item  TirarCraps: Hace todo lo concerniente a la validaci'on, no se hace zoom en TirarDados.
\item TirarDados: aqui puede verse lo que pasa cuando se hace un tirar dados de una mesa (Va impreso aparte o digital por el tama~no)
\end{enumerate}

%aca iria una imagen------------------------------------------------------------
\imagenvertical{DS_Craps/TirarDados/RecepcionpPedido.png}{Tirar Dados Recepcion de pedido}{0.5}
\imagenvertical{DS_Craps/TirarDados/tirarCraps.png}{Tirar Dados: TirarCraps}{0.5}
% \imagen{}{TirarDados}{0.5}
Tirar dados \tam


\escenario{Apostar Craps de punta a punta}{

El jugador Cosme Fulanito desea apostar en el juego de craps. La apuesta realizada es a ganar sobre el n'umero cinco. La cantidad apostada es una ficha de valor 20 y dos de valor 10. Cosme est'a en la mesa 25.

En la mesa est'a el punto establecido y adem'as existe el pr'oximo tirador.

Para que su apuesta sea efectiva deber'a pasar por las siguientes validaciones:

\begin{enumerate}
 \item Cosme Fulanito es un jugador de la mesa en la que acusa estar;
 \item Los valores de fichas utilizados son v'alidas para 'este d'ia;
 \item Debe tener los fondos suficientes para poder pagar lo que apost'o o que sea un cliente vip;
\end{enumerate}

Si alguna de las validaciones anteriormente mencionadas no se cumple entonces no se le dejar'a realizar la apuesta. En caso contrario si.

Si se cumplen todas las validaciones se agregar'a la apuesta a la mesa y adem'as se notificar'a a todos los dem'as jugadores de la mesa 25 del cambio sucedido.
}

\imagen{DS_Craps/ApostarCraps/DS_ApostarCrapsCliente.png}{Apostar Craps en el cliente}{0.6}

\imagen{DS_Craps/ApostarCraps/DS_ApostarCrapsRecepcionDelPedido.png}{Apostar Craps, recepci'on del pedido}{0.6}

\imagen{DS_Craps/ApostarCraps/DS_ApostarCrapsDentroDelModelo.png}{Apostar Craps, operatoria dentro del modelo}{0.6}

\imagen{DS_Craps/ApostarCraps/DS_ApostarCrapsEmisionDeRespuesta.png}{Apostar Craps, emisi'on de la respuesta}{0.6}

\imagen{DS_Craps/ApostarCraps/DS_ApostarCrapsEH.png}{Apostar Craps, notificaci'on a cada jugador}{0.6}

\imagen{DS_Craps/ApostarCraps/DS_ApostarCrapsNotificarEstado.png}{Apostar Craps, armado de la notificaci'on}{0.6}

\subsubsubsection{Las siguientes funciones pertenecen al Apostar Craps. Fueron separadas porque tienen menor relevancia}

\imagen{DS_Craps/DS_ApuestasVigentes.png}{Apuestas Vigentes}{0.6}

\imagen{DS_Craps/DS_Dado1UltimoTiro.png}{Valor del Dado 1 en el 'ultimo tiro}{0.6}

\imagen{DS_Craps/DS_Dado2UltimoTiro.png}{Valor del Dado 2 en el 'ultimo tiro}{0.6}

\imagen{DS_Craps/DS_EsProximoTiroDeSalida.png}{Determina si el pr'oximo es tiro de salida}{0.6}

\imagen{DS_Craps/DS_JugadoresEnMesa.png}{Jugadores en mesa}{0.6}

\imagen{DS_Craps/DS_TiradorProximoTiro.png}{Tirador del pr'oximo tiro}{0.6}

\imagen{DS_Craps/DS_ValorPuntoProximoTiro.png}{Valor del punto para el pr'oximo tiro}{0.6}


