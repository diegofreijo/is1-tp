\subsubsection{Medio de comunicaci'on}
El servidor dise'nado utiliza como medio de comunicaci'on (aunque permite ser f'acilmente reemplazado) con los clientes el sistema de archivos, es decir que los XML de pedidos y respuestas son depositados como archivos en directorios. En particular existen dos directorios: un \italica{buffer de entrada} en donde los clientes depositan sus pedidos y un \italica{buffer de salida} en donde el servidor deposita las respuestas listas para ser tomadas por los clientes. 'Esto no respeta en su totalidad la arquitectura propuesta por los Timbalistas porque no mantiene un medio aislado por conector si no que para todos es el mismo. Ademas, no es el mismo medio de entrada que de salida ya que exite un buffer para el primero y uno para el segundo. La elecci'on se realiz'o por cuestiones de simpleza en el dise'no e implementaci'on de la capa de Comunicacion, y porque los Timbalistas mismos dijeron que el documento era s'olo una gu'ia que pod'ia ser respetada en el grado que se desee.

\subsubsection{B'usqueda de nuevos pedidos}
La forma en la cual el servidor buscar'a nuevos pedidos es por \italica{pooling}. Iterar'a indefinidamente (hasta que se apague el servidor) buscando nuevos archivos en el buffer de entrada. De no encontrar archivo para procesar, esperar'a cierto lapso de tiempo para luego volver a consultar por nuevos pedidos. As'i fue hecho por ser la forma m'as simple y flexible de relizarlo (una norificaci'on por interrupciones por parte del sistema operativo capaz que no se puede realizar en todo lenguaje que se vaya a implementar).

\subsubsection{Threading y concurrencia}
La operatoria del servidor ser'a, en todo momento, mono-thread. Uno s'olo ser'a el encargado de verificar nuevos pedidos y, al encontrar uno, 'el mismo ser'a el encargado de hacer todo el procesamiento hasta depositar la respuesta correspondiente. Luego, continua con la recepci'on del pr'oximo pedido. 'Esto simplifica enormemente el funcionamiento del servidor, debido que no se requieren controles por problemas de concurrencia. Adem'as evita la necesidad de estructuras de dato extra que permitan sincronizar todos los threads a utilizar. Es por eso que se opt'o por hacerlo de esa forma.

