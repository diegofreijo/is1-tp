Para la construcci'on de los escenarios y diagramas de secuencia fue basada en el echo de que contabamos con los mensajes de protocolo los cuales inducen interaci'on y modificaci'on de nuestro modelo.

Tambi'en decidimos hacer que nuestros escenarios sean lo m'as gen'ericos posible siempre y cuando esto no complique la lectura. 

En cuanto a la profundidad para algunos DS mostramos la interacci'on de punta 
a punta estos se encuentran en la secci'on del mismo nombre.
Factorizandolos en 2 o 3 DS's cada uno. El resto de los diagramas, salvo excepciones, comienzan con la llamada al m'etodo de la fachada del modelo.

Dado que la secci'on de recepci'on de pedidos y el despacho es muy parecida en estos diagramas 
decidimos obviarlos (salvo, claro, en los de punta a punta).


Por simpleza no usamos los \textit{ObtenerInstancia} de los Singletons. Estos estar'an implicitos.


Para tratar con los XML asumiremos que contamos con cierta funcionalidad, provista por los objetos, por ejemplo:
\begin{itemize}
\item  Para obtener el atributo \textbf{idMesa} de un XML usamos \textit{XML.idMesa}, donde idMesa es el tag que se encuentra en el XML, y  \textit{XML.idMesa}, nos devoveria un entero almacenado en ese tag.

\item As'i tambi'en para la lista de jugadores, \textit{XML.jugadoresEnMesa} devuelve una \textbf{lista$<$jugadores$>$}.
\end{itemize}

% 
% \escenario{ \textbf{Startup del Casino}
% \begin{itemize}
% \item se inicia el servidor
% \item se leen los achivos de configuraci'on
% \end{itemize}
% 
% }
% 
% \escenario{ \textbf{Startup Cliente}
% \begin{itemize}
% \item se incia el cliente
% \item ???????????????????????????????????
% \end{itemize}
% 
% }

\subsection{Diagramas de punta a punta}

En esta secci'on mostraremos DS desde que se leen los arhivos XML hasta que el mensajero de salida despacha los archivos XML de respuesta.

\subsubsection{Craps}

\escenario{Tirar Dados de punta a punta}{

El usuario puede o no estar en la mesa. El usuario puede o no ser el tirador. En caso de que sea el tirador y est'e en la mesa. Se tiran los dados. Si sali'o alg'un valor de punto este se setea. Hay alguna cantidad de apuestas de alg'un tipo que se resuelven o no con su l'ogica particular, dependiendo del contexto. Estamos en un ``est'an saliendo''. Es una jugada feliz 

}
% 
% $^1$ Resoluci'on de distintas apuestas se ve en otros DS's (en sitio a perder y a venir)
% $^2$ En otro DS se ve la situacion con el caso de que el punto est'e establecido
% $^3$ En otro DS se ve la jugada Todos ponen

\textbf{Aclaraciones: }El pozo feliz se reparte si o si. El servidor de Jugadas no devuelve una feliz mientras el pozo no lleque al m'inimo.
La notificaci'on de que una mesa cambió se pude ver en el DS \textit{apostarCraps}, sin bien en este DS est'a instanciado en una mesa particular, aplica si la mesa fuera gen'erica.


Este DS se dividió en 3 secciones:
\begin{enumerate}
\item Recepcion de pedido: es la recepcion de pedido,  y la respuesta hacia el modulo de comunciación, no se muestra lo que sucede en la llamada TirarCraps (usuario, unXML)
\item  TirarCraps: Hace todo lo concerniente a la validaci'on, no se hace zoom en TirarDados.
\item TirarDados: aqui puede verse lo que pasa cuando se hace un tirar dados de una mesa (Va impreso aparte o digital por el tama~no)
\end{enumerate}

%aca iria una imagen------------------------------------------------------------
\imagenvertical{DS_Craps/TirarDados/RecepcionpPedido.png}{Tirar Dados Recepcion de pedido}{0.5}
\imagenvertical{DS_Craps/TirarDados/tirarCraps.png}{Tirar Dados: TirarCraps}{0.5}
% \imagen{}{TirarDados}{0.5}
Tirar dados \tam

\escenario{Apostar Craps de punta a punta}{

El jugador Cosme Fulanito desea apostar en el juego de craps. La apuesta realizada es a ganar sobre el n'umero cinco. La cantidad apostada es una ficha de valor 20 y dos de valor 10. Cosme est'a en la mesa 25.
}
%Se crea el XML correspondiente con el nombre del archivo "apuestaCraps059999" y se deposita en la carpeta del servidor.

%En el servidor se levanta dicho XML, se lo parsea y se efect'uan las valiaciones correspondientes. A saber:

Para que su apuesta sea efectiva deber'a pasar por las siguientes validaciones:

\begin{enumerate}
 \item Cosme Fulanito es un jugador de la mesa en la que acusa estar;
 \item Los valores de fichas utilizados son v'alidas para 'este d'ia;
 \item Debe tener los fondos suficientes para poder pagar lo que apost'o o que sea un cliente vip;
\end{enumerate}

Si alguna de las validaciones anteriormente mencionadas no se cumple entonces no se le dejar'a realizar la apuesta. En caso contrario si.

Si se cumplen todas las validaciones se agregar'a la apuesta a la mesa y adem'as se notificar'a a todos los dem'as jugadores de la mesa 25 del cambio sucedido.



 \escenario{ PedirEstado de punta a punta }{
Un usuario desea informarse sobre el estado del casino. Se le informar'a s'olo si ha ingresado en el casino, m'as haya si es en modo jugador o en modo observador.

El estado del casino est'a formado por:
\begin{itemize}
 \item La lista de jugadores y observadores ingresados en el casino;
 \item El valor del pozo feliz y del pozo progresivo;
 \item El estado de las mesas del juegos de craps:
	\begin{itemize}
	 \item Los jugadores;
	 \item El 'ultimo tirador y el pr'oximo;
	 \item Si el siguiente es tiro de salida o ya est'a el punto establecido;
	 \item El valor de los dados en el 'ultimo tiro;
	\end{itemize}
 \item El estado de las mesas del juego tragamonedas:
	\begin{itemize}
 	 \item Los jugadores;
 	 \item El valor de los rodillos en el 'ultimo tiro;
 	 \item El 'ultimo tirador y el pr'oximo;
	\end{itemize}
\end{itemize}
 }



\subsubsection{Tragamonedas}
\input{escenarios_tragamonedas_pap.tex}

\subsection{Casino}
 \escenario{Entrar Casino}{
Un usuario desea entrar al casino, puede hacerlo en modo jugador o en modo observador.
Si ya ha ingresado en modo jugador no se lo dejar'a entrar nuevamente. Si est'a en modo observador y desea ingresar en el mismo modo tampoco podr'a hacerlo.

En cambio si quiere entrar como jugador (independientemente de si ingres'o como observador o si no ingres'o) se deber'a validar que sea un usuario autorizado por marketing:

\begin{itemize}
 \item En caso afirmativo quedar'a ingresado en modo jugador.
 \item En caso negativo quedar'a en modo observador o fuera del casino seg'un cual fuese su estado anterior.
\end{itemize}
 }
\imagen{DS_Casino/EntarCasino/DS_EntrarCasinoFueraDelModelo.png}{Entrar casino fuera del Modelo}{0.6}
%aca iria una imagen------------------------------------------------------------
Entrar casino dentro del Modelo \tam

\clearpage



%aca iria una imagen------------------------------------------------------------

\clearpage

\textbf{Pedir Estado Casino}

Un usuario desea informarse sobre el estado del casino. Se le informar'a s'olo si ha ingresado en el casino, m'as haya si es en modo jugador o en modo observador.

El estado del casino est'a formado por:

\begin{itemize}
 \item La lista de jugadores y observadores ingresados en el casino;
 \item El valor del pozo feliz y del pozo progresivo;
 \item El estado de las mesas del juegos de craps:
	\begin{itemize}
	 \item Los jugadores;
	 \item El 'ultimo tirador y el pr'oximo;
	 \item Si el siguiente es tiro de salida o ya est'a el punto establecido;
	 \item El valor de los dados en el 'ultimo tiro;
	\end{itemize}
 \item El estado de las mesas del juego tragamonedas:
	\begin{itemize}
 	 \item Los jugadores;
 	 \item El valor de los rodillos en el 'ultimo tiro;
 	 \item El 'ultimo tirador y el pr'oximo;
	\end{itemize}
\end{itemize}
\tam

%aca iria una imagen------------------------------------------------------------

\clearpage

\textbf{Salir Casino}

Usuario desea salir del casino.

Si ha ingresado como observador no tendr'a ning'un tipo de validaci'on, por consiguiente tampoco problemas.

Si ha ingresado como jugador, para poder salir deber'a estar fuera de toda mesa. Es decir, no puede pretender salir del casino si es que est'a dentro de una mesa jugando.

\imagen{DS_Casino/SalirCasino/DS_SalirCasino.png}{Salir del Casino}{0.6}

%aca iria una imagen------------------------------------------------------------





\subsection{Diagr'amas de que no son de punta a punta}
Dado que la secci'on de recepci'on de pedidos y el despacho es muy parecida en estos diagramas decidimos obviarlo.

\subsubsection{Funcionalidades de inicializaci'on}
Los escenarios aqui presentados son muy gen'ericos.


\escenario{ la Configuraci'on general del Casino}
{
Se setea el valor de las fichas, el saldo del casino y la pasword del aministrador
}
% imagen

\imagenvertical{DS_InicioServidor/DS_InicializarConfiguracion.png}{Inicializar Configuraci'on}{0.5}

\escenario{Jugadores Registrados}{
En este DS se ve como se setea la lista de jugadores registrados del casino
}
\imagenvertical{DS_InicioServidor/DS_InicializarJugadoresRegistrados.png}{Inicializar Jugadores Registrados}{0.4}

% imagen


\escenario{Inicializar Mesas}{
En este DS puede verse como se inicalizan las mesas abiertas, se asigna el observador de cambios.
}
\imagen{DS_InicioServidor/DS_InicializarMesas.png}{Inicializar Mesas}{0.5}
% imagen


\escenario{Inicio del Servidor}{
En este DS se puede ver como se ``enciende'' el servidor,
como se crea el el Obtenerdor de pedidos, el receptor de pedidos de archivos,  

}
\imagenvertical{DS_InicioServidor/DS_InicioServidor.png}{Inicio Servidor}{0.4}


\subsubsection{Funcionalidades generales de los administradores}

En el DS: puede verse la secuencia de un seteo de el modo dirigido de dados.


En el DS: puede verse el pedido de el reporte PedidoReporteDetalleMovimientosPorJugador()





\subsubsection{Tragamonedas}

 \escenario{Tirar Tragamonedas}
 { El usuario puede estar en modo jugador o no. Una vez validado el usuario, ahora jugador, puede ingresar al Tragamonedas o no. Puede o no haber hecho una apuesta. Es una jugada normal. Gira los rodillos. La apuesta se resuelve y se le acreditan creditos si corresponde.}
%imagen
\imagenvertical{DS_Tragamonedas/TirarTragamonedas/Tirar Tragamonedas.png}{Tirar Tragamonedas}{0.5}


 \escenario{Entrar Tragamonedas}{
El jugador puede estar en el casino en modo jugador o no.
Podr'ia estar en alguna mesa. En este escenario mostramos en particular como se hace para enviar los mensajes donde se Acepta o Deniega, en este caso la Entrada al juego.
Se muesta la iteracci'on con los mesajeros. 
}

%imagen


\subsubsection{Craps}
\escenario{ Entrar Craps}{
Este escenario es bastante gen'erico. Se muestra como se valida cada cosa, como actua el sistema en cada caso
y que mensaje de error da.

El usuario puede o no estar en el casino en modo jugador.(incluye modo observador o no haber ingresado)
Puede estar en otra mesa o puede desear crearla.
}

% imagen


\escenario{Resolverse Apuesta de Sitio a Ganar}{
La ronda esta en ``Est'an Saliendo'' sali'o un 4. La apuesta se resuelve, pasa a estar cerrada
}

% imagen



\escenario{Resolverse Apuesta Venir}{
Se estableci'o el punto. Se le paga. La apuesta se cierra
}

% imagen


\escenario{ \textbf{Jugador de Craps haciendo una apuesta}
  \begin{itemize}
   \item El usuario esta en una mesa de craps 
  \item elige un valor de ficha de \$20
  \item elige un valor de ficha de \$15
  \item elige el tipo de apuesta
  \item por cada eleccion se ve un mensaje en el log
  \item si elige una apuesta antes de una ficha  da un error
  \end{itemize}
}





